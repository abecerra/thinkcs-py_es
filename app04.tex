% LaTeX source for textbook ``How to think like a computer scientist''
% Copyright (c)  2001  Allen B. Downey, Jeffrey Elkner, and Chris Meyers.

% Permission is granted to copy, distribute and/or modify this
% document under the terms of the GNU Free Documentation License,
% Version 1.1  or any later version published by the Free Software
% Foundation; with the Invariant Sections being "Contributor List",
% with no Front-Cover Texts, and with no Back-Cover Texts. A copy of
% the license is included in the section entitled "GNU Free
% Documentation License".

% This distribution includes a file named fdl.tex that contains the text
% of the GNU Free Documentation License.  If it is missing, you can obtain
% it from www.gnu.org or by writing to the Free Software Foundation,
% Inc., 59 Temple Place - Suite 330, Boston, MA 02111-1307, USA.
%

\chapter{Lecturas adicionales recomendadas}

¿Así que, hacia adónde ir desde aquí? Hay muchas direcciones para
avanzar, extender su conocimiento de Python específicamente y 
sobre la ciencia de la computación en general.

Los ejemplos en este libro han sido deliberadamente sencillos, 
por esto no han mostrado las capacidades más excitantes de Python.
Aquí hay una pequeña muestra de las extensiones de Python 
y sugerencias de proyectos que las utilizan.

\begin{itemize}

\item La programación de interfaces gráficas de usario, tiene muchos mas
elementos de los que vimos en el último capítulo.

\item La programación en la Web integra a Python con Internet. Por ejemplo,
usted puede construir programas cliente que abran y lean  páginas
remotas (casi) tan fácilmente como abren un archivo en disco. También 
hay módulos en Python que permiten acceder a archivos remotos por medio
de ftp, y otros que posibilitan enviar y recibir correo electrónico. Python
también se usa extensivamente para desarrollar servidores web que
presten servicios.


\item Las bases de datos son como superarchivos, donde la información
se almacena en esquemas predefinidos y las relaciones entre los 
datos permiten navegar por ellos de diferentes formas. Python tiene
varios módulos que facilitan conectar programas a varios motores de
bases de datos, de código abierto y comerciales.

\item La programación con hilos permite ejecutar diferentes hilos
de ejecución dentro de un mismo programa. Si usted ha tenido la 
experiencia de desplazarse al inicio de una página web mientras 
el navegador continúa cargando el resto de ella, entonces tiene
una noción de lo que los hilos pueden lograr.

\item Cuando la preocupación es la velocidad, se pueden escribir
extensiones a Python en un lenguaje compilado como C o C++. Estas
extensiones forman la base de muchos módulos en la biblioteca de
Python. El mecanismo para enlazar funciones y datos es algo 
complejo. La herramienta SWIG (Simplified Wrapper and Interface Generator)
simplifica mucho estas tareas.
\end{itemize}


\section{Libros y sitios web relacionados con Python}

Aquí están las recomendaciones de los autores sobre sitios web:

\begin{itemize}

\item La página web de  \texttt{www.python.org} es el lugar para
empezar cualquier búsqueda de material relacionado con Python.  Encontrará
ayuda, documentación, enlaces a otros sitios, SIG (Special Interest
Groups), y listas de correo a las que se puede unir.

\item EL proyecto Open Book \texttt{www.ibiblio.com/obp} no sólo contiene
este libro en línea, también los libros similares para Java y 
C++ de Allen Downey. Además, está {\em Lessons in Electric Circuits} de Tony R.  Kuphaldt, {\em Getting down with ...} un conjunto de tutoriales (que cubren varios
tópicos en ciencias de la computación) que han sido escritos y editados por 
estudiantes universitarios; {\em Python for Fun}, un conjunto
de casos de estudio en Python escrito por  Chris Meyers, y  {\em The Linux
Cookbook} de Michael Stultz, con 300 páginas de consejos y sugerencias.

\item Finalmente si usted Googlea la cadena ``python -snake -monty'' obtendrá
 unos $337000000$ resultados.

\end{itemize}

%\adjustpage{-1}
%\pagebreak

Aquí hay algunos libros que contienen más material sobre el lenguaje
Python:

\begin{itemize}

\item {\em Core Python Programming}, de Wesley Chun, es un gran libro
de 750 páginas, aproximadamente. La primera parte cubre las características
básicas. La segunda introduce adecuadamente muchos tópicos más avanzados,
incluyendo muchos de los que mencionamos anteriormente.

\item {\em Python Essential Reference}, de David M. Beazley, es un 
pequeño libro, pero contiene mucha información sobre el lenguaje
y la biblioteca estándar. También provee un excelente índice.

\item {\em Python Pocket Reference}, de Mark Lutz, realmente cabe 
en su bolsillo. Aunque no es tan comprensivo como {\em Python Essential
Reference}; es una buena referencia para las funciones y módulos más
usados.  Mark Lutz también es el autor de {\em Programming Python},
uno de los primeros (y más grandes) libros sobre Python que no está
dirigido a novatos. Su libro posterior {\em Learning Python} es más
pequeño y accesible.

\item {\em Python Programming on Win32}, de Mark Hammond y Andy
Robinson, se ``debe tener'' si pretende construir aplicaciones
para el sistema operativo Windows. Entre otras cosas cubre la
integración entre Python y COM, construye una pequeña aplicación con
 wxPython, e incluso realiza guiones que agregan funcionalidad
a aplicaciones como Word y Excel.

\end{itemize}

\section{Libros generales de ciencias de la computación recomendados}

Las siguientes sugerencias de lectura incluyen muchos de los libros
favoritos de los autores. Tratan sobre buenas prácticas de programación
y las ciencias de la computación en general.

\begin{itemize}

\item {\em The Practice of Programming} de Kernighan y Pike no sólo
cubre el diseño y la codificación de algoritmos y estructuras de datos,
sino que también trata la depuración, las pruebas y la optimización
de los programas. La mayoría de los ejemplos está escrita en  C++ y
 Java, no hay ninguno en Python.

\item {\em The Elements of Java Style}, editado por Al Vermeulen, es 
otro libro pequeño que discute algunos de los puntos mas sutiles
de la buena programación, como el uso de buenas convenciones para
los nombres, comentarios, incluso el uso de los espacios en blanco
y la indentación (algo que no es problema en Python). El libro
también cubre la programación por contrato que usa aserciones para
atrapar errores mediante el chequeo de pre y postcondiciones, y la
programación multihilo.

\item {\em Programming Pearls}, de Jon Bentley,  es un libro clásico.
Comprende varios casos de estudio que aparecieron originalmente
en la columna del autor en las  {\em Communications of the ACM}.  Los
estudios examinan los compromisos que hay que tomar cuando se programa
y por qué tan a menudo es una mala idea apresurarse con la primera
idea que se tiene para desarrollar un programa. Este libro es uno poco más 
viejo que los otros (1986), así que los ejemplos están escritos en lenguajes
más viejos. Hay muchos problemas para resolver, algunos traen pistas 
y otros su solución. Este libro fue muy popular, incluso hay un 
segundo volumen.

\item {\em The New Turing Omnibus}, de A.K Dewdney, hace una amable
introducción a 66 tópicos en ciencias de la computación, que
van desde la computación paralela hasta los virus de computador,
desde escanografías hasta algoritmos genéticos. Todos son cortos
e interesantes. Un libro anterior de  Dewdney {\em The Armchair Universe}
es una colección de artículos de su columna {\em Computer Recreations} en 
la revista {\em Scientific American (Investigación y Ciencia)},
estos libros son una rica fuente de ideas para emprender proyectos.

\item {\em Turtles, Termites and Traffic Jams}, de Mitchel Resnick,
trata sobre el poder de la descentralización y cómo el comportamiento
complejo puede emerger de la simple actividad coordinada de una multitud
de agentes. Introduce el lenguaje StarLogo que permite escribir
programas multiagentes. Los programas examinados demuestran el
comportamiento complejo agregado, que a menudo es contraintuitivo. 
Muchos de estos programas fueron escritos por estudiantes de colegio
y universidad. Programas similares pueden escribirse en Python
usando hilos y gráficos simples.

\item {\em G\"{o}del, Escher y Bach}, de Douglas Hofstadter.  Simplemente,
si usted ha encontrado magia en la recursión, también la encontrará en éste
best seller. Uno de los temas que trata Hofstadter es el de los 
 ``ciclos extraños'', en los que los patrones evolucionan y ascienden hasta
que se encuentran a sí mismos otra vez. La tesis de Hofstadter es que 
esos ``ciclos extraños'' son una parte esencial de lo que separa lo 
animado de lo inanimado. Él muestra patrones como éstos en la 
música de Bach, los cuadros de Escher y el teorema de incompletitud de
 Gödel.

\end{itemize}
