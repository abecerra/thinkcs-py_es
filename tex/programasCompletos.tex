
\chapter{Programas completos}

\section{Clase punto}

\begin{verbatim}
class Punto:
  def __init__(self, x=0, y=0):
    self.x = x
    self.y = y

  def __str__(self):
    return '(' + str(self.x) + ',' + str(self.y) + ')'

  def __add__(self, otro):
    return Punto(self.x + otro.x, self.y + otro.y)

  def __sub__(self, otro):
    return Punto(self.x - otro.x, self.y - otro.y)

  def __mul__(self, otro):
    return self.x * otro.x + self.y * otro.y

  def __rmul__(self, otro):
    return Punto(otro * self.x, otro * self.y)

  def invertir(self):
    self.x, self.y = self.y, self.x

  def DerechoYAlReves(derecho):
    from copy import copy
    alreves = copy(derecho)
    alreves.invertir()
    print(str(derecho) + str(alreves))
\end{verbatim}

\section{Clase hora}

\begin{verbatim}
class Hora:
  def __init__(self, hora=0, minutos=0, segundos=0):
    self.hora = hora
    self.minutos = minutos
    self.segundos = segundos

  def __str__(self):
    return str(self.hora) + ":" + str(self.minutos) + 
           ":" + str(self.segundos)

  def convertirAsegundoss(self):
    minutos = self.hora * 60 + self.minutos
    segundos = self.minutos * 60 + self.segundos
    return segundos

  def incrementar(self, segs):
    segs = segs + self.segundos

    self.hora = self.hora + segs/3600
    segs = segs % 3600
    self.minutos = self.minutos + segs/60
    segs = segs % 60
    self.segundos = segs

def crearHora(segs):
  H = Hora()
  H.hora = segs/3600
  segs = segs - H.hora * 3600
  H.minutos = segs/60
  segs = segs - H.minutos * 60
  H.segundos = segs
  return H
\end{verbatim}

\section{Cartas, mazos y juegos}

\begin{verbatim}
import random

class Carta:
  listaFiguras = ["Treboles", "Diamantes", 
                  "Corazones", "Picas"]
  listaValores = [ "narf", "As", "2", "3", "4", "5", "6", 
               "7","8", "9", "10","Jota", "Reina", "Rey"]

  def __init__(self, figura=0, valor=0):
    self.figura = figura
    self.valor = valor

  def __str__(self):
    return self.listaValores[self.valor] + " de " 
           + self.listaFiguras[self.figura]

  def __cmp__(self, otro):
    # revisa las figuras
    if self.figura > otro.figura: 
       return 1
    if self.figura < otro.figura: 
       return -1
    # las figuras son iguales, se chequean los valores
    if self.valor > otro.valor: 
      return 1
    if self.valor < otro.valor: 
      return -1
    # los valores son iguales,  hay empate
    return 0

class Mazo:
  def __init__(self):
    self.Cartas = []
    for figura in range(4):
      for valor in range(1, 14):
        self.Cartas.append(Carta(figura, valor))

  def imprimirMazo(self):
    for Carta in self.Cartas:
      print(Carta)

  def __str__(self):
    s = ""
    for i in range(len(self.Cartas)):
      s = s + " "*i + str(self.Cartas[i]) + "\n"
    return s

  def barajar(self):
    import random
    nCartas = len(self.Cartas)
    for i in range(nCartas):
      j = random.randrange(i, nCartas)
      [self.Cartas[i], self.Cartas[j]] = [self.Cartas[j], 
                                          self.Cartas[i]]

  def eliminarCarta(self, Carta):
    if Carta in self.Cartas:
      self.Cartas.remove(Carta)
      return 1
    else: return 0

  def entregarCarta(self):
    return self.Cartas.pop()

  def estaVacio(self):
    return (len(self.Cartas) == 0)

  def repartir(self, manos, nCartas=999):
    nmanos = len(manos)
    for i in range(nCartas):
      if self.estaVacio(): 
         break    # rompe el ciclo si no hay cartas
      # quita la carta del tope
      Carta = self.entregarCarta()      
      # quien tiene el proximo turnoo?
      mano = manos[i % nmanos]    
      # agrega la carta a la mano
      mano.agregarCarta(Carta)         

class mano(Mazo):
  def __init__(self, nombre=""):
    self.Cartas = []
    self.nombre = nombre

  def agregarCarta(self,Carta) :
    self.Cartas.append(Carta)

  def __str__(self):
    s = "mano " + self.nombre
    if self.estaVacio():
      s = s + " esta vacia\n"
    else:
      s = s + " contiene\n"
    return s + Mazo.__str__(self)

class JuegoCartas:
  def __init__(self):
    self.Mazo = Mazo()
    self.Mazo.barajar()

class ManoJuegoSolterona(mano):
  def eliminarParejas(self):
    cont = 0
    originalCartas = self.Cartas[:]
    for carta in originalCartas:
      m = Carta(3-carta.figura, carta.valor)
      if m in self.Cartas:
        self.Cartas.remove(carta)
        self.Cartas.remove(m)
        print("mano %s: %s parejas %s" % 
                 (self.nombre,carta,m))
        cont = cont+1
    return cont

class JuegoSolterona(JuegoCartas):
  def jugar(self, nombres):
    # elimina la reina de treboles
    self.Mazo.eliminarCarta(Carta(0,12))

    # crea manos con base en los nombres
    self.manos = []
    for nombre in nombres : 
        self.manos.append(ManoJuegoSolterona(nombre))

    # reparte las Cartas
    self.Mazo.repartir(self.manos)
    print("---------- Cartas se han repartido")
    self.imprimirmanos()

    # eliminar parejas iniciales
    parejas = self.eliminarParejas()
    print("----- parejas descartadas, empieza el juego")
    self.imprimirmanos()

    # jugar hasta que se eliminan 50 cartas
    turno = 0
    nummanos = len(self.manos)
    while parejas < 25:
      parejas = parejas + self.jugarUnturno(turno)
      turno = (turno + 1) % nummanos

    print("---------- Juego Terminado")
    self.imprimirmanos ()

  def eliminarParejas(self):
    cont = 0
    for mano in self.manos:
      cont = cont + mano.eliminarParejas()
    return cont

  def jugarUnturno(self, i):
    if self.manos[i].estaVacio():
      return 0
    vecino = self.encontrarvecino(i)
    cartaEscogida = self.manos[vecino].entregarCarta()
    self.manos[i].agregarCarta(cartaEscogida)
    print("mano", self.manos[i].nombre, 
          "escogi�", cartaEscogida)
    count = self.manos[i].eliminarParejas()
    self.manos[i].barajar()
    return count

  def encontrarvecino(self, i):
    nummanos = len(self.manos)
    for siguiente in range(1,nummanos):
      vecino = (i + siguiente) % nummanos
      if not self.manos[vecino].estaVacio():
        return vecino

  def imprimirmanos(self) :
    for mano in self.manos :
      print(mano)
\end{verbatim}

\section{Listas enlazadas}
\begin{verbatim}

\end{verbatim}
\begin{verbatim}
def imprimirlista(Nodo) :
  while Nodo :
    print(Nodo),
    Nodo = Nodo.siguiente
  print()

def imprimirAlReves(lista) :
  if lista == None : 
      return
  cabeza = lista
  resto = lista.siguiente
  imprimirAlReves(resto)
  print(cabeza),

def imprimirAlRevesBien(lista) :
  print("("),
  if lista != None :
    cabeza = lista
    resto = lista.siguiente
    imprimirAlReves(resto)
    print(cabeza),
  print(")"),

def eliminarSegundo(lista) :
  if lista == None : return
  first  = lista
  second = lista.siguiente
  first.siguiente = second.siguiente
  second.siguiente = None
  return second

class Nodo :
  def __init__(self, carga=None) :
    self.carga = carga
    self.siguiente  = None

  def __str__(self) :
    return str(self.carga)

  def imprimirAlReves(self) :
    if self.siguiente != None :
      resto = self.siguiente
      resto.imprimirAlReves()
    print(self.carga),

class ListaEnlazada :
  def __init__(self) :
    self.numElementos = 0
    self.cabeza   = None

  def imprimirAlReves(self) :
    print("("),
    if self.cabeza != None :
      self.cabeza.imprimirAlReves()
    print(")"),

  def agregarAlPrincipio(self, carga) :
    nodo = Nodo(carga)
    nodo.siguiente = self.cabeza
    self.cabeza = nodo
    self.numElementos = self.numElementos + 1
\end{verbatim}

\section{Clase pila}

\begin{verbatim}
class Pila:              
  def __init__(self):
    self.items = []

  def meter(self, item):
    self.items.append(item)

  def sacar(self):
    return self.items.pop()

  def estaVacia(self):
    return(self.items == [])

def evalPostfija(expr) :
  import re
  expr = re.split("([^0-9])", expr)
  pila = Pila()
  for lexema in expr :
    if  lexema == '' or lexema == ' ':
      continue
    if  lexema == '+' :
      suma = pila.sacar() + pila.sacar()
      pila.meter(suma)
    elif lexema == '*' :
      producto = pila.sacar() * pila.sacar()
      pila.meter(producto)
    else :
      pila.meter(int(lexema))
  return pila.sacar()
\end{verbatim}
\begin{verbatim}

\end{verbatim}

\section{Colas PEPS y de colas de prioridad}

\begin{verbatim}
class Cola:
  def __init__(self):
    self.numElementos = 0
    self.primero = None

  def estaVacia(self):
    return (self.numElementos == 0)

  def meter(self, carga):
    nodo = Nodo(carga) 
    nodo.siguiente = None
    if self.primero == None:
       # si esta vacia este nodo sera el primero
       self.primero = nodo
    else:
       # encontrar el ultimo nodo
       ultimo = self.primero
       while ultimo.siguiente:
	 ultimo = ultimo.siguiente
	 # pegar el nuevo
	 ultimo.siguiente = nodo
       self.numElementos = self.numElementos + 1

  def sacar(self):
    carga = self.primero.carga
    self.primero = self.primero.siguiente
    self.numElementos = self.numElementos - 1
    return carga

class ColaMejorada:
  def __init__(self):
    self.numElementos = 0
    self.primero = None
    self.ultimo = None
        
  def estaVacia(self):
    return (self.numElementos == 0)

  def meter(self, carga):
    nodo = Nodo(carga) 
    nodo.siguiente = None
    if self.numElementos == 0:
      # si esta vacia, el nuevo nodo 
      # es primero y ultimo
      self.primero = self.ultimo = nodo
    else:
      # encontrar el ultimo nodo
      ultimo = self.ultimo
      # pegar el nuevo nodo
      ultimo.siguiente = nodo
      self.ultimo = nodo
      self.numElementos = self.numElementos + 1

    def sacar(self):
      carga = self.primero.carga
      self.primero = self.primero.siguiente
      self.numElementos = self.numElementos - 1
      if self.numElementos == 0:
        self.ultimo = None
      return carga

class ColaPrioridad:
  def __init__(self):
    self.items = []
      
  def estaVacia(self):
    return self.items == []
  
  def meter(self, item):
    self.items.append(item)

  def eliminar(self) :
    maxi = 0
    for i in range(1,len(self.items)) :
      if self.items[i] > self.items[maxi] :
	maxi = i
    item = self.items[maxi]
    self.items[maxi:maxi+1] = []
    return item

  def sacar(self):
    maxi = 0
    for i in range(0,len(self.items)): 
      if self.items[i] > self.items[maxi]:
	maxi = i
      item = self.items[maxi]
      self.items[maxi:maxi+1] = []
      return item


class golfista:
  def __init__(self, nombre, puntaje):
    self.nombre = nombre
    self.puntaje= puntaje
    
  def __str__(self):
    return "%-16s: %d" % (self.nombre, self.puntaje)

  def __cmp__(self, otro):
    # el menor tiene mayor prioridad
    if self.puntaje < otro.puntaje: return 1
      if self.puntaje > otro.puntaje: return -1
        return 0
\end{verbatim}

\section{Árboles}

\begin{verbatim}
class Arbol:
  def __init__(self, carga, izquierdo=None, derecho=None):
    self.carga = carga
    self.izquierdo = izquierdo
    self.derecho = derecho
      
  def __str__(self):
    return str(self.carga)
  
  def obtenerizquierdo(self):
    return self.izquierdo
  
  def obtenerderecho(self):
    return self.derecho
  
  def obtenercarga(self):
    return self.carga
    
  def asignarcarga(self, carga):
    self.carga = carga
  
  def asignarizquierdo(self, i):
    self.izquierdo = i
    
  def asignarderecho(self, d):
    self.derecho = d

def total(arbol):
  if arbol.izquierdo == None or arbol.derecho== None:
    return arbol.carga
  else:
    return total(arbol.izquierdo) + total(arbol.derecho) + 
             arbol.carga
   
def imprimirarbol(arbol):
  if arbol == None:
    return 
  else:
    print(arbol.carga)
    imprimirarbol(arbol.izquierdo)
    imprimirarbol(arbol.derecho)

def imprimirarbolPostorden(arbol):
  if arbol == None:
    return
  else:
    imprimirarbolPostorden(arbol.izquierdo)
    imprimirarbolPostorden(arbol.derecho)
    print(arbol.carga)   

def imprimirabolEnOrden(arbol):
  if arbol == None:
    return
  imprimirabolEnOrden(arbol.izquierdo)
  print(arbol.carga,imprimirabolEnOrden(arbol.derecho))

def imprimirarbolSangrado(arbol, nivel=0):
  if arbol == None:
    return
  imprimirarbolSangrado(arbol.derecho, nivel+1)
  print(" "*nivel + str(arbol.carga))
  imprimirarbolSangrado(arbol.izquierdo, nivel+1)
\end{verbatim}

\section{Árboles de expresiones}

\begin{verbatim}
def obtenerLexema(listaLexemas, esperado):
  if listaLexemas[0] == esperado:
      del listaLexemas[0]
      return 1
  else:
      return 0

def obtenerNumero(listaLexemas):
  x = listaLexemas[0]
  if type(x) != type(0):
      return None
  del listaLexemas[0]
  return arbol (x, None, None)


def obtenerProducto(listaLexemas):
  a = obtenerNumero(listaLexemas)
  if obtenerLexema(listaLexemas, "*"):
    b = obtenerProducto(listaLexemas) 
    return arbol ("*", a, b)
  else:
    return a

def obtenerNumero(listaLexemas):
  if obtenerLexema(listaLexemas, "("):
    # obtiene la subexpresi�n
    x = obtenerSuma(listaLexemas) 
    # elimina los par�ntesis
    obtenerLexema(listaLexemas, ")") 
    return x
  else:
    x = listaLexemas[0]
    if type(x) != type(0):
      return None
    listaLexemas[0:1] = []
   return Arbol (x, None, None)  
\end{verbatim}

\section{Adivinar el animal}

\begin{verbatim}
def animal():
  # Un solo nodo
  raiz = Arbol("pajaro")
      # Hasta que el usuario salga
  while True:
    print()
    if not si("Esta pensando en un animal? "):
	break
    # Recorrer el arbol
    arbol = raiz
    while arbol.obtenerizquierdo() != None:
      pregunta = arbol.obtenercarga() + "? "
      if si(pregunta):
	  arbol = arbol.obtenerderecho()
      else:
	  arbol = arbol.obtenerizquierdo()
    # conjetura
    conjetura = arbol.obtenercarga()
    pregunta = "�Es un" + conjetura + "? "
    if si(pregunta):
      print("�Soy el mejor!")
      continue
    # obtener mas informacion
    pregunta = "�Cual es el nombre el animal? "
    animal = raw_input(pregunta)
    pregunta = "�Que pregunta permitiria distinguir 
		un %s de un %s? "
    q = raw_input(pregunta % (animal,conjetura))
    # agrega un nuevo nodo arbol
    arbol.asignarcarga(q)
    pregunta = "�Si el animal fuera %s 
		la respuesta ser�a? "
    if si(pregunta % animal):
      arbol.asignarizquierdo(Arbol(conjetura))
      arbol.asignarderecho(Arbol(animal))
    else:
      arbol.asignarizquierdo(Arbol(animal))
      arbol.asignarderecho(Arbol(conjetura))

def si(preg):
  from string import lower
  r = lower(raw_input(preg))
  return (r[0] == 's')
\end{verbatim}

\section{Clase \texttt{Fraccion}}

\texttt{}
\begin{verbatim}[language=Python,showstringspaces=false,tabsize=4]
class Fraccion:
  def __init__(self,numerador,denominador=1):
    self.numerador = numerador
    self.denominador = denominador

  def __str__(self):
    return "%d/%d" % (self.numerador, self.denominador)
  
  def __mul__(self, otro):
    if type(otro) == type(5):
      otro = Fraccion(otro)
    return Fraccion(self.numerador * otro.numerador,
		    self.denominador * otro.denominador)
  
  __rmul__ = __mul__ 

  #suma de fracciones
  def __add__(self, otro):
    if type(otro) == type(5):
      otro = Fraccion(otro)
    return Fraccion(self.numerador * otro.denominador +
		    self.denominador * otro.numerador,
		    self.denominador * otro.denominador)
  
  __radd__ = __add__
  
  def __init__(self, numerador, denominador=1):      
    g=MDC(numerador,denominador)
    self.numerador=numerador / g
    self.denominador=denominador / g
  
  def __cmp__(self, otro):
    dif = (self.numerador * otro.denominador -
	   otro.numerador * self.denominador)
    return dif

def MDC (m,n):
      if m%n==0:
	  return n
      else:
	  return MDC(n,m%n)
\end{verbatim}

