
\chapter{Lista de los colaboradores}

Este libro vino a la luz debido a una colaboración que no sería posible
sin la licencia de documentación libre de la GNU (Free Documentation
License). Quisiéramos agradecer a la Free Software Foundation por
desarrollar esta licencia y, por supuesto, por ponerla a nuestra disposición.

Nosotros queremos agradecer a los mas de 100 juiciosos y reflexivos
lectores que nos han enviado sugerencias y correcciones durante los
años pasados. En el espíritu del software libre, decidimos expresar
nuestro agradecimiento en la forma de una lista de colaboradores.
Desafortunadamente, esta lista no está completa, pero estamos haciendo
nuestro mejor esfuerzo para mantenerla actualizada.

Si tiene la oportunidad de leer la lista, tenga en cuenta que cada
persona mencionada aquí le ha ahorrado a usted y a todos los lectores
subsecuentes la confusión debida a un error técnico o debido a una
explicación confusa, solo por enviarnos una nota.

Después de tantas correcciones, todavía pueden haber errores en este
libro. Si ve uno, esperamos que tome un minuto para contactarnos.
El correo electrónico es \texttt{feedback@thinkpython.com}. Si hacemos
un cambio debido a su sugerencias, usted aparecerá en la siguiente
versión de la lista de colaboradores (a menos que usted pida ser omitido).
Gracias!
\begin{itemize}
\item Lloyd Hugh Allen remitió una corrección a la Sección 8.4. %He can be reached at: \texttt{lha2@columbia.edu}
\item Yvon Boulianne corrigió un error semántico en el Capítulo 5. %She can be reached at: \texttt{mystic@monuniverse.net}
\item Fred Bremmer hizo una corrección en la Sección 2.1. %He can be reached at:  \texttt{Fred.Bremmer@ubc.cu}
\item Jonah Cohen escribió los guiones en Perl para convertir la fuente
LaTeX, de este libro, a un maravilloso HTML.

%His Web page is \texttt{jonah.ticalc.org}%and his email is \texttt{JonahCohen@aol.com}
\item Michael Conlon remitió una corrección de gramática en el Capítulo
3 una mejora de estilo en el Capítulo 2, e inició la discusión de
los aspectos técnicos de los intérpretes.

%Michael can be reached at: \texttt{michael.conlon@sru.edu}
\item Benoit Girard envió una corrección a un extraño error en la Sección
5.6.

%Benoit can be reached at:%\texttt{benoit.girard@gouv.qc.ca}
\item Courtney Gleason y Katherine Smith escribieron \texttt{horsebet.py},
que se usaba como un caso de estudio en una versión anterior de este
libro. Su programa se puede encontrar en su website.

%Courtney can be reached at: {\tt%orion1558@aol.com}
\item Lee Harr sometió más correcciones de las que tenemos espacio para
enumerar aquí, y, por supuesto, debería ser listado como uno de los
editores principales del texto.

%He can be reached at: \texttt{missive@linuxfreemail.com}
\item James Kaylin es un estudiante usando el texto. Él ha enviado numerosas
correcciones.

%James can be reached by email at: \texttt{Jamarf@aol.com}
\item David Kershaw arregló la función errónea \texttt{imprimaDoble} en
la Sección 3.10.

%He can be reached at: \texttt{david\_kershaw@merck.com}
\item Eddie Lam ha enviado numerosas correcciones a los Capítulos 1, 2,
y 3. Él corrigió el Makefile para que creara un índice, la primera
vez que se compilaba el documento, y nos ayudó a instalar un sistema
de control de versiones.

%Eddie can be reached at:%\texttt{nautilus@yoyo.cc.monash.edu.au}
\item Man-Yong Lee envió una corrección al código de ejemplo en la Sección
2.4.

%He can be reaced at: \texttt{yong@linuxkorea.co.kr}
\item David Mayo notó que la palabra ``inconscientemente'' debe cambiarse
por ``subconscientemente''.

%David can be reached at:\texttt{bdbear44@netscape.net}
\item Chris McAloon envió varias correcciones a las Secciones 3.9 y 3.10.

%He can be reached at: \texttt{cmcaloon@ou.edu}
\item Matthew J. Moelter ha sido un contribuidor de mucho tiempo quien remitió
numerosas correcciones y sugerencias al libro.

%He can be reached at:%\texttt{mmoelter@calpoly.edu}
\item Simon Dicon Montford reportó una definición de función que faltaba
y varios errores en el Capítulo 3. Él también encontró errores en
la función \texttt{incrementar} del Capítulo 13.

%He can be reached at: \texttt{dicon@bigfoot.com}
\item John Ouzts corrigió la definición de ``valor de retorno'' en el
Capítulo 3.

%He can be reached at: \texttt{jouzts@bigfoot.com}
\item Kevin Parks envió sugerencias valiosas para mejorar la distribución
del libro.

%He can be reached at: \texttt{cpsoct@lycos.com}
\item David Pool envió la corrección de un error en el glosario del Capítulo
1 y palabras de estímulo.

%He can be reached at: \texttt{pooldavid@hotmail.com}
\item Michael Schmitt envió una corrección al capítulo de archivos y excepciones.

%He can be reached at: \texttt{ipv6\_128@yahoo.com}
\item Robin Shaw notó un error en la Sección 13.1, donde la función imprimirHora
se usaba en un ejemplo sin estar definida.

%Robin can be reached at: \texttt{randj@iowatelecom.net}
\item Paul Sleigh encontró un error en el Capítulo 7 y otro en los guiones
de Perl de Jonah Cohen que generan HTML a partir de LaTeX.

%He can be reached at: \texttt{bat@atdot.dotat.org}

%\item Christopher Smith is a computer science teacher at the Blake%School in Minnesota who teaches Python to his beginning students.

%He can be reached at: \texttt{csmith@blakeschool.org or smiles@saysomething.com}
\item Craig T. Snydal está probando el texto en un curso en Drew University.
El ha aportado varias sugerencias valiosas y correcciones.

%and can be reached at: \texttt{csnydal@drew.edu}
\item Ian Thomas y sus estudiantes están usando el texto en un curso de
programación. Ellos son los primeros en probar los capítulos de la
segunda mitad del libro y han enviado numerosas correcciones y sugerencias.

%Ian can be reached at: \texttt{ithomas@sd70.bc.ca}
\item Keith Verheyden envió una corrección al Capítulo 3.

%He can be reached at: \texttt{kverheyd@glam.ac.uk}
\item Peter Winstanley descubrió un viejo error en nuestro Latín, en el
capítulo 3.

%He can be reached at:\texttt{Peter.Winstanley@scotland.gsi.gov.uk} 
\item Chris Wrobel hizo correcciones al código en el capítulo sobre archivos,
E/S y excepciones.

%He can be reached at: \texttt{ferz980@yahoo.com}
\item Moshe Zadka hizo contribuciones inestimables a este proyecto. Además
de escribir el primer bosquejo del capítulo sobre Diccionarios, también
proporcionó una dirección continua en los primeros años del libro.

%He can be reached at: \texttt{moshez@math.huji.ac.il}
\item Christoph Zwerschke envió varias correcciones y sugerencias pedagógicas,
y explicó la diferencia entre {\em gleich} y {\em selbe}.
\item James Mayer nos envió un montón de errores tipográficos y de deletreo,
incluyendo dos en la lista de colaboradores

% james.mayer@acm.org
\item Hayden McAfee descubrió una inconsistencia potencialmente confusa
entre dos ejemplos.
\item Angel Arnal hace parte de un equipo internacional de traductores que
trabajan en la versión española del texto. Él también ha encontrado
varios errores en la versión inglesa.
\end{itemize}

