
\chapter{Clases y funciones}

\label{time} \index{función} \index{método}

\section{Hora}

Como otro ejemplo de tipo de dato definido por el usuario definiremos
una clase llamada \texttt{Hora}:\inputencoding{latin9}
\begin{lstlisting}
class Hora:
  pass
\end{lstlisting}
\inputencoding{utf8}
Ahora podemos crear un nuevo objeto \texttt{Hora} y asignarle atributos
para las horas, minutos y segundos:\inputencoding{latin9}
\begin{lstlisting}
tiempo = Hora()
tiempo.hora = 11
tiempo.minutos = 59
tiempo.segundos = 30
\end{lstlisting}
\inputencoding{utf8}
El diagrama para el objeto \texttt{Hora} luce así:

\beforefig \centerline{\includegraphics{illustrations/time}} \afterfig

\section{Funciones puras}

\index{función pura} \index{tipo función!pura}

En las siguientes secciones escribiremos dos versiones de una función
denominada \texttt{sumarHoras}, que calcule la suma de dos \texttt{Horas}.
Esto demostrará dos clases de funciones: las puras y los modificadores.

La siguiente es una versión de \texttt{sumarHoras}:

\inputencoding{latin9}\begin{lstlisting}
def sumarHoras(t1, t2):
  sum = Hora()
  sum.hora = t1.hora + t2.hora
  sum.minutos = t1.minutos + t2.minutos
  sum.segundos = t1.segundos + t2.segundos
  return sum
\end{lstlisting}
\inputencoding{utf8} La función crea un nuevo objeto \texttt{Hora}, inicializa sus atributos
y retorna una referencia hacia el nuevo objeto. Esto se denomina \textbf{función
pura}, porque no modifica ninguno de los objetos que se le pasaron
como parámetro ni tiene efectos laterales, como desplegar un valor
o leer entrada del usuario.

Aquí hay un ejemplo de uso de esta función. Crearemos dos objetos
\texttt{Hora}: \texttt{horaPan}, que contiene el tiempo que le toma
a un panadero hacer pan y \texttt{horaActual}, que contiene la hora
actual. Luego usaremos \texttt{sumarHoras} para averiguar a qué hora
estará listo el pan. Si no ha terminado la función \texttt{imprimirHora}
aún, adelántese a la Sección \ref{printTime} antes de intentar esto:

\inputencoding{latin9}\begin{lstlisting}
>>> horaActual = Hora()
>>> horaActual.hora = 9
>>> horaActual.minutos = 14
>>> horaActual.segundos =  30

>>> horaPan = Hora()
>>> horaPan.hora =  3
>>> horaPan.minutos =  35
>>> horaPan.segundos =  0

>>> horaComer = sumarHoras(horaActual, horaPan)
>>> imprimirHora(horaComer)
\end{lstlisting}
\inputencoding{utf8} La salida de este programa es \texttt{12:49:30}, que está correcta.
Por otro lado, hay casos en los que no funciona bien. ¿Puede pensar
en uno?

El problema radica en que esta función no considera los casos donde
el número de segundos o minutos suman más de sesenta. Cuando eso ocurre
tenemos que ``acarrear'' los segundos extra a la columna de minutos.
También puede pasar lo mismo con los minutos.

Aquí hay una versión correcta:\inputencoding{latin9}
\begin{lstlisting}
def sumarHoras(t1, t2):
  sum = Hora()
  sum.hora = t1.hora + t2.hora
  sum.minutos = t1.minutos + t2.minutos
  sum.segundos = t1.segundos + t2.segundos

  if sum.segundos >= 60:
    sum.segundos = sum.segundos - 60
    sum.minutos = sum.minutos + 1

  if sum.minutos >= 60:
    sum.minutos = sum.minutos - 60
    sum.hora = sum.hora + 1

  return sum
\end{lstlisting}
\inputencoding{utf8}
Aunque ahora ha quedado correcta, ha empezado a agrandarse. Más adelante
sugeriremos un enfoque alternativo que produce un código más corto.

\section{Modificadoras}

\label{increment} \index{modificadora} \index{tipo función!modificadora}

A veces es deseable que una función modifique uno o varios de los
objetos que recibe como parámetros. Usualmente, el código que hace
el llamado a la función conserva una referencia a los objetos que
está pasando, así que cualquier cambio que la función les haga será
evidenciado por dicho código. Este tipo de funciones se denominan
\textbf{modificadoras}.

\texttt{incrementar}, que agrega un número de segundos a un objeto
\texttt{Hora}, se escribiría más naturalmente como función modificadora.
Un primer acercamiento a la función luciría así:\inputencoding{latin9}
\begin{lstlisting}
def incrementar(h, segundos):
  h.segundos = h.segundos + segundos

  if h.segundos >= 60:
    h.segundos = h.segundos - 60
    h.minutos = h.minutos + 1

  if h.minuto >= 60:
    h.minutos = h.minutos - 60
    h.hora = h.hora + 1

  return h
\end{lstlisting}
\inputencoding{utf8}
La primera línea ejecuta la operación básica, las siguientes consideran
los casos especiales que ya habíamos visto.

¿Es correcta esta función? ¿Que pasa si el parámetro \texttt{segundos}
es mucho más grande que sesenta? En ese caso, no sólo es suficiente
añadir uno, tenemos que sumar de uno en uno hasta que \texttt{segundos}
sea menor que sesenta. Una solución consiste en reemplazar las sentencias
\texttt{if} por sentencias \texttt{while}:\inputencoding{latin9}
\begin{lstlisting}
def incrementar(hora, segundos):
  hora.segundos = hora.segundos + segundos

  while hora.segundos >= 60:
    hora.segundos = hora.segundos - 60
    hora.minutos = hora.minutos + 1

  while hora.minutos >= 60:
    hora.minutos = hora.minutos - 60
    hora.hora = hora.hora + 1

  return hora

  time.segundos = time.segundos + segundos
\end{lstlisting}
\inputencoding{utf8}
Ahora, la función sí es correcta, aunque no sigue el proceso más eficiente.

\section{¿Cual es el mejor estilo?}

\index{estilo de programación funcional}

Todo lo que puede hacerse con modificadoras también se puede hacer
con funciones puras. De hecho, algunos lenguajes de programación sólo
permiten funciones puras. La evidencia apoya la tesis de que los programas
que usan solamente funciones puras se desarrollan más rápido y son
menos propensos a errores que los programas que usan modificadoras.
Sin embargo, las funciones modificadoras, a menudo, son convenientes
y, a menudo, los programas funcionales puros son menos eficientes.

En general, le recomendamos que escriba funciones puras cada vez que
sea posible y recurrir a las modificadoras solamente si hay una ventaja
en usar este enfoque. Esto se denomina un \textbf{estilo de programación
funcional}.

\section{Desarrollo con prototipos vs. planificación}

\label{convert} \index{desarrollo con prototipos}

En este capítulo mostramos un enfoque de desarrollo de programas que
denominamos \textbf{desarrollo con prototipos}. Para cada problema
escribimos un bosquejo (o prototipo) que ejecutará el cálculo básico
y lo probará en unos cuantos casos de prueba, corrigiendo errores
a medida que surgen.

Aunque este enfoque puede ser efectivo, puede conducirnos a código
innecesariamente complicado —ya que considera muchos casos especiales—y
poco confiable—ya que es difícil asegurar que hemos descubierto todos
los errores.

Una alternativa es el \textbf{desarrollo planificado}, en el que la
profundización en el dominio del problema puede darnos una comprensión
profunda que facilita bastante la programación. En el caso anterior,
comprendimos que un objeto \texttt{Hora} realmente es un número de
tres dígitos en base 60! El componente \texttt{segundos} contiene
las ``unidades,'' el componente \texttt{minutos} la ``columna de
sesentas,'' y el componente \texttt{hora} contiene la ``columna
de tres mil seiscientos.''

Cuando escribimos \texttt{sumarHoras} e \texttt{incrementar}, realmente
estábamos haciendo una suma en base 60, razón por la cual teníamos
que efectuar un acarreo de una columna a la siguiente.

Esta observación sugiere otro enfoque al problema—podemos convertir
un objeto \texttt{Hora} en un número único y aprovecharnos del hecho
de que el computador sabe realizar aritmética. La siguiente función
convierte un objeto \texttt{Hora} en un entero:\inputencoding{latin9}
\begin{lstlisting}
def convertirASegundos(t):
  minutos = t.hora * 60 + t.minutos
  segundos = minutos * 60 + t.segundos
  return segundos
\end{lstlisting}
\inputencoding{utf8}
Ahora necesitamos una forma de convertir desde entero a un objeto
\texttt{Hora}:\inputencoding{latin9}
\begin{lstlisting}
def crearHora(segundos):
  h = Hora()
  h.hora = segundos/3600
  segundos = segundos - h.hora * 3600
  h.minutos = segundos/60
  segundos = segundos - h.minutos * 60
  h.segundos = segundos
  return h
\end{lstlisting}
\inputencoding{utf8}
Usted debe pensar unos minutos para convencerse de que esta técnica
sí convierte, de una base a otra, correctamente. Asumiendo que ya
está convencido, se pueden usar las funciones anteriores para reescribir
\texttt{sumarHoras}:

\inputencoding{latin9}\begin{lstlisting}
def sumarHoras(t1, t2):
  segundos = convertirASegundos(t1) + convertirASegundos(t2)
  return crearHora(segundos)
\end{lstlisting}
\inputencoding{utf8} Esta versión es mucho más corta que la original, y es mucho más fácil
de demostrar que es correcta (asumiendo, como de costumbre, que las
funciones que llama son correctas).

\section{Generalización}

\index{generalización}

Desde cierto punto de vista, convertir de base 60 a base 10 y viceversa
es más difícil que calcular solamente con horas. La conversión de
bases es más abstracta, mientras que nuestra intuición para manejar
horas está más desarrollada.

Pero si tenemos la intuición de tratar las horas como números en base
60 y hacemos la inversión de escribir las funciones de conversión
(\texttt{convertirASegundos} y \texttt{crearHora}), obtenemos un programa
más corto, legible, depurable y confiable.

También facilita la adición de más características. Por ejemplo, piense
en el problema de restar dos \texttt{Hora}s para averiguar el tiempo
que transcurre entre ellas. La solución ingenua haría resta llevando
préstamos. En cambio, usar las funciones de conversión sería mas fácil.

Irónicamente, algunas veces el hacer de un problema algo más difícil
(o más general) lo hace más fácil (porque hay menos casos especiales
y menos oportunidades para caer en errores).

\section{Algoritmos}

\index{algoritmo}

Cuando usted escribe una solución general para una clase de problemas,
en vez de encontrar una solución específica a un solo problema, ha
escrito un \textbf{algoritmo}. Mencionamos esta palabra antes, pero
no la definimos cuidadosamente. No es fácil de definir, así que intentaremos
dos enfoques.

Primero, considere algo que no es un algoritmo. Cuando usted aprendió
a multiplicar dígitos, probablemente memorizó la tabla de multiplicación.
De hecho, usted memorizó 100 soluciones específicas. Este tipo de
conocimiento no es algorítmico.

Pero si usted fuera ``perezoso,'' probablemente aprendió a hacer
trampa por medio de algunos trucos. Por ejemplo, para encontrar el
producto entre $n$ y 9, usted puede escribir $n-1$ como el primer
dígito y $10-n$ como el segundo. Este truco es una solución general
para multiplicar cualquier dígito por el 9. ¡Este es un algoritmo!

Similarmente, las técnicas que aprendió para hacer suma con acarreo
( llevando para la columna hacia la derecha), resta con préstamos,
y división larga, todas son algoritmos. Una de las características
de los algoritmos es que no requieren inteligencia para ejecutarse.
Son procesos mecánicos en el que cada paso sigue al anterior de acuerdo
con un conjunto de reglas sencillas.

En nuestra opinión, es vergonzoso que los seres humanos pasemos tanto
tiempo en la escuela aprendiendo a ejecutar algoritmos que, literalmente,
no requieren inteligencia.

Por otro lado, el proceso de diseñar algoritmos es interesante, intelectualmente
desafiante y una parte central de lo que denominamos programación.

Algunas cosas que la gente hace naturalmente sin dificultad o pensamiento
consciente, son las mas difíciles de expresar algorítmicamente. Entender
el lenguaje natural es una de ellas. Todos lo hacemos, pero hasta
ahora nadie ha sido capaz de explicar {\em como} lo hacemos, al
menos no con un algoritmo.

\section{Glosario}
\begin{description}
\item [{Función pura:}] función que no modifica ninguno de los objetos
que recibe como parámetros. La mayoría de las funciones puras son
fructíferas.
\item [{Modificadora:}] función que cambia uno o varios de los objetos
que recibe como parámetros. La mayoría de los modificadoras no retornan
nada.
\item [{Estilo de programación funcional}] estilo de diseño de programas
en el que la mayoría de funciones son puras.
\item [{Desarrollo con prototipos:}] es la forma de desarrollar programas
empezando con un prototipo que empieza a mejorarse y probarse gradualmente.
\item [{Desarrollo planeado:}] es la forma de desarrollar programas que
implica un conocimiento de alto nivel sobre el problema y mas planeación
que el desarrollo incremental o el desarrollo con prototipos.
\item [{Algoritmo:}] conjunto de instrucciones para resolver una clase
de problemas por medio de un proceso mecánico, no inteligente.

\index{función pura} \index{modificadora} \index{estilo de programación funcional}
\index{desarrollo incremental} \index{desarrollo!incremental} \index{desarrollo planeado}
\index{desarrollo!planeado} \index{algoritmo}
\end{description}

\section{Ejercicios}
\begin{enumerate}
\item Reescriba la función \texttt{incrementar} de forma que no contenga
ciclos y siga siendo correcta.
\item Reescriba \texttt{incrementar} usando convertirASegundos y crearHora.
\item Reescriba \texttt{incrementar} como una función pura, y escriba llamados
a funciones de las dos versiones.
\item Escriba una función \texttt{imprimirHora} que reciba un objeto \texttt{Hora}
como argumento y lo imprima de la forma \texttt{horas:minutos:segundos}.
\item Escriba una función booleana \texttt{despues} que reciba dos objetos
\texttt{Hora}, \texttt{t1} y \texttt{t2} como argumentos, y retorne
cierto si \texttt{t1} va después de \texttt{t2} cronológicamente y
falso en caso contrario.
\end{enumerate}

