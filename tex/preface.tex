
\chapter{Prefacio}

Por Jeff Elkner

Este libro debe su existencia a la colaboración hecha posible por
Internet y el movimiento de software libre. Sus tres autores—un profesor
de colegio, un profesor de secundaria y un programador profesional—tienen
todavía que verse cara a cara, pero han podido trabajar juntos y han
sido ayudados por maravillosas personas, quienes han donado su tiempo
y energía para ayudar a hacer ver mejor este libro.

Nosotros pensamos que este libro es un testamento a los beneficios
y futuras posibilidades de esta clase de colaboración, el marco que
se ha puesto en marcha por Richard Stallman y el movimiento de software
libre.

\section*{Cómo y porqué vine a utilizar Python}

En 1999, el examen del College Board's Advanced Placement (AP) de
Informática se hizo en C++ por primera vez. Como en muchas escuelas
de Estados Unidos, la decisión para cambiar el lenguaje tenía un impacto
directo en el plan de estudios de informática en la escuela secundaria
de Yorktown en Arlington, Virginia, donde yo enseño. Hasta este punto,
Pascal era el lenguaje de instrucción en nuestros cursos del primer
año y del AP. Conservando la práctica usual de dar a los estudiantes
dos años de exposición al mismo lenguaje, tomamos la decisión de cambiar
a C++ en el curso del primer año durante el periodo escolar 1997-98
de modo que siguiéramos el cambio del College Board's para el curso
del AP el año siguiente.

Dos años después, estoy convencido de que C++ no era una buena opción
para introducir la informática a los estudiantes. Aunque es un lenguaje
de programación de gran alcance, también es extremadamente difícil
de aprender y de enseñar. Me encontré constantemente peleando con
la sintaxis difícil de C++ y sus múltiples maneras de hacer las cosas,
y estaba perdiendo muchos estudiantes, innecesariamente, como resultado.
Convencido de que tenía que haber una mejor opción para nuestras clases
de primer año, fui en busca de una alternativa a C++.

Necesitaba un lenguaje que pudiera correr en las máquinas en nuestro
laboratorio Linux, también en las plataformas de Windows y Macintosh,
que muchos de los estudiantes tienen en casa. Quería que fuese un
lenguaje de código abierto, para que los estudiantes lo pudieran usar
en casa sin pagar por una licencia. Quería un lenguaje usado por programadores
profesionales, y que tuviera una comunidad activa alrededor de él.
Tenía que soportar la programación procedimental y orientada a objetos.
Y más importante, tenía que ser fácil de aprender y de enseñar. Cuando
investigué las opciones con estas metas en mente, Python saltó como
el mejor candidato para la tarea.

Pedí a uno de los estudiantes más talentosos de Yorktown, Matt Ahrens,
que le diera a Python una oportunidad. En dos meses él no sólo aprendió
el lenguaje, sino que escribió una aplicación llamada pyTicket que
permitió a nuestro personal atender peticiones de soporte tecnológico
vía web. Sabia que Matt no podría terminar una aplicación de esa escala
en tan poco tiempo con C++, y esta observación, combinada con el gravamen
positivo de Matt sobre Python, sugirió que este lenguaje era la solución
que buscaba.

\section*{Encontrando un libro de texto}

Al decidir utilizar Python en mis dos clases de informática introductoria
para el año siguiente, el problema más acuciante era la carencia de
un libro.

El contenido libre vino al rescate. A principios del año, Richard
Stallman me presentó a Allen Downey. Los dos habíamos escrito a Richard
expresando interés en desarrollar un contenido gratis y educativo.
Allen ya había escrito un libro de texto para el primer año de informática,
{\em Como pensar como un científico de la computación}. Cuando
leí este libro, inmediatamente quise usarlo en mi clase. Era el texto
más claro y mas provechoso de introducción a la informática que había
visto. Acentúa los procesos del pensamiento implicados en la programación
más bien que las características de un lenguaje particular. Leerlo
me hizo sentir un mejor profesor inmediatamente. {\em Como pensar
como un científico de la computación con Java} no solo es un libro
excelente, sino que también había sido publicado bajo la licencia
publica GNU, lo que significa que podría ser utilizado libremente
y ser modificado para resolver otras necesidades. Una vez que decidí
utilizar Python, se me ocurrió que podía traducir la versión original
del libro de Allen (en Java) al nuevo lenguaje (Python). Aunque no
podía escribir un libro de texto solo, tener el libro de Allen me
facilitó la tarea, y al mismo tiempo demostró que el modelo cooperativo
usado en el desarrollo de software también podía funcionar para el
contenido educativo.

Trabajar en este libro, por los dos últimos años, ha sido una recompensa
para mis estudiantes y para mí; y mis estudiantes tuvieron una gran
participación en el proceso. Puesto que podía realizar cambios inmediatos,
siempre que alguien encontrara un error de deletreo o un paso difícil,
yo les animaba a que buscaran errores en el libro, dándoles un punto
cada vez que hicieran una sugerencia que resultara en un cambio en
el texto. Eso tenía la ventaja doble de animarles a que leyeran el
texto más cuidadosamente y de conseguir la corrección del texto por
sus lectores críticos más importantes, los estudiantes usándolo para
aprender informática.

Para la segunda parte del libro, enfocada en la programación orientada
a objetos, sabía que alguien con más experiencia en programación que
yo era necesario para hacer el trabajo correctamente. El libro estuvo
incompleto la mayoría del año hasta que la comunidad de software abierto
me proporcionó de nuevo los medios necesarios para su terminación.

Recibí un correo electrónico de Chris Meyers, expresando su interés
en el libro. Chris es un programador profesional que empezó enseñando
un curso de programación el año anterior, usando Python en el Lane
Community College en Eugene, Oregon. La perspectiva de enseñar el
curso llevó a Chris al libro, y él comenzó a ayudarme inmediatamente.
Antes del fin de año escolar él había creado un proyecto complementario
en nuestro Sitio Web \url{http://www.ibiblio.org/obp}, titulado {\em
Python for Fun} y estaba trabajando con algunos de mis estudiantes
más avanzados como profesor principal, guiándolos mas allá de donde
yo podía llevarlos.

\section*{Introduciendo la programación con Python}

El proceso de uso y traducción de {\em Como pensar como un científico
de la computación}, por los últimos dos años, ha confirmado la conveniencia
de Python para enseñar a estudiantes principiantes. Python simplifica
bastante los ejemplos de programación y hace que las ideas importantes
sean más fáciles de enseñar.

El primer ejemplo del texto ilustra este punto. Es el tradicional
``hola, mundo'', programa que en la versión C++ del libro se ve
así:
\begin{verbatim}
   #include <iostream.h>

   void main()
   {
     cout << "Hola, mundo." << endl;
   }
\end{verbatim}
en la versión Python es:
\begin{verbatim}
    print("Hola, Mundo!")
\end{verbatim}
Aunque este es un ejemplo trivial, las ventajas de Python salen a
la luz. El curso de Informática I, en Yorktown, no tiene prerrequisitos,
es por eso que muchos de los estudiantes, que ven este ejemplo, están
mirando a su primer programa. Algunos de ellos están un poco nerviosos,
porque han oído que la programación de computadores es difícil de
aprender. La versión C++ siempre me ha forzado a escoger entre dos
opciones que no me satisfacen: explicar el \texttt{\#include}, \texttt{void
main()}, y las sentencias \{, y \} y arriesgar a confundir o intimidar
a algunos de los estudiantes al principio, o decirles, ``No te preocupes
por todo eso ahora; lo retomaré más tarde,'' y tomar el mismo riesgo.
Los objetivos educativos en este momento del curso son introducir
a los estudiantes la idea de sentencia y permitirles escribir su primer
programa. Python tiene exactamente lo que necesito para lograr esto,
y nada más.

Comparar el texto explicativo de este programa en cada versión del
libro ilustra más de lo que esto significa para los estudiantes principiantes.
Hay trece párrafos de explicación de ``Hola, mundo!'' en la versión
C++; en la versión Python, solo hay dos. Aún mas importante, los 11
párrafos que faltan no hablan de ``grandes ideas'' en la programación
de computadores, sino de minucias de la sintaxis de C++. Encontré
la misma situación al repasar todo el libro. Párrafos enteros desaparecían
en la versión Python del texto, porque su sencilla sintaxis los hacía
innecesarios.

Usar un lenguaje de muy alto nivel, como Python, le permite a un profesor
posponer los detalles de bajo nivel de la máquina hasta que los estudiantes
tengan el bagaje que necesitan para entenderlos. Permite ``poner
cosas primero'' pedagógicamente. Unos de los mejores ejemplos de
esto es la manera en la que Python maneja las variables. En C++ una
variable es un nombre para un lugar que almacena una cosa. Las variables
tienen que ser declaradas con tipos, al menos parcialmente, porque
el tamaño del lugar al cual se refieren tiene que ser predeterminado.
Así, la idea de una variable se liga con el hardware de la máquina.
El concepto poderoso y fundamental de variable ya es difícil para
los estudiantes principiantes (de informática y álgebra). Bytes y
direcciones de memoria no ayudan para nada. En Python una variable
es un nombre que se refiere a una cosa. Este es un concepto más intuitivo
para los estudiantes principiantes y está más cerca del significado
de ``variable'' que aprendieron en los cursos de matemática del
colegio. Yo me demoré menos tiempo ayudándolos con el concepto de
variable y en su uso este año, que en el pasado.

Otro ejemplo de cómo Python ayuda en la enseñanza y aprendizaje de
la programación es su sintaxis para las funciones. Mis estudiantes
siempre han tenido una gran dificultad comprendiendo las funciones.
El problema principal se centra alrededor de la diferencia entre una
definición de función y un llamado de función, y la distinción relacionada
entre un parámetro y un argumento. Python viene al rescate con una
bella sintaxis. Una definición de función empieza con la palabra clave
\texttt{def}, y simplemente digo a mis estudiantes: ``cuando definas
una función, empieza con \texttt{def}, seguido del nombre de la función
que estás definiendo, cuando llames una función, simplemente llama
(digita) su nombre.'' Los parámetros van con las definiciones y los
argumentos van con los llamados. No hay tipos de retorno, tipos para
los parámetros, o pasos de parámetro por referencia y valor, y ahora
yo puedo enseñar funciones en la mitad de tiempo que antes, con una
mejor comprensión.

Usar Python ha mejorado la eficacia de nuestro programa de informática
para todos los estudiantes. Veo un nivel general de éxito más alto
y un nivel más bajo de frustración, de lo que ya había experimentado
durante los dos años que enseñé C++. Avanzo más rápido y con mejores
resultados. Más estudiantes terminan el curso con la habilidad de
crear programas significativos; esto genera una actitud positiva hacia
la experiencia de la programación.

\section*{Construyendo una comunidad}

He recibido correos electrónicos de todas partes del mundo, de personas
que están usando este libro para aprender o enseñar programación.
Una comunidad de usuarios ha comenzado a emerger, y muchas personas
han contribuido al proyecto mandando materiales a través del sitio
Web complementario: \\
 \\
\url{ http://www.thinkpython.com }\\
 \\

Con la publicación del libro, en forma impresa, espero que continúe
y se acelere el crecimiento de esta comunidad de usuarios.

La emergencia de esta comunidad y la posibilidad que sugiere para
otras experiencias de colaboración similar entre educadores han sido
las partes más excitantes de trabajar en este proyecto, para mí. Trabajando
juntos, nosotros podemos aumentar la calidad del material disponible
para nuestro uso y ahorrar tiempo valioso.

Yo les invito a formar parte de nuestra comunidad y espero escuchar
de ustedes. Por favor escriba a los autores a \texttt{\url{feedback@thinkpython.com}}.

\vspace{0.25in}
 
\begin{flushleft}
Jeffrey Elkner\\
 Escuela Secundaria Yortown\\
 Arlington, Virginia.\\
 
\par\end{flushleft}


