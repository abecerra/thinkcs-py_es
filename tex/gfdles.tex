
\chapter{Licencia de documentación libre de GNU}

Versión 1.2, Noviembre 2002 \\

This is an unofficial translation of the GNU Free Documentation License
into Spanish. It was not published by the Free Software Foundation,
and does not legally state the distribution terms for documentation
that uses the GNU FDL – only the original English text of the GNU
FDL does that. However, we hope that this translation will help Spanish
speakers understand the GNU FDL better.

Esta es una traducción no oficial de la GNU Free Document License
a Español (Castellano). No ha sido publicada por la Free Software
Foundation y no establece legalmente los términos de distribución
para trabajos que usen la GFDL (sólo el texto de la versión original
en Inglés de la GFDL lo hace). Sin embargo, esperamos que esta traducción
ayude a los hispanohablantes a entender mejor la GFDL. La versión
original de la GFDL está disponible en la Free Software Foundation{[}1{]}.

Esta traducción está basada en una la versión 1.1 de Igor Támara y
Pablo Reyes. Sin embargo la responsabilidad de su interpretación es
de Joaquín Seoane.

Copyright (C) 2000, 2001, 2002 Free Software Foundation, Inc. 59 Temple
Place, Suite 330, Boston, MA 02111-1307 USA. Se permite la copia y
distribución de copias literales de este documento de licencia, pero
no se permiten cambios{[}1{]}.

%\rule{\linewidth}{1pt}

\section*{Preámbulo}

El propósito de esta licencia es permitir que un manual, libro de
texto, u otro documento escrito sea libre en el sentido de libertad:
asegurar a todo el mundo la libertad efectiva de copiarlo y redistribuirlo,
con o sin modificaciones, de manera comercial o no. En segundo término,
esta licencia proporciona al autor y al editor{[}2{]} una manera de
obtener reconocimiento por su trabajo, sin que se le considere responsable
de las modificaciones realizadas por otros.

Esta licencia es de tipo copyleft, lo que significa que los trabajos
derivados del documento deben a su vez ser libres en el mismo sentido.
Complementa la Licencia Pública General de GNU, que es una licencia
tipo copyleft diseñada para el software libre.

Hemos diseñado esta licencia para usarla en manuales de software libre,
ya que el software libre necesita documentación libre: un programa
libre debe venir con manuales que ofrezcan la mismas libertades que
el software. Pero esta licencia no se limita a manuales de software;
puede usarse para cualquier texto, sin tener en cuenta su temática
o si se publica como libro impreso o no. Recomendamos esta licencia
principalmente para trabajos cuyo fin sea instructivo o de referencia.

\section{Aplicabilidad y definiciones}

Esta licencia se aplica a cualquier manual u otro trabajo, en cualquier
soporte, que contenga una nota del propietario de los derechos de
autor que indique que puede ser distribuido bajo los términos de esta
licencia. Tal nota garantiza en cualquier lugar del mundo, sin pago
de derechos y sin límite de tiempo, el uso de dicho trabajo según
las condiciones aquí estipuladas. En adelante la palabra Documento
se referirá a cualquiera de dichos manuales o trabajos. Cualquier
persona es un licenciatario y será referido como Usted. Usted acepta
la licencia si copia. modifica o distribuye el trabajo de cualquier
modo que requiera permiso según la ley de propiedad intelectual.

Una Versión Modificada del Documento significa cualquier trabajo que
contenga el Documento o una porción del mismo, ya sea una copia literal
o con modificaciones y/o traducciones a otro idioma.

Una Sección Secundaria es un apéndice con título o una sección preliminar
del Documento que trata exclusivamente de la relación entre los autores
o editores y el tema general del Documento (o temas relacionados)
pero que no contiene nada que entre directamente en dicho tema general
(por ejemplo, si el Documento es en parte un texto de matemáticas,
una Sección Secundaria puede no explicar nada de matemáticas). La
relación puede ser una conexión histórica con el tema o temas relacionados,
o una opinión legal, comercial, filosófica, ética o política acerca
de ellos.

Las Secciones Invariantes son ciertas Secciones Secundarias cuyos
títulos son designados como Secciones Invariantes en la nota que indica
que el documento es liberado bajo esta Licencia. Si una sección no
entra en la definición de Secundaria, no puede designarse como Invariante.
El documento puede no tener Secciones Invariantes. Si el Documento
no identifica las Secciones Invariantes, es que no las tiene.

Los Textos de Cubierta son ciertos pasajes cortos de texto que se
listan como Textos de Cubierta Delantera o Textos de Cubierta Trasera
en la nota que indica que el documento es liberado bajo esta Licencia.
Un Texto de Cubierta Delantera puede tener como mucho 5 palabras,
y uno de Cubierta Trasera puede tener hasta 25 palabras.

Una copia Transparente del Documento, significa una copia para lectura
en máquina, representada en un formato cuya especificación está disponible
al público en general, apto para que los contenidos puedan ser vistos
y editados directamente con editores de texto genéricos o (para imágenes
compuestas por puntos) con programas genéricos de manipulación de
imágenes o (para dibujos) con algún editor de dibujos ampliamente
disponible, y que sea adecuado como entrada para formateadores de
texto o para su traducción automática a formatos adecuados para formateadores
de texto. Una copia hecha en un formato definido como Transparente,
pero cuyo marcaje o ausencia de él haya sido diseñado para impedir
o dificultar modificaciones posteriores por parte de los lectores
no es Transparente. Un formato de imagen no es Transparente si se
usa para una cantidad de texto sustancial. Una copia que no es Transparente
se denomina Opaca.

Como ejemplos de formatos adecuados para copias Transparentes están
ASCII puro sin marcaje, formato de entrada de Texinfo, formato de
entrada de LaTeX, SGML o XML usando una DTD disponible públicamente,
y HTML, PostScript o PDF simples, que sigan los estándares y diseñados
para que los modifiquen personas. Ejemplos de formatos de imagen transparentes
son PNG, XCF y JPG. Los formatos Opacos incluyen formatos propietarios
que pueden ser leídos y editados únicamente en procesadores de palabras
propietarios, SGML o XML para los cuáles las DTD y/o herramientas
de procesamiento no estén ampliamente disponibles, y HTML, PostScript
o PDF generados por algunos procesadores de palabras sólo como salida.

La Portada significa, en un libro impreso, la página de título, más
las páginas siguientes que sean necesarias para mantener legiblemente
el material que esta Licencia requiere en la portada. Para trabajos
en formatos que no tienen página de portada como tal, Portada significa
el texto cercano a la aparición más prominente del título del trabajo,
precediendo el comienzo del cuerpo del texto.

Una sección Titulada XYZ significa una parte del Documento cuyo título
es precisamente XYZ o contiene XYZ entre paréntesis, a continuación
de texto que traduce XYZ a otro idioma (aquí XYZ se refiere a nombres
de sección específicos mencionados más abajo, como Agradecimientos,
Dedicatorias , Aprobaciones o Historia. Conservar el Título de tal
sección cuando se modifica el Documento significa que permanece una
sección Titulada XYZ según esta definición .

El Documento puede incluir Limitaciones de Garantía cercanas a la
nota donde se declara que al Documento se le aplica esta Licencia.
Se considera que estas Limitaciones de Garantía están incluidas, por
referencia, en la Licencia, pero sólo en cuanto a limitaciones de
garantía: cualquier otra implicación que estas Limitaciones de Garantía
puedan tener es nula y no tiene efecto en el significado de esta Licencia.

%\rule{\linewidth}{1pt}

\section{Copia literal}

Usted puede copiar y distribuir el Documento en cualquier soporte,
sea en forma comercial o no, siempre y cuando esta Licencia, las notas
de copyright y la nota que indica que esta Licencia se aplica al Documento
se reproduzcan en todas las copias y que usted no añada ninguna otra
condición a las expuestas en esta Licencia. Usted no puede usar medidas
técnicas para obstruir o controlar la lectura o copia posterior de
las copias que usted haga o distribuya. Sin embargo, usted puede aceptar
compensación a cambio de las copias. Si distribuye un número suficientemente
grande de copias también deberá seguir las condiciones de la sección
3.

Usted también puede prestar copias, bajo las mismas condiciones establecidas
anteriormente, y puede exhibir copias públicamente.

\section{Copiado en cantidad}

Si publica copias impresas del Documento (o copias en soportes que
tengan normalmente cubiertas impresas) que sobrepasen las 100, y la
nota de licencia del Documento exige Textos de Cubierta, debe incluir
las copias con cubiertas que lleven en forma clara y legible todos
esos Textos de Cubierta: Textos de Cubierta Delantera en la cubierta
delantera y Textos de Cubierta Trasera en la cubierta trasera. Ambas
cubiertas deben identificarlo a Usted clara y legiblemente como editor
de tales copias. La cubierta debe mostrar el título completo con todas
las palabras igualmente prominentes y visibles. Además puede añadir
otro material en las cubiertas. Las copias con cambios limitados a
las cubiertas, siempre que conserven el título del Documento y satisfagan
estas condiciones, pueden considerarse como copias literales.

Si los textos requeridos para la cubierta son muy voluminosos para
que ajusten legiblemente, debe colocar los primeros (tantos como sea
razonable colocar) en la verdadera cubierta y situar el resto en páginas
adyacentes.

Si Usted publica o distribuye copias Opacas del Documento cuya cantidad
exceda las 100, debe incluir una copia Transparente, que pueda ser
leída por una máquina, con cada copia Opaca, o bien mostrar, en cada
copia Opaca, una dirección de red donde cualquier usuario de la misma
tenga acceso por medio de protocolos públicos y estandarizados a una
copia Transparente del Documento completa, sin material adicional.
Si usted hace uso de la última opción, deberá tomar las medidas necesarias,
cuando comience la distribución de las copias Opacas en cantidad,
para asegurar que esta copia Transparente permanecerá accesible en
el sitio establecido por lo menos un año después de la última vez
que distribuya una copia Opaca de esa edición al público (directamente
o a través de sus agentes o distribuidores).

Se solicita, aunque no es requisito, que se ponga en contacto con
los autores del Documento antes de redistribuir gran número de copias,
para darles la oportunidad de que le proporcionen una versión actualizada
del Documento.

%\rule{\linewidth}{1pt}

\section{Modificaciones}

Puede copiar y distribuir una Versión Modificada del Documento bajo
las condiciones de las secciones 2 y 3 anteriores, siempre que usted
libere la Versión Modificada bajo esta misma Licencia, con la Versión
Modificada haciendo el rol del Documento, por lo tanto dando licencia
de distribución y modificación de la Versión Modificada a quienquiera
posea una copia de la misma. Además, debe hacer lo siguiente en la
Versión Modificada:
\begin{itemize}
\item Usar en la Portada (y en las cubiertas, si hay alguna) un título distinto
al del Documento y de sus versiones anteriores (que deberían, si hay
alguna, estar listadas en la sección de Historia del Documento). Puede
usar el mismo título de versiones anteriores al original siempre y
cuando quien las publicó originalmente otorgue permiso.
\item Listar en la Portada, como autores, una o más personas o entidades
responsables de la autoría de las modificaciones de la Versión Modificada,
junto con por lo menos cinco de los autores principales del Documento
(todos sus autores principales, si hay menos de cinco), a menos que
le eximan de tal requisito.
\item Mostrar en la Portada como editor el nombre del editor de la Versión
Modificada.
\item Conservar todas las notas de copyright del Documento.
\item Añadir una nota de copyright apropiada a sus modificaciones, adyacente
a las otras notas de copyright.
\item Incluir, inmediatamente después de las notas de copyright, una nota
de licencia dando el permiso para usar la Versión Modificada bajo
los términos de esta Licencia, como se muestra en la Adenda al final
de este documento.
\item Conservar en esa nota de licencia el listado completo de las Secciones
Invariantes y de los Textos de Cubierta que sean requeridos en la
nota de Licencia del Documento original.
\item Incluir una copia sin modificación de esta Licencia.
\item Conservar la sección Titulada Historia, conservar su Título y añadirle
un elemento que declare al menos el título, el año, los nuevos autores
y el editor de la Versión Modificada, tal como figuran en la Portada.
Si no hay una sección Titulada Historia en el Documento, crear una
estableciendo el título, el año, los autores y el editor del Documento,
tal como figuran en su Portada, añadiendo además un elemento describiendo
la Versión Modificada, como se estableció en la oración anterior.
\item Conservar la dirección en red, si la hay, dada en el Documento para
el acceso público a una copia Transparente del mismo, así como las
otras direcciones de red dadas en el Documento para versiones anteriores
en las que estuviese basado. Pueden ubicarse en la sección Historia.
Se puede omitir la ubicación en red de un trabajo que haya sido publicado
por lo menos cuatro años antes que el Documento mismo, o si el editor
original de dicha versión da permiso.
\item En cualquier sección Titulada Agradecimientos o Dedicatorias, Conservar
el Título de la sección y conservar en ella toda la sustancia y el
tono de los agradecimientos y/o dedicatorias incluidas por cada contribuyente.
\item Conservar todas las Secciones Invariantes del Documento, sin alterar
su texto ni sus títulos. Números de sección o el equivalente no son
considerados parte de los títulos de la sección.
\item Borrar cualquier sección titulada Aprobaciones. Tales secciones no
pueden estar incluidas en las Versiones Modificadas.
\item No cambiar el título de ninguna sección existente a Aprobaciones ni
a uno que entre en conflicto con el de alguna Sección Invariante.
\item Conservar todas las Limitaciones de Garantía.
\end{itemize}
Si la Versión Modificada incluye secciones o apéndices nuevos que
califiquen como Secciones Secundarias y contienen material no copiado
del Documento, puede opcionalmente designar algunas o todas esas secciones
como invariantes. Para hacerlo, añada sus títulos a la lista de Secciones
Invariantes en la nota de licencia de la Versión Modificada. Tales
títulos deben ser distintos de cualquier otro título de sección.

Puede añadir una sección titulada Aprobaciones, siempre que contenga
únicamente aprobaciones de su Versión Modificada por otras fuentes
–por ejemplo, observaciones de peritos o que el texto ha sido aprobado
por una organización como la definición oficial de un estándar.

Puede añadir un pasaje de hasta cinco palabras como Texto de Cubierta
Delantera y un pasaje de hasta 25 palabras como Texto de Cubierta
Trasera en la Versión Modificada. Una entidad solo puede añadir (o
hacer que se añada) un pasaje al Texto de Cubierta Delantera y uno
al de Cubierta Trasera. Si el Documento ya incluye un textos de cubiertas
añadidos previamente por usted o por la misma entidad que usted representa,
usted no puede añadir otro; pero puede reemplazar el anterior, con
permiso explícito del editor que agregó el texto anterior.

Con esta Licencia ni los autores ni los editores del Documento dan
permiso para usar sus nombres para publicidad ni para asegurar o implicar
aprobación de cualquier Versión Modificada.

\section{Combinación de documentos}

Usted puede combinar el Documento con otros documentos liberados bajo
esta Licencia, bajo los términos definidos en la sección 4 anterior
para versiones modificadas, siempre que incluya en la combinación
todas las Secciones Invariantes de todos los documentos originales,
sin modificar, listadas todas como Secciones Invariantes del trabajo
combinado en su nota de licencia. Así mismo debe incluir la Limitación
de Garantía.

El trabajo combinado necesita contener solamente una copia de esta
Licencia, y puede reemplazar varias Secciones Invariantes idénticas
por una sola copia. Si hay varias Secciones Invariantes con el mismo
nombre pero con contenidos diferentes, haga el título de cada una
de estas secciones único añadiéndole al final del mismo, entre paréntesis,
el nombre del autor o editor original de esa sección, si es conocido,
o si no, un número único. Haga el mismo ajuste a los títulos de sección
en la lista de Secciones Invariantes de la nota de licencia del trabajo
combinado.

En la combinación, debe combinar cualquier sección Titulada Historia
de los documentos originales, formando una sección Titulada Historia;
de la misma forma combine cualquier sección Titulada Agradecimientos,
y cualquier sección Titulada Dedicatorias. Debe borrar todas las secciones
tituladas Aprobaciones.

\section{Colecciones de documentos}

Puede hacer una colección que conste del Documento y de otros documentos
liberados bajo esta Licencia, y reemplazar las copias individuales
de esta Licencia en todos los documentos por una sola copia que esté
incluida en la colección, siempre que siga las reglas de esta Licencia
para cada copia literal de cada uno de los documentos en cualquiera
de los demás aspectos.

Puede extraer un solo documento de una de tales colecciones y distribuirlo
individualmente bajo esta Licencia, siempre que inserte una copia
de esta Licencia en el documento extraído, y siga esta Licencia en
todos los demás aspectos relativos a la copia literal de dicho documento.

\section{Agregación con trabajos independientes}

Una recopilación que conste del Documento o sus derivados y de otros
documentos o trabajos separados e independientes, en cualquier soporte
de almacenamiento o distribución, se denomina un agregado si el copyright
resultante de la compilación no se usa para limitar los derechos de
los usuarios de la misma más allá de lo que los de los trabajos individuales
permiten. Cuando el Documento se incluye en un agregado, esta Licencia
no se aplica a otros trabajos del agregado que no sean en sí mismos
derivados del Documento.

Si el requisito de la sección 3 sobre el Texto de Cubierta es aplicable
a estas copias del Documento y el Documento es menor que la mitad
del agregado entero, los Textos de Cubierta del Documento pueden colocarse
en cubiertas que enmarquen solamente el Documento dentro del agregado,
o el equivalente electrónico de las cubiertas si el documento está
en forma electrónica. En caso contrario deben aparecer en cubiertas
impresas enmarcando todo el agregado.

\section{Traducción}

La Traducción es considerada como un tipo de modificación, por lo
que usted puede distribuir traducciones del Documento bajo los términos
de la sección 4. El reemplazo de las Secciones Invariantes con traducciones
requiere permiso especial de los dueños de derecho de autor, pero
usted puede añadir traducciones de algunas o todas las Secciones Invariantes
a las versiones originales de las mismas. Puede incluir una traducción
de esta Licencia, de todas las notas de licencia del documento, así
como de las Limitaciones de Garantía, siempre que incluya también
la versión en Inglés de esta Licencia y las versiones originales de
las notas de licencia y Limitaciones de Garantía. En caso de desacuerdo
entre la traducción y la versión original en Inglés de esta Licencia,
la nota de licencia o la limitación de garantía, la versión original
en Inglés prevalecerá.

Si una sección del Documento está Titulada Agradecimientos, Dedicatorias
o Historia el requisito (sección 4) de Conservar su Título (Sección
1) requerirá, típicamente, cambiar su título.

\section{Terminación}

Usted no puede copiar, modificar, sublicenciar o distribuir el Documento
salvo por lo permitido expresamente por esta Licencia. Cualquier otro
intento de copia, modificación, sublicenciamiento o distribución del
Documento es nulo, y dará por terminados automáticamente sus derechos
bajo esa Licencia. Sin embargo, los terceros que hayan recibido copias,
o derechos, de usted bajo esta Licencia no verán terminadas sus licencias,
siempre que permanezcan en total conformidad con ella.

\section{Revisiones futuras de esta licencia}

De vez en cuando la Free Software Foundation puede publicar versiones
nuevas y revisadas de la Licencia de Documentación Libre GNU. Tales
versiones nuevas serán similares en espíritu a la presente versión,
pero pueden diferir en detalles para solucionar nuevos problemas o
intereses. Vea \url{http://www.gnu.org/copyleft/}.

Cada versión de la Licencia tiene un número de versión que la distingue.
Si el Documento especifica que se aplica una versión numerada en particular
de esta licencia o cualquier versión posterior, usted tiene la opción
de seguir los términos y condiciones de la versión especificada o
cualquiera posterior que haya sido publicada (no como borrador) por
la Free Software Foundation. Si el Documento no especifica un número
de versión de esta Licencia, puede escoger cualquier versión que haya
sido publicada (no como borrador) por la Free Software Foundation.

\section{ADENDA: Cómo usar esta Licencia en sus documentos}

Para usar esta licencia en un documento que usted haya escrito, incluya
una copia de la Licencia en el documento y ponga el siguiente copyright
y nota de licencia justo después de la página de título:
\begin{quote}
Copyright (c) AÑO SU NOMBRE. Se concede permiso para copiar, distribuir
y/o modificar este documento bajo los términos de la Licencia de Documentación
Libre de GNU, Versión 1.2 o cualquier otra versión posterior publicada
por la Free Software Foundation; sin Secciones Invariantes ni Textos
de Cubierta Delantera ni Textos de Cubierta Trasera. Una copia de
la licencia está incluida en la sección titulada GNU Free Documentation
License. 
\end{quote}
Si tiene Secciones Invariantes, Textos de Cubierta Delantera y Textos
de Cubierta Trasera, reemplace la frase sin ... Trasera por esto:
\begin{quote}
siendo las Secciones Invariantes LISTE SUS TÍTULOS, siendo los Textos
de Cubierta Delantera LISTAR, y siendo sus Textos de Cubierta Trasera
LISTAR. 
\end{quote}
Si tiene Secciones Invariantes sin Textos de Cubierta o cualquier
otra combinación de los tres, mezcle ambas alternativas para adaptarse
a la situación.

Si su documento contiene ejemplos de código de programa no triviales,
recomendamos liberar estos ejemplos en paralelo bajo la licencia de
software libre que usted elija, como la Licencia Pública General de
GNU (GNU General Public License), para permitir su uso en software
libre.

\rule{1\linewidth}{1pt}

Notas

{[}1{]} Ésta es la traducción del Copyright de la Licencia, no es
el Copyright de esta traducción no autorizada.

{[}2{]} La licencia original dice publisher, que es, estrictamente,
quien publica, diferente de editor, que es más bien quien prepara
un texto para publicar. En castellano editor se usa para ambas cosas.

{[}3{]} En sentido estricto esta licencia parece exigir que los títulos
sean exactamente Acknowledgements, Dedications, Endorsements e History,
en inglés.
