% LaTeX source for textbook ``How to think like a computer scientist''
% Copyright (c)  2001  Allen B. Downey, Jeffrey Elkner, and Chris Meyers.

% Permission is granted to copy, distribute and/or modify this
% document under the terms of the GNU Free Documentation License,
% Version 1.1  or any later version published by the Free Software
% Foundation; with the Invariant Sections being "Contributor List",
% with no Front-Cover Texts, and with no Back-Cover Texts. A copy of
% the license is included in the section entitled "GNU Free
% Documentation License".

% This distribution includes a file named fdl.tex that contains the text
% of the GNU Free Documentation License.  If it is missing, you can obtain
% it from www.gnu.org or by writing to the Free Software Foundation,
% Inc., 59 Temple Place - Suite 330, Boston, MA 02111-1307, USA.

\chapter{Interludio 1: Triqui}
\index{Triqui}



\section{Motivación}

Con el fin de poner en práctica varios elementos (funciones, ciclos, listas)
vamos a desarrollar un sencillo juego de triqui para dos jugadores. La idea
es seguir un desarrollo iterativo. El tablero de juego es una matriz 3x3. 

Si tablero[i][j] es:

\begin{description}
  \item[' ':]  nadie ha jugado en esa casilla
  \item['O': ] el primer jugador jugó en esa casilla
  \item['X':] el segundo jugador jugó en esa casilla
\end{description}

Asi que nuestro programa empieza humildemente:

\beforeverb
\begin{verbatim}
# triqui0.py
def crear():
    """ Crea una matrix 3x3 vacia"""
    m =  [ [' ',' ',' '],
           [' ',' ',' '],
           [' ',' ',' '] ]
    return m


def imprimir(tablero):
    """ Imprime el tablero de juego"""
    for i in range(3):
        print "|",
        for j in range(3):
            print tablero[i][j],
        print "|"
        
triqui = crear()
imprimir(triqui)
\end{verbatim}
\afterverb

Ahora podemos agregar un ciclo para jugar, sencillo, con un solo jugador:

\beforeverb
\begin{verbatim}
# triqui1.py
# Solo juega un jugador por siempre 
while True:
    print "Juegue jugador O"
    f = input("fila? ")
    c = input("columna? ")
    triqui[f][c] = "O"
    imprimir(triqui)
\end{verbatim}
\afterverb

Agregar al segundo jugador es sencillo, está en triqui2.py. Para verificar si 
un jugador gana la partida, vamos a empezar por las diagonales, implementando
un ciclo para la verificar si alguien gana en la diagonal principal:

\beforeverb
\begin{verbatim}
# triqui3.py
def ganaDiagonal1(jugador,tablero):
    """Chequea si el jugador gana en la diagonal 1 \"""
    for i in range(3):
        if tablero[i][i]!=jugador:
            return False
    return True
\end{verbatim}
\afterverb

La idea es que si encuentra algo diferente del símbolo del jugador ('X' ó 'O'),
retorna False. Sino, retorna True. La otra diagonal requiere mas trabajo, usamos 
el hecho de que tablero[i][2-i] va dando los elementos de la segunda
diagonal para i de 0 a 2. ¡Verifiquelo!

\beforeverb
\begin{verbatim}
# triqui3.py
def ganaDiagonal2(jugador,tablero):
    """Chequea si el jugador gana en la diagonal 2 / """
    for i in range(3):
        if tablero[i][2-i]!=jugador:
            return False
    return True
\end{verbatim}
\afterverb

Falta llamar las funciones en el ciclo del juego, y si gana terminamos el juego 
con la sentencia break. Por ejemplo, para el primer jugador:

\beforeverb
\begin{verbatim}
# triqui3.py
print "Juegue jugador O"
    f = input("fila? ")
    c = input("columna? ")
    triqui[f][c] = "O"
    imprimir(triqui)
    if ganaDiagonal1("O",triqui) or ganaDiagonal2("O",triqui):
        print "Gana el jugador O!!!!"
        break
\end{verbatim}
\afterverb

Agregar las funciones para verificar si alguien gana por alguna fila es sencillo.
Seguimos la misma estructura de las diagonales, creando una función ganaFila, que
es llamada para todas las filas por ganaHorizontal. Las funciones para chequear
las columnas son casi idénticas.

\beforeverb
\begin{verbatim}
# triqui4.py
def ganaFila(fila,jugador,tablero):
    """Chequea si el jugador gana en la fila dada"""
    for i in range(3):
        if tablero[fila][i]!=jugador:
            return False
    return True

def ganaHorizontal(jugador,tablero):
    """ Chequea todas las filas """
    for i in range(3):
        if ganaFila(i,jugador,tablero):
            return True
    return False
\end{verbatim}
\afterverb

Para llamar las funciones modificamos el ciclo del juego. Por ejemplo, para el jugador 'X':
\beforeverb
\begin{verbatim}
# triqui4.py
while True:
    print "Juegue jugador X"
    f = input("fila? ")
    c = input("columna? ")
    triqui[f][c] = "X"
    imprimir(triqui)
    if ganaDiagonal1("X",triqui) or ganaDiagonal2("X",triqui) or ganaHorizontal("X",triqui) or ganaVertical("X",triqui):
        print "Gana el jugador X!!!!"
        break
\end{verbatim}
\afterverb

Como el código dentro del ciclo para el jugador 'O' y el 'X' es casi idéntico, lo podemos poner en una
función que tenga como parámetro el símbolo del jugador:

\beforeverb
\begin{verbatim}
# triqui5.py
def jugar(jugador,tablero):
    print "Juegue jugador ", jugador
    f = input("fila? ")
    c = input("columna? ")
    tablero[f][c] = jugador
    imprimir(triqui)
    diag = ganaDiagonal1(jugador,tablero) or ganaDiagonal2(jugador,tablero)
    linea = ganaHorizontal(jugador,tablero) or ganaVertical(jugador,tablero)
    return  diag or linea
\end{verbatim}
\afterverb

Con esto nuestro ciclo de juego es más pequeño:

\beforeverb
\begin{verbatim}
# triqui5.py
while True:
    if jugar("O",triqui):
        print "Gana el jugador O !!!!"
        break
    if jugar("X",triqui):
        print "Gana el jugador X !!!!"
        break
\end{verbatim}
\afterverb


Ahora podemos agregar validación al juego. Tanto f como  c deben ser enteros
en el intervalo [0,2], y la casilla tablero[f][c] debe estar vacía para que
una jugada sea válida. Esto se puede poner en un ciclo que le pida al jugador
digitar los valores para f y c varias veces, hasta que sean correctos:

\beforeverb
\begin{verbatim}
# triqui6.py

def valido(n):
    return 0<=n<=2
    
def jugar(jugador,tablero):
    while True:     
        print "Juegue jugador ", jugador
        f = input("fila? ")
        c = input("columna? ")
        if type(f)==type(c)==type(1) and valido(f) and valido(c) and tablero[f][c]==' ':
            tablero[f][c] = jugador
            break      

    imprimir(tablero)
    diag = ganaDiagonal1(jugador,tablero) or ganaDiagonal2(jugador,tablero)
    linea = ganaHorizontal(jugador,tablero) or ganaVertical(jugador,tablero)
    return  diag or linea

\end{verbatim}
\afterverb


Ahora agregamos una función para chequear si hay empate entre los jugadores.
Esto sucede si el tablero está lleno, o sea que no hay ninguna casilla vacía 
(con el caracter ' '):

\beforeverb
\begin{verbatim}
# triqui7.py

def empate(tablero):
    """ Revisa si hay un empate """
    for i in range(3):
        for j in range(3):
            if tablero[i][j]==' ':
                return False
    return True

\end{verbatim}
\afterverb

Llamamos a empate después de cada jugador:
\beforeverb
\begin{verbatim}
# triqui8.py
while True:
    if jugar("O",triqui):
        print "Gana el jugador O !!!!"
        break
    if empate(triqui):
        print "Empate !!!"
        break
    if jugar("X",triqui):
        print "Gana el jugador X !!!!"
        break
    if empate(triqui):
        print "Empate !!!"
        break
\end{verbatim}
\afterverb

Y también agregamos un mensaje de retroalimentación para el jugador cuando no 
ha escogido una casilla válida:

\beforeverb
\begin{verbatim}
# triqui8.py
def jugar(jugador,tablero):
    """ Registra una jugada en el tablero de juego
        retorna True si el jugador gana la partida
    """
    while True:     
        print "Juegue jugador ", jugador
        f = input("fila? ")
        c = input("columna? ")
        if type(f)==type(c)==type(1) and valido(f) and valido(c) and tablero[f][c]==' ':
            tablero[f][c] = jugador
            break
        else:
            # Añadimos este mensaje de error
            print "Posición inválida!"

    imprimir(triqui)
    diag = ganaDiagonal1(jugador,tablero) or ganaDiagonal2(jugador,tablero)
    linea = ganaHorizontal(jugador,tablero) or ganaVertical(jugador,tablero)
    return  diag or linea
\end{verbatim}
\afterverb

Finalmente, tenemos un programa completo que permite jugar triqui, textualmente, en 
el archivo triqui9.py.

