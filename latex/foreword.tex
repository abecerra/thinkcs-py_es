
\chapter{Prólogo}

Por David Beazley

Como educador, investigador y autor de libros, estoy encantado de
ver la terminación de este texto. Python es un lenguaje de programación
divertido y extremadamente fácil de usar que ha ganado renombre constantemente
en los años recientes. Desarrollado hace diez años por Guido van Rossum,
la sintaxis simple de Python y su ``sabor'' se derivan, en gran
parte del ABC, un lenguaje de programación para enseñanza que se desarrolló
en los 1980s. Sin embargo, Python también fue creado para resolver
problemas reales y tiene una amplia gama de características que se
encuentran en lenguajes de programación como C++, Java, Modula-3,
y Scheme. Debido a esto, uno de las características notables de Python
es la atracción que ejerce sobre programadores profesionales, científicos,
investigadores, artistas y educadores.

A pesar de ésta atracción que ejerce en muchas comunidades diversas,
usted puede todavía preguntarse ``¿porque Python?'' o ``¿porque
enseñar programación con Python?'' Responder éstas preguntas no es
una tarea fácil— especialmente cuando la opinión popular está del
lado masoquista de usar alternativas como C++ y Java. Sin embargo,
pienso que la respuesta más directa es que la programación en Python
es simplemente más divertida y más productiva.

Cuando enseño cursos de informática, yo quiero cubrir conceptos importantes,
hacer el material interesante y enganchar a los estudiantes. Desafortunadamente,
hay una tendencia en la que los cursos de programación introductorios
dedican demasiada atención hacia la abstracción matemática y a hacer
que los estudiantes se frustren con problemas molestos relacionados
con la sintaxis, la compilación y la presencia de reglas arcanas en
los lenguajes. Aunque la abstracción y el formalismo son importantes
para los ingenieros de software y para los estudiantes de ciencias
de la computación, usar este enfoque hace la informática muy aburrida.
Cuando enseño un curso no quiero tener un grupo de estudiantes sin
inspiración. Quisiera verlos intentando resolver problemas interesantes,
explorando ideas diferentes, intentando enfoques no convencionales,
rompiendo reglas y aprendiendo de sus errores. En el proceso no quiero
perder la mitad del semestre tratando de resolver problemas sintácticos
oscuros, interpretando mensajes de error del compilador incomprensibles,
o descifrando cuál de las muchas maneras en que un programa puede
generar un error grave de memoria se está presentando.

Una de las razones del por qué me gusta Python es que proporciona
un equilibrio muy bueno entre lo práctico y lo conceptual. Puesto
que se interpreta Python, los principiantes pueden empezar a hacer
cosas interesantes casi de inmediato sin perderse en problemas de
compilación y enlace. Además, Python viene con una biblioteca grande
de módulos, que pueden ser usados en dominios que van desde programación
en la web hasta aplicaciones gráficas. Tener un foco práctico es una
gran manera de enganchar a los estudiantes y permite que emprendan
proyectos significativos. Sin embargo, Python también puede servir
como una excelente base para introducir conceptos importantes de la
informática. Puesto que Python soporta completamente procedimientos
y clases, los estudiantes pueden ser introducidos gradualmente a temas
como la abstracción procedimental, las estructuras de datos y la programación
orientada a objetos—lo que se puede aplicar después a cursos posteriores
en Java o C++. Python proporciona, incluso, varias características
de los lenguajes de programación funcionales y puede usarse para introducir
conceptos que se pueden explorar con más detalle en cursos con Scheme
y Lisp.

Leyendo, el prefacio de Jeffrey, estoy sorprendido por sus comentarios
de que Python le permita ver un ``más alto nivel de éxito y un nivel
bajo de frustración'' y que puede ``avanzar mas rápido con mejores
resultados.'' Aunque estos comentarios se refieren a sus cursos introductorios,
a veces uso Python por estas mismas razones en los cursos de informática
avanzada en la Universidad de Chicago. En estos cursos enfrento constantemente
la tarea desalentadora de cubrir un montón de material difícil durante
nueve semanas. Aunque es totalmente posible para mi infligir mucho
dolor y sufrimiento usando un lenguaje como C++, he encontrado a menudo
que este enfoque es improductivo—especialmente cuando el curso se
trata de un asunto sin relación directa con la ``programación.''
He encontrado que usar Python me permite enfocar el tema del curso
y dejar a los estudiantes desarrollar proyectos substanciales.

Aunque Python siga siendo un lenguaje joven y en desarrollo, creo
que tiene un futuro brillante en la educación. Este libro es un paso
importante en esa dirección.

\vspace{0.25in}
 
\begin{flushleft}
David Beazley \\
 Universidad de Chicago, Autor de {\em Python Essential Reference} 
\par\end{flushleft}
