% LaTeX source for textbook ``How to think like a computer scientist''
% Copyright (c)  2001  Allen B. Downey, Jeffrey Elkner, and Chris Meyers.

% Permission is granted to copy, distribute and/or modify this
% document under the terms of the GNU Free Documentation License,
% Version 1.1  or any later version published by the Free Software
% Foundation; with the Invariant Sections being "Contributor List",
% with no Front-Cover Texts, and with no Back-Cover Texts. A copy of
% the license is included in the section entitled "GNU Free
% Documentation License".

% This distribution includes a file named fdl.tex that contains the text
% of the GNU Free Documentation License.  If it is missing, you can obtain
% it from www.gnu.org or by writing to the Free Software Foundation,
% Inc., 59 Temple Place - Suite 330, Boston, MA 02111-1307, USA.
%

\chapter{Interludio 2: Creando un nuevo tipo de datos}
\label{overloading}
\index{tipo de dato!definido por el usuario}

Los lenguajes de programación orientados a objetos permiten a los
programadores crear nuevos tipos de datos que se comportan de 
manera muy similar a los tipos primitivos. Exploraremos esta 
característica construyendo una clase \texttt{Fraccionario} que
se comporte como los tipos de datos numéricos primitivos (enteros
y flotantes).

Los números fraccionario o racionales son valores que se pueden 
expresar como una división entre dos números enteros, como  $\frac{1}{3}$.  
El número superior es el numerador y el inferior es
es el denominador.

\index{racional}
\index{fracción}
\index{numerador}
\index{denominador}

La clase  \texttt{Fraccion} empieza con un método constructor que
recibe como parámetros al numerador y al denominador:

%\adjustpage{-2}
%\pagebreak

%\beforeverb
\begin{verbatim}
class Fraccion:
  def __init__(self, numerador, denominador=1):
    self.numerador = numerador
    self.denominador = denominador
\end{verbatim}
%\afterverb
%
El denominador es opcional.  Una Fracción con un solo 
parámetro representa a un número entero.  Si el numerador es
$n$, construimos la fracción $n/1$.

El siguiente paso consiste en escribir un método \texttt{\_\_str\_\_} que despliegue las fracciones de 
una manera natural. Como estamos acostumbrados a la notación
``numerador/denominador'', lo más natural es:

\beforeverb
\begin{verbatim}
class Fraccion:
  ...
  def __str__(self):
    return "%d/%d" % (self.numerador, self.denominador)
\end{verbatim}
\afterverb
%
Para realizar pruebas, ponemos este código en un archivo
\texttt{Fraccion.py} y lo importamos en el intérprete
de Python. Ahora creamos un objeto fracción y lo imprimimos.

\beforeverb
\begin{verbatim}
>>> from Fraccion import Fraccion
>>> s = Fraccion(5,6)
>>> print "La fraccion es", s
La fraccion es 5/6
\end{verbatim}
\afterverb
%
El método \texttt{print}, automáticamente invoca al método 
\texttt{\_\_str\_\_} de manera implícita.


\section {Multiplicación de fracciones}
\index{multiplicación!de fracciones}
\index{fracciones!multiplicación}

Nos gustaría aplicar los mismos operadores de suma, resta, 
multiplicación y división a las fracciones. Para lograr esto
podemos sobrecargar los operadores matemáticos en la clase
\texttt{Fraccion}.

\index{sobrecarga}
\index{operadores!sobrecarga de}
\index{operador matemático}

La multiplicación es la operación más sencilla entre fraccionarios. El
resultado de multiplicar dos fracciones a y v es una nueva fracción en la que el numerador es el producto de los dos numeradores (de a y b) y el denominador es el producto de los dos denominadores (de a y b). 

Python define que  el método  \texttt{\_\_mul\_\_} se puede definir
en una clase para sobrecargar el operador \texttt{*}:

\beforeverb
\begin{verbatim}
class Fraccion:
  ...
  def __mul__(self, otro):
    return Fraccion(self.numerador*otro.numerador,
                    self.denominador*otro.denominador)
\end{verbatim}
\afterverb
%
Podemos probar este método calculando un producto sencillo:

\beforeverb
\begin{verbatim}
>>> print Fraccion(5,6) * Fraccion(3,4)
15/24
\end{verbatim}
\afterverb
%
Funciona, pero se puede mejorar. Podemos manejar el caso en el 
que se multiplique una fracción por un número entero. Por medio
de la función \texttt{type} se puede probar si \texttt{otro} es
un entero y convertirlo a una fracción antes de realizar el
producto:

\beforeverb
\begin{verbatim}
class Fraccion:
  ...
  def __mul__(self, otro):
    if type(otro) == type(5):
      otro = Fraccion(otro)
    return Fraccion(self.numerador   * otro.numerador,
                    self.denominador * otro.denominador)
\end{verbatim}
\afterverb
%
Ahora, la multiplicación entre enteros y fracciones funciona, pero 
sólo si la fracción es el operando a la izquierda :

\beforeverb
\begin{verbatim}
>>> print Fraccion(5,6) * 4
20/6
>>> print 4 * Fraccion(5,6)
TypeError: __mul__ nor __rmul__ defined for these operands
\end{verbatim}
\afterverb
%
Para evaluar un operador binario como la multiplicación, Python
chequea el operando izquierdo primero, para ver si su clase 
define el método  \texttt{\_\_mul\_\_}, y que tenga soporte para
el tipo del segundo operando. En este caso el operador primitivo
para multiplicar enteros no soporta las fracciones.

Después, Python chequea si el operando a la derecha provee un 
método  \texttt{\_\_rmul\_\_} que soporte el tipo del operando de
la izquierda.  En este caso, como no hay definición de  \texttt{\_\_rmul\_\_} en la clase \texttt{Fraccion}, se genera
un error de tipo. 

Hay una forma sencilla de definir \texttt{\_\_rmul\_\_}:

\beforeverb
\begin{verbatim}
class Fraccion:
  ...
  __rmul__ = __mul__
\end{verbatim}
\afterverb
%
Esta asignación dice que  \texttt{\_\_rmul\_\_} contiene el mismo
código que \texttt{\_\_mul\_\_}. Si ahora evaluamos 
\texttt{4 * Fraccion(5,6)}, Python llama a \texttt{\_\_rmul\_\_} y le pasa
al 4 como parámetro:

\beforeverb
\begin{verbatim}
>>> print 4 * Fraccion(5,6)
20/6
\end{verbatim}
\afterverb
%
Como  \texttt{\_\_rmul\_\_} tiene el mismo código que  \texttt{\_\_mul\_\_}, 
y el método \texttt{\_\_mul\_\_} puede recibir un parámetro entero, 
nuestra multiplicación de fracciones funciona bien.


\section{Suma de fracciones}
\index{suma!de fracciones}
\index{fracciones!suma}

La suma es más complicada que la multiplicación. La suma 
$a/b + c/d$ da como resultado $\frac{(ad+cb)}{bd}$.

Basándonos en la multiplicación, podemos escribir los métodos
\texttt{\_\_add\_\_} y \texttt{\_\_radd\_\_}:

\beforeverb
\begin{verbatim}
class Fraccion:
  ...
  def __add__(self, otro):
    if type(otro) == type(5):
      otro = Fraccion(otro)
    return Fraccion(self.numerador   * otro.denominador +
                    self.denominador * otro.numerador,
                    self.denominador * otro.denominador)

  __radd__ = __add__
\end{verbatim}
\afterverb
%
Podemos probar estos métodos con objetos  \texttt{Fraccion} y 
con números enteros.

\beforeverb
\begin{verbatim}
>>> print Fraccion(5,6) + Fraccion(5,6)
60/36
>>> print Fraccion(5,6) + 3
23/6
>>> print 2 + Fraccion(5,6)
17/6
\end{verbatim}
\afterverb
%
Los primeros ejemplos llaman al método \texttt{\_\_add\_\_}; el 
último ejemplo llama al método \texttt{\_\_radd\_\_}.


\section{El algoritmo de Euclides}
\index{máximo divisor común}
\index{Euclides}
\index{pseudocódigo}
\index{simplificar}

En el ejemplo anterior, calculamos $5/6 + 5/6$ y obtuvimos
$60/36$.  Es correcto, pero no es la manera más sencilla de 
presentar la respuesta. Para {\bf simplificar} la fracción tenemos
que dividir el numerador y el denominador por su  {\bf máximo 
divisor común (MDC)}, que para este caso es 12.  Entonces, un
 resultado mas sencillo es $5/3$.

En general, cada vez que creamos un nuevo objeto de tipo \texttt{Fraccion} deberiamos simplificarlo dividiendo el 
numerador y el denominador por su MDC. Si la fracción no se puede
simplificar, el MDC es 1.

Euclides de Alejandría (aprox. 325--265 A.C) presentó un algoritmo
para encontrar el MDC de dos números enteros $m$ y $n$:

\begin{quote}
Si  $n$ divide a  $m$ exactamente, entonces $n$ es el MDC. Sino,
el MDC de $m$ y $n$ es el MDC de $n$ y el residuo de la división 
 $m/n$.
\end{quote}

Esta definición recursiva se puede implementar en una función:

\beforeverb
\begin{verbatim}
def MDC (m, n):
  if m % n == 0:
    return n
  else:
    return MDC(n, m%n)
\end{verbatim}
\afterverb
%
En la primera línea el operador residuo nos permite chequear
si n divide a n exactamente. En la última línea, lo usamos
para calcular el residuo de la división.

Como todas las operaciones que hemos escrito crean nuevas
fracciones como resultado, podemos simplificar todos los 
valores de retorno modificando el método constructor.

\beforeverb
\begin{verbatim}
class Fraccion:
  def __init__(self, numerador, denominador=1):
    g = MDC (numerador, denominador)
    self.numerador   =   numerador / g
    self.denominador = denominador / g
\end{verbatim}
\afterverb
%
Ahora, cada vez que creamos una nueva  \texttt{Fraccion}, ¡se
simplifica!.

\beforeverb
\begin{verbatim}
>>> Fraccion(100,-36)
-25/9
\end{verbatim}
\afterverb
%
Una característica adicional que nos provee  \texttt{MDC} 
es que si la fracción es negativa, el signo menos siempre 
se mueve hacia el numerador.


\section{Comparando fracciones}
\index{comparación!de fracciones}
\index{fracciones!comparación de}

Si vamos a comparar dos objetos  \texttt{Fraccion}, digamos 
\texttt{a} y \texttt{b}, evaluando la expresión \texttt{a == b}. 
Como la implementación de  \texttt{==} chequea igualdad superficial
de objetos por defecto, sólo retornará cierto si  \texttt{a}
y \texttt{b} \textit{son} el mismo objeto.

Es mucho más probable que deseemos retornar cierto si  $a$ y $b$ 
tienen el mismo valor ---esto es, chequear igualdad profunda.

Tenemos que enseñarle a las fracciones cómo compararse entre sí.
Como vimos en la sección \ref{comparecard}, podemos sobrecargar
todos los operadores de comparación por medio de la implementación
de un método  \texttt{\_\_cmp\_\_}.

Por convención, el método  \texttt{\_\_cmp\_\_} retorna un 
número negativo si  \texttt{self} es menos que \texttt{otro}, cero
si son iguales, y un número positivo si \texttt{self} es 
mayor que  \texttt{otro}.

La forma más sencilla de comparar fracciones consiste en 
hacer una multiplicación cruzada. Si  $a/b > c/d$, entonces $ad > bc$.
Con esto en mente, implementamos \texttt{\_\_cmp\_\_}:

\beforeverb
\begin{verbatim}
class Fraccion:
  ...
  def __cmp__(self, otro):
    dif = (self.numerador  * otro.denominador -
            otro.numerador * self.denominador)
    return dif
\end{verbatim}
\afterverb
%
Si \texttt{self} es mayor que \texttt{otro}, entonces \texttt{dif}
será positivo.  Si \texttt{otro} es mayor, \texttt{dif}
será negativo.  Si son iguales, \texttt{dif} es cero.


\section {Extendiendo las fracciones}

Todavía no hemos terminado. Tenemos que implementar 
la resta sobrecargando el método  \texttt{\_\_sub\_\_} y la
división con el método \texttt{\_\_div\_\_}.

Podemos restar por medio de la suma si antes negamos (cambiamos de 
signo) al segundo operando. También podemos dividir por medio de
la multiplicación si antes invertimos el segundo operando.

Siguiendo este razonamiento, una forma de realizar las operaciones 
resta y división consiste en definir primero la negación por medio 
de la sobrecarga de \texttt{\_\_neg\_\_} y la inversión sobre
sobrecargando a \texttt{\_\_invert\_\_}.

Un paso adicional sería implementar \texttt{\_\_rsub\_\_} y \texttt{\_\_rdiv\_\_}.
Desafortunadamente no podemos usar el mismo truco que aplicamos
para la suma y la multiplicación, porque la resta y la división 
no son conmutativas. En estas operaciones el orden de los operandos
altera el resultado, así que no podemos asignar a \texttt{\_\_rsub\_\_} y 
a \texttt{\_\_rdiv\_\_} los método \texttt{\_\_sub\_\_} y \texttt{\_\_div\_\_},
respectivamente.  

Para realizar la  {\bf negación unaria}, sobrecargamos a  \texttt{\_\_neg\_\_}.

\index{operador unario}
\index{negación}

Podemos calcular potencias sobrecargando a \texttt{\_\_pow\_\_},
pero la implementación tiene un caso difícil: si el exponente no 
es un entero, puede que no sea posible representar el resultado
como una  \texttt{Fraccion}.  Por ejemplo, la siguiente expresión \texttt{Fraccion(2) ** Fraccion(1,2)}
es la raíz cuadrada de 2, que es un número irracional (no puede
representarse por ninguna fracción). Así que no es fácil escribir 
una función general para  \texttt{\_\_pow\_\_}.

\index{irracional}


Hay otra extensión a la clase  \texttt{Fraccion} que usted puede
imaginar. Hasta aquí, hemos asumido que el numerador y el 
denominador son enteros. También podemos permitir que sean 
de tipo long.

\begin{quote}
{\em  Como ejercicio, complemente la implementación de la clase
 \texttt{Fraccion} para que permita las operaciones de resta,
división, exponenciación. Además, debe soportar denominadores y 
numeradores de tipo long (enteros grandes)}.
\end{quote}



\section{Glosario}

\begin{description}

\item[Máximo divisor común (MDC):] el entero positivo más grande
que divide exactamente a dos números (por ejemplo, el numerador y 
el denominador en una fracción).

\item[Simplificar:] cambiar una fracción en otra equivalente que
tenga un MDC de 1.

\item[negación unaria:] la operación que calcula un inverso 
aditivo, usualmente representada con un signo menos. Es  
``unaria'' en contraposición con el menos binario que 
representa a la resta.


\index{máximo divisor común}
\index{simplificar}
\index{negación unaria}

\end{description}
