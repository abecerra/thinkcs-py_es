%                                    This will higlight source code like Code::Blocks IDE.
%                                    Install this theme copying codeblocks.py to
%                                    "/usr/local/lib/python2.7/dist-packages/pygments/styles/"

\usepackage{upquote}               % Show "realistic" quotes in verbatim.
                                   % To use single straight quote in minted with pygmentize 1.6
\usepackage{minted}                % Usamos minted para el resaltado de sintaxis de los códigos.
\usemintedstyle{codeblocks}        % Elegimos el perfil de "codeblocks".

\label{PYTHON_CODE}
% Con newminted{python} definimos un nuevo ambiente:
%	\begin{pythoncode}
%#!/usr/bin/env python
%# coding: utf-8
%	
%print("Hello World!")
%	\end{pythoncode}

\newminted{python}{
	linenos=true,
	numbersep=5pt,                 % Setea la separación de los números de línea en 5pt. 
	frame=single,
	framesep=2mm,                  % Setea la separación del frame a 2mm.
%	gobble=2,                      % Borra los 2 espacios adicionales agregados.
%	mathescape,                    % Permite insertar secuencias de escape matemáticas de LaTeX.
	rulecolor=\color{gray15p},     % color de la línea del recuadro.
	framerule=0.4mm,               % ancho de línea de recuadro
	bgcolor=gray05p,               % color de fondo del recuadro.
	fontsize=\small,
}

\label{PYTHON_FILE}
% Con \newmintedfile[pythonfile]{python} definimos un nuevo comando:
%\pythonfile{./src/test.py}

\newmintedfile[pythonfile]{python}{
	linenos=true,
	numbersep=5pt,                 % Setea la separación de los números de línea en 5pt. 
	frame=single,
	framesep=2mm,                  % Setea la separación del frame a 2mm.
%	gobble=2,                      % Borra los 2 espacios adicionales agregados.
%	mathescape,                    % Permite insertar secuencias de escape matemáticas de LaTeX.
	rulecolor=\color{gray15p},     % color de la línea del recuadro.
	framerule=0.4mm,               % ancho de línea de recuadro
	bgcolor=gray05p,               % color de fondo del recuadro.
	fontsize=\small,
}

\label{C_CODE}
% Con \newminted{c} definimos un nuevo ambiente:
%	\begin{ccode}
%#include <stdio.h>
%
%int main (int argc, char** argv)
%{
%    printf("Hola Mundo!\n");
%
%    return 0;
%}
%	\end{ccode}
	
\newminted{c}{
	linenos=true,
	numbersep=5pt,                 % Setea la separación de los números de línea en 5pt. 
	frame=single,
	framesep=2mm,                  % Setea la separación del frame a 2mm.
%	gobble=2,                      % Borra los 2 espacios adicionales agregados.
%	mathescape,                    % Permite insertar secuencias de escape matemáticas de LaTeX.
	rulecolor=\color{gray15p},     % color de la línea del recuadro.
	framerule=0.4mm,               % ancho de línea de recuadro
	bgcolor=gray05p,               % color de fondo del recuadro.
	fontsize=\small,
}

\label{C_FILE}
%Con \newmintedfile[cfile]{c} definimos un nuevo comando:
%\cfile{./src/test.c}

\newmintedfile[cfile]{c}{
	linenos=true,
	numbersep=5pt,                 % Setea la separación de los números de línea en 5pt. 
	frame=single,
	framesep=2mm,                  % Setea la separación del frame a 2mm.
%	gobble=2,                      % Borra los 2 espacios adicionales agregados.
%	mathescape,                    % Permite insertar secuencias de escape matemáticas de LaTeX.
	rulecolor=\color{gray15p},     % color de la línea del recuadro.
	framerule=0.4mm,               % ancho de línea de recuadro
	bgcolor=gray05p,               % color de fondo del recuadro.
	fontsize=\small,
}

\label{PYTHON_CONSOLE}
% Con newminted{pycon} definimos un nuevo ambiente:
%	\begin{pyconcode}
%>>> print("Hello World!")
%Hello World!
%>>>
%	\end{pyconcode}

\newminted{pycon}{
	linenos=true,
	numbersep=5pt,                 % Setea la separación de los números de línea en 5pt. 
	frame=single,
	framesep=2mm,                  % Setea la separación del frame a 2mm.
%	gobble=2,                      % Borra los 2 espacios adicionales agregados.
%	mathescape,                    % Permite insertar secuencias de escape matemáticas de LaTeX.
	rulecolor=\color{gray15p},     % color de la línea del recuadro.
	framerule=0.4mm,               % ancho de línea de recuadro
	bgcolor=gray05p,               % color de fondo del recuadro.
	fontsize=\small,
}

\label{PYTHON_CONSOLE_FILE}
% Con \newmintedfile[pyconfile]{python} definimos un nuevo comando:
%\pyconfile{./src/test.py}

\newmintedfile[pyconfile]{pycon}{
	linenos=true,
	numbersep=5pt,                 % Setea la separación de los números de línea en 5pt. 
	frame=single,
	framesep=2mm,                  % Setea la separación del frame a 2mm.
%	gobble=2,                      % Borra los 2 espacios adicionales agregados.
%	mathescape,                    % Permite insertar secuencias de escape matemáticas de LaTeX.
	rulecolor=\color{gray15p},     % color de la línea del recuadro.
	framerule=0.4mm,               % ancho de línea de recuadro
	bgcolor=gray05p,               % color de fondo del recuadro.
	fontsize=\small,
}

\renewcommand\listingscaption{Code}
\addto\captionsspanish{
	\renewcommand{\listingscaption}{Código}
}