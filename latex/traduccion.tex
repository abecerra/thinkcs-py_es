
\chapter{Traducción al español}

Al comienzo de junio de 2007 tomé la iniciativa de traducir el texto
``How to think like a Computer Scientist, with Python'' al español.
Rápidamente me dí cuenta de que ya había un trabajo inicial de traducción
empezado por:
\begin{itemize}
\item Angel Arnal 
\item I Juanes 
\item Litza Amurrio 
\item Efrain Andia
\end{itemize}
Ellos habían traducido los capítulos 1,2,10,11, y 12, así como el
prefacio, la introducción y la lista de colaboradores. Tomé su valioso
trabajo como punto de partida, adapté los capítulos, traduje las secciones
faltantes del libro y añadí un primer capítulo adicional sobre solución
de problemas.

Aunque el libro traduce la primera edición del original, todo se ha
corregido para que sea compatible con Python 2.7, por ejemplo se usan
booleanos en vez de enteros en los condicionales y ciclos.

Para realizar este trabajo ha sido invaluable la colaboración de familiares,
colegas, amigos y estudiantes que han señalado errores, expresiones
confusas y han aportado toda clase de sugerencias constructivas. Mi
agradecimiento va para los traductores antes mencionados y para los
estudiantes de Biología que tomaron el curso de Informática en la
Pontificia Universidad Javeriana (Cali-Colombia), durante el semestre
2014-1:
\begin{itemize}
\item Estefanía Lopez 
\item Gisela Chaves 
\item Marlyn Zuluaga 
\item Francisco Sanchez 
\item María del Mar Lopez 
\item Diana Ramirez 
\item Guillermo Perez 
\item María Alejandra Gutierrez 
\item Sara Rodriguez 
\item Claudia Escobar
\item Yisveire Fontalvo
\end{itemize}
\vspace{0.25in}
 

Para la segunda edición todo el código fuente se cambió para ejecutarse
con Python 3 y se añadieron 2 capítulos interludios y un posludio.
como capítulo final.
\begin{flushleft}
Andrés Becerra Sandoval \\
 Universidad Santiago de Cali \\
 andres.becerra00@usc.edu.co \\
\par\end{flushleft}
