
\chapter{Pilas}

\section{Tipos abstractos de datos}

\index{tipos abstractos de datos} \index{TAD} \index{encapsulamiento}

Los tipos de datos que ha visto hasta el momento son concretos, en
el sentido que hemos especificado completamente como se implementan.
Por ejemplo, la clase \texttt{Carta} representa una carta por medio
de dos enteros. Pero esa no es la única forma de representar una carta;
hay muchas representaciones alternativas.

Un \textbf{tipo abstracto de datos}, o TAD, especifica un conjunto
de operaciones (o métodos) y la semántica de las operaciones (lo que
hacen), pero no especifica la implementación de las operaciones. Eso
es lo que los hace abstractos.

¿Qué es lo que los hace tan útiles?
\begin{itemize}
\item La tarea de especificar un algoritmo se simplifica si se pueden denotar
las operaciones sin tener que pensar al mismo tiempo como se implementan.
\item Como usualmente hay muchas formas de implementar un TAD, puede ser
provechoso escribir un algoritmo que pueda usarse con cualquier implementación
alternativa.
\item Los TADs bien conocidos, como el TAD Pila de este capítulo, a menudo
se encuentran implementados en las bibliotecas estándar de los lenguajes
de programación, así que pueden escribirse una sola vez y usarse muchas
veces.
\item Las operaciones de los TADs nos proporcionan un lenguaje de alto nivel
para especificar algoritmos.
\end{itemize}
Cuando hablamos de TADs hacemos la distinción entre el código que
utiliza el TAD, denominado código \textbf{cliente}, del código que
implementa el TAD, llamado código \textbf{proveedor}.

\index{cliente} \index{proveedor}

\section{El TAD Pila}

\index{pila} \index{colección} \index{TAD!Pila}

Como ya hemos aprendido a usar otras colecciones como los diccionarios
y las listas, en este capítulo exploraremos un TAD muy general, la
\textbf{pila}.

Una pila es una colección, esto es, una estructura de datos que contiene
múltiples elementos. \index{interfaz}

Un TAD se define por las operaciones que se pueden ejecutar sobre
él, lo que recibe el nombre de \textbf{interfaz}. La interfaz de una
pila comprende las siguientes operaciones:
\begin{description}
\item [{texttt{\_\_init\_\_}:}] inicializa una pila vacía.
\item [{texttt{meter}:}] agrega un objeto a la pila.
\item [{texttt{sacar}:}] elimina y retorna un elemento de la pila. El
objeto que se retorna siempre es el último que se agregó.
\item [{texttt{estaVacia}:}] revisa si la pila está vacía.
\end{description}
Una pila también se conoce como una estructura``último que Entra,
Primero que Sale '' o UEPS, porque el último dato que entró es el
primero que va a salir.

\section{Implementando pilas por medio de listas de Python}

\index{Pila} \index{clase!Pila} \index{Estructura de datos genérica}
\index{Estructura de datos!genérica}

Las operaciones de listas que Python proporciona son similares a las
operaciones que definen una pila. La interfaz no es lo que uno se
espera, pero podemos escribir código que traduzca desde el TAD pila
a las operaciones primitivas de Python.

Este código se denomina la \textbf{implementación} del TAD Pila. En
general, una implementación es un conjunto de métodos que satisfacen
los requerimientos sintácticos y semánticos de una interfaz.

Aquí hay una implementación del TAD Pila que usa una lista de Python:

\beforeverb 
\begin{pythoncode}
class Pila :
  def __init__(self) :
    self.items = []

  def meter(self, item) :
    self.items.append(item)

  def sacar(self) :
    return self.items.pop()

  def estaVacia(self) :
    return (self.items == [])
\end{pythoncode}
\afterverb Una objeto \texttt{Pila} contiene un atributo llamado
\texttt{items} que es una lista de los objetos que están en la Pila.
El método de inicialización le asigna a \texttt{items} la lista vacía.

Para meter un nuevo objeto en la Pila, \texttt{meter} lo pega a \texttt{items}.
Para sacar un objeto de la Pila, \texttt{sacar} usa al método \texttt{pop}
que proporciona Python para eliminar el último elemento de una lista.

Finalmente, para verificar si la Pila está vacía, \texttt{estaVacia}
compara a \texttt{items} con la lista vacía.

\index{barniz}

Una implementación como ésta, en la que los métodos son simples llamados
de métodos existentes, se denomina \textbf{barniz}. En la vida real,
el barniz es una delgada capa de protección que se usa algunas veces
en la fabricación de muebles para ocultar la calidad de la madera
que recubre. Los científicos de la computación usan esta metáfora
para describir una pequeña porción de código que oculta los detalles
de una implementación para proporcionar una interfaz más simple o
más estandarizada.

\section{Meter y sacar}

\index{meter} \index{sacar} \index{estructura de datos genérica}
\index{estructura de datos!genérica}

Una Pila es una \textbf{estructura de datos genérica}, o sea que podemos
agregar datos de cualquier tipo a ella. El siguiente ejemplo mete
dos enteros y una cadena en la Pila:

\beforeverb 
\begin{pyconcode}
>>> s = Pila()
>>> s.meter(54)
>>> s.meter(45)
>>> s.meter("+")
\end{pyconcode}
\afterverb Podemos usar los métodos \texttt{estaVacia} y \texttt{sacar}
para eliminar e imprimir todos los objetos en la Pila:

\beforeverb 
\begin{pythoncode}
while not s.estaVacia() :
  print(s.sacar()),
\end{pythoncode}
\afterverb La salida es \texttt{+ 45 54}. En otras palabras, acabamos
de usar la Pila para imprimir los objetos al revés, ¡y de una manera
muy sencilla!

Compare esta porción de código con la implementación de \texttt{imprimirAlReves}
en la Sección~\ref{listrecursion}. Hay una relación muy profunda
e interesante entre la versión recursiva de \texttt{imprimirAlReves}
y el ciclo anterior. La diferencia reside en que \texttt{imprimirAlReves}
usa la Pila que provee el ambiente de ejecución de Python para llevar
pista de los nodos mientras recorre la lista, y luego los imprime
cuando la recursión se empieza a devolver. El ciclo anterior hace
lo mismo, pero explícitamente por medio de un objeto Pila.

\section{Evaluando expresiones postfijas con una Pila}

\index{postfija} \index{infija} \index{expresión}

En la mayoría de los lenguajes de programación las expresiones matemáticas
se escriben con el operador entre los operandos, como en \texttt{1+2}.
Esta notación se denomina \textbf{infija}. Una alternativa que algunas
calculadoras usan se denomina notación \textbf{postfija}. En la notación
postfija, el operador va después de los operandos, como en \texttt{1
2 +}.

La razón por la que esta notación es útil reside en que hay una forma
muy natural de evaluar una expresión postfija usando una Pila:
\begin{itemize}
\item Comenzando por el inicio de la expresión, ir leyendo cada término
(operador u operando).

\begin{itemize}
\item Si el término es un operando, meterlo en la Pila.
\item Si el término es un operador, sacar dos operandos de la Pila, ejecutar
la operación sobre ellos, y meter el resultado en la Pila.
\end{itemize}
\item Cuando llegue al final de la expresión, tiene que haber un solo aperando
en la Pila, ese es el resultado de la expresión.
\end{itemize}
\begin{quote}
{\em Como ejercicio, aplique este algoritmo a la expresión \texttt{1
2 + 3 {*}}.} 
\end{quote}
Este ejemplo demuestra una de las ventajas de la notación postfija—no
hay necesidad de usar paréntesis para controlar el orden de las operaciones.
Para obtener el mismo resultado en notación infija tendríamos que
haber escrito \texttt{(1 + 2) {*} 3}.
\begin{quote}
{\em Como ejercicio, escriba una expresión postfija equivalente
a \texttt{1 + 2 {*} 3}.} 
\end{quote}

\section{Análisis sintáctico}

\index{análisis sintáctico} \index{lexema} \index{delimitador}
\index{expresión regular}

Para implementar el algoritmo anterior, necesitamos recorrer una cadena
y separarla en operandos y operadores. Este proceso es un ejemplo
de \textbf{análisis sintáctico}, y los resultados –los trozos individuales
que obtenemos—se denominan \textbf{lexemas}. Tal vez recuerde estos
conceptos introducidos en el capítulo 2.

Python proporciona el método \texttt{split} en los módulos \texttt{string}
y \texttt{re} (expresiones regulares). La función \texttt{string.split}
parte una cadena en una lista de cadenas usando un caracter como \textbf{delimitador}.
Por ejemplo:

\beforeverb 
\begin{pyconcode}
>>> import string
>>> string.split("Ha llegado la hora"," ")
['Ha', 'llegado', 'la', 'hora']
\end{pyconcode}
\afterverb En este caso, el delimitador es el caracter espacio, así
que la cadena se separa cada vez que se encuentra un espacio.

La función \texttt{re.split} es mas poderosa, nos permite especificar
una expresión regular en lugar de un delimitador. Una expresión regular
es una forma de especificar un conjunto de cadenas. Por ejemplo, \verb+[A-Z]+
es el conjunto de todas las letras y \verb+[0-9]+ es el conjunto
de todos los números. El operador \verb+^+ niega un conjunto, así
que \verb+[^0-9]+ es el conjunto complemento al de números (todo
lo que no es un número), y esto es exactamente lo que deseamos usar
para separar una expresión postfija:

\beforeverb 
\begin{pyconcode}
>>> import re
>>> re.split("([^0-9])", "123+456*/")
['123', '+', '456', '*', '', '/', '']
\end{pyconcode}
\afterverb Observe que el orden de los argumentos es diferente al
que se usa en la función \texttt{string.split}; el delimitador va
antes de la cadena.

La lista resultante contiene a los operandos \texttt{123} y \texttt{456},
y a los operadores \texttt{{*}} y \texttt{/}. También incluye dos
cadenas vacías que se insertan después de los operandos.

\section{Evaluando expresiones postfijas}

Para evaluar una expresión postfija, utilizaremos el analizador sintáctico
de la sección anterior y el algoritmo de la anterior a esa. Por simplicidad,
empezamos con un evaluador que solo implementa los operadores \texttt{+}
y \texttt{{*}}:

\adjustpage{-3} %\pagebreak

%\beforeverb
\begin{pythoncode}
def evalPostfija(expr):
  import re
  listaLexemas = re.split("([^0-9])", expr)
  Pila = Pila()
  Por lexema in listaLexemas:
    if  lexema == '' or lexema == ' ':
      continue
    if  lexema == '+':
      suma = Pila.sacar() + Pila.sacar()
      Pila.meter(suma)
    elif lexema == '*':
      producto = Pila.sacar() * Pila.sacar()
      Pila.meter(producto)
    else:
      Pila.meter(int(lexema))
  return Pila.sacar()
\end{pythoncode}
%\afterverbLa primera condición ignora los espacios y las cadenas
vacías. Las siguientes dos condiciones detectan los operadores. Asumimos
por ahora —intrépidamente—, que cualquier caracter no numérico es
un operando.

Verifiquemos la función evaluando la expresión \texttt{(56+47){*}2}
en notación postfija:

\beforeverb 
\begin{pyconcode}
>>> print(evalPostfija ("56 47 + 2 *"))
206
\end{pyconcode}
\afterverb Bien, por ahora.

\section{Clientes y proveedores}

\index{encapsulamiento} \index{TAD}

Una de los objetivos fundamentales de un TAD es separar los intereses
del proveedor, el que escribe el código que implementa el Tipo Abstracto
de Datos, y los del cliente, el que lo usa. El proveedor sólo tiene
que preocuparse por que la implementación sea correcta —de acuerdo
con la especificación del TAD—y no por el cómo va a ser usado.

Por otro lado, el cliente {\em asume} que la implementación del
TAD es correcta y no se preocupa por los detalles. Cuando usted utiliza
los tipos primitivos de Python, se está dando el lujo de pensar como
cliente exclusivamente.

Por supuesto, cuando se implementa un TAD, también hay que escribir
algún código cliente que permita chequear su funcionamiento. En ese
caso, usted asume los dos roles y la labor puede ser un tanto confusa.
Hay que concentrarse para llevar la pista del rol que se está jugando
en un momento determinado.

\section{Glosario}

\index{TAD} \index{cliente} \index{proveedor} \index{infija} \index{postfija}
\index{análisis sintáctico} \index{lexema} \index{delimitador}
\begin{description}
\item [{Tipo Abstracto de Datos (TAD):}] Un tipo de datos (casi siempre
es una colección de objetos) que se define por medio de un conjunto
de operaciones y que puede ser implementado de diferentes formas.
\item [{Interfaz:}] conjunto de operaciones que define al TAD.
\item [{Implementación:}] código que satisface los requerimientos sintácticos
y semánticos de una interfaz de un TAD.
\item [{Cliente:}] un programa (o la persona que lo escribió) que usa un
TAD.
\item [{Proveedor:}] el código (o la persona que lo escribió) que implementa
un TAD.
\item [{Barniz:}] una definición de clase que implementa un TAD con métodos
que son llamados a otros métodos, a menudo realizando unas transformaciones
previas. El barniz no realiza un trabajo significativo, pero sí mejora
o estandariza las interfaces a las que accede el cliente.
\item [{Estructura de datos genérica:}] estructura de datos que puede
contener objetos de todo tipo.
\item [{Infija:}] una forma de escribir expresiones matemáticas con los
operadores entre los operandos.
\item [{Postfija:}] una forma de escribir expresiones matemáticas con los
operadores después de los operandos.
\item [{Análisis sintáctico:}] leer una cadena de caracteres o lexemas
y analizar su estructura gramatical.
\item [{Lexema:}] conjunto de caracteres que se considera como una unidad
para los propósitos del análisis sintáctico, tal como las palabras
del lenguaje natural.
\item [{Delimitador:}] caracter que se usa para separar lexemas, tal como
los signos de puntuación en el lenguaje natural.
\end{description}

