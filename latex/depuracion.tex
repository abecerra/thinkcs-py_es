
\chapter{Depuración }

\index{depuración}

Hay diferentes tipos de error que pueden suceder en un programa y
es muy útil distinguirlos a fin de rastrearlos más rápidamente:
\begin{itemize}
\item Los errores sintácticos se producen cuando Python traduce el código
fuente en código objeto. Usualmente indican que hay algún problema
en la sintaxis del programa. Por ejemplo, omitir los puntos seguidos
al final de una sentencia \texttt{def} produce un mensaje de error
un poco redundante \texttt{SyntaxError: invalid syntax}.
\item Los errores en tiempo de ejecución se producen por el sistema de ejecución,
si algo va mal mientras el programa corre o se ejecuta. La mayoría
de errores en tiempo de ejecución incluyen información sobre la localización
del error y las funciones que se estaban ejecutando. Ejemplo: una
recursión infinita eventualmente causa un error en tiempo de ejecución
de ``maximum recursion depth exceeded.''
\item Los errores semánticos se dan en programas que compilan y se ejecutan
normalmente, pero no hacen lo que se pretendía. Ejemplo: una expresión
podría evaluarse en un orden inesperado, produciendo un resultado
incorrecto.
\end{itemize}
\index{error en tiempo de compilación} \index{error sintáctico}
\index{error en tiempo de ejecución} \index{error semántico} \index{error!en tiempo de compilación}
\index{error!sintaxis} \index{error!tiempo de ejecución} \index{error!semántica}
\index{excepción}

El primer paso en la depuración consiste en determinar la clase de
error con la que se está tratando. Aunque las siguientes secciones
se organizan por tipo de error, algunas técnicas se aplican en más
de una situación.

\section{Errores sintácticos}

\index{mensajes de error} \index{compilador}

Los errores sintácticos se corrigen fácilmente una vez que usted ha
determinado a qué apuntan. Desafortunadamente, en algunas ocasiones
los mensajes de error no son de mucha ayuda. Los mensajes de error
más comunes son \texttt{SyntaxError: invalid syntax} y \texttt{SyntaxError: invalid
token}, que no son muy informativos.

Por otro lado, el mensaje sí dice dónde ocurre el problema en el programa.
Más precisamente, dice dónde fue que Python encontró un problema,
que no necesariamente es el lugar dónde está el error. Algunas veces
el error está antes de la localización que da el mensaje de error,
a menudo en la línea anterior.

\index{desarrollo incremental de programas}

Si usted está construyendo los programas incrementalmente, debería
tener una buena idea de dónde se localiza el error. Estará en la última
línea que se agregó.

Si usted está copiando código desde un libro, comience por comparar
su código y el del libro muy cuidadosamente. Chequee cada carácter.
Al mismo tiempo, recuerde que el libro puede tener errores, así que
si encuentra algo que parece un error sintáctico, entonces debe serlo.

Aquí hay algunas formas de evitar los errores sintácticos más comunes:

\index{syntax}
\begin{enumerate}
\item Asegúrese de no usar una palabra reservada de Python como nombre de
variable
\item Chequee que haya colocado dos puntos seguidos al final de la cabecera
de cada sentencia compuesta, incluyendo los ciclos \texttt{for}, \texttt{while},
los condicionales \texttt{if}, las definiciones de función \texttt{def}
y las \texttt{clases}.
\item Chequee que la indentación o sangrado sea consistente. Se puede indentar
con espacios o tabuladores, pero es mejor no mezclarlos. Cada nivel
debe sangrarse la misma cantidad de espacios o tabuladores.
\item Asegúrese de que las cadenas en los programas estén encerradas entre
comillas.
\item Si usted tiene cadenas multilínea creadas con tres comillas consecutivas,
verifique su terminación. Una cadena no terminada puede causar un
error denominado \texttt{invalid token} al final de su programa, o
puede tratar la siguiente parte del programa como si fuera una cadena,
hasta toparse con la siguiente cadena. En el segundo caso, ¡puede
que Python no produzca ningún mensaje de error!
\item Un paréntesis sin cerrar—\verb+(+, \verb+{+, o \verb+[+—hace que
Python continúe con la siguiente línea como si fuera parte de la sentencia
actual. Generalmente, esto causa un error inmediato en la siguiente
línea.
\item Busque por la confusión clásica entre \texttt{=} y \texttt{==}, adentro
y afuera de los condicionales.
\end{enumerate}
Si nada de esto funciona, avance a la siguiente sección...

\subsection{No puedo ejecutar mi programa sin importar lo que haga}

Si el compilador dice que hay un error y usted no lo ha visto, eso
puede darse porque usted y el compilador no están observando el mismo
código. Chequee su ambiente de programación para asegurarse de que
el programa que está editando es el que Python está tratando de ejecutar.
Si no está seguro, intente introducir deliberadamente un error sintáctico
obvio al principio del programa. Ahora ejecútelo o impórtelo de nuevo.
Si el compilador no encuentra el nuevo error, probablemente hay algún
problema de configuración de su ambiente de programación.

Si esto pasa, una posible salida es empezar de nuevo con un programa
como ``Hola todo el mundo!,'' y asegurarse de que pueda ejecutarlo
correctamente. Después, añadir gradualmente a éste los trozos de su
programa.

\section{Errores en tiempo de ejecución}

Cuando su programa está bien sintácticamente Python puede importarlo
y empezar a ejecutarlo. ¿Qué podría ir mal ahora?

\subsection{Mi programa no hace absolutamente nada}

Este problema es el más común cuando su archivo comprende funciones
y clases pero no hace ningún llamado para empezar la ejecución. Esto
puede ser intencional si usted sólo planea importar este módulo para
proporcionar funciones y clases.

Si no es intencional, asegúrese de que está llamando a una función
para empezar la ejecución, o ejecute alguna desde el indicador de
entrada (prompt). Revise la sección posterior sobre el ``Flujo de
Ejecución''.

\subsection{Mi programa se detiene}

\index{ciclo infinito} \index{recursión infinita} \index{detención}

Si un programa se detiene y parece que no está haciendo nada, decimos
que está ``detenido.'' Esto a veces sucede porque está atrapado
en un ciclo infinito o en una recursión infinita.
\begin{itemize}
\item Si hay un ciclo sospechoso de ser la causa del problema, añada un
\texttt{print} inmediatamente antes del ciclo que diga ``entrando
al ciclo'' y otra inmediatamente después, que diga ``saliendo del
ciclo.''

Ejecute el programa. Si obtiene el primer mensaje y no obtiene el
segundo, ha encontrado su ciclo infinito. revise la sección posterior
``Ciclo Infinito''.
\item La mayoría de las veces, una recursión infinita causará que el programa
se ejecute por un momento y luego produzca un error ``RuntimeError:
Maximum recursion depth exceeded''. Si esto ocurre revise la sección
posterior ``Recursión Infinita''.

Si no está obteniendo este error, pero sospecha que hay un problema
con una función recursiva o método, también puede usar las técnicas
de la sección ``Recursión Infinita''.
\item Si ninguno de estos pasos funciona, revise otros ciclos y otras funciones
recursivas, o métodos.
\item Si eso no funciona entonces es posible que usted no comprenda el flujo
de ejecución que hay en su programa. Vaya a la sección posterior ``Flujo
de ejecución''.
\end{itemize}

\subsubsection{Ciclo infinito}

\index{ciclo infinito} \index{ciclo!infinito} \index{condición}
\index{ciclo!condición}

Si usted cree que tiene un ciclo infinito añada un \texttt{print}
al final de éste que imprima los valores de las variables de ciclo
(las que aparecen en la condición) y el valor de la condición.

Por ejemplo:
\begin{pythoncode}
while x > 0 and y < 0 :
  # hace algo con x
  # hace algo con y

  print( "x: ", x )
  print( "y: ", y )
  print( "condicion: ", (x > 0 and y < 0))
\end{pythoncode}
 Ahora, cuando ejecute el programa, usted verá tres líneas de salida
para cada iteración del ciclo. En la última iteración la condición
debe ser \texttt{falsa}. Si el ciclo sigue, usted podrá ver los valores
de \texttt{x} y \texttt{y}, y puede deducir por qué no se están actualizando
correctamente.

\subsubsection{Recursión infinita}

\index{recursión Infinita} \index{recursión Infinita}

La mayoría de las veces una recursión infinita causará que el programa
se ejecute durante un momento y luego producirá un error: \texttt{Maximum
recursion depth exceeded}.

Si sospecha que una función o método está causando una recursión infinita,
empiece por chequear la existencia de un caso base. En otras palabras,
debe haber una condición que haga que el programa o método retorne
sin hacer un llamado recursivo. Si no lo hay, es necesario reconsiderar
el algoritmo e identificar un caso base.

Si hay un caso base, pero el programa no parece alcanzarlo, añada
un \texttt{print} al principio de la función o método que imprima
los parámetros. Ahora, cuando ejecute el programa usted verá unas
pocas líneas de salida cada vez que la función o método es llamada,
y podrá ver los parámetros. Si no están cambiando de valor acercándose
al caso base, usted podrá deducir por qué ocurre esto.

\subsubsection{Flujo de ejecución}

\index{Flujo de Ejecución} \index{ejecución!flujo de}

Si no está seguro de cómo se mueve el flujo de ejecución a través
de su programa, añada \texttt{print} al comienzo de cada función con
un mensaje como ``entrando a la función \texttt{foo},'' donde \texttt{foo}
es el nombre de la función.

Ahora, cuando ejecute el programa, se imprimirá una traza de cada
función a medida que van siendo llamadas.

\subsection{Cuando ejecuto el programa obtengo una excepción}

\index{excepción} \index{error en tiempo de ejecución}

Si algo va mal durante la ejecución, Python imprime un mensaje que
incluye el nombre de la excepción, la línea del programa donde ocurrió,
y un trazado inverso.

\index{trazado inverso}

El trazado inverso identifica la función que estaba ejecutándose,
la función que la llamó, la función que llamó a {\em esta} última,
y así sucesivamente. En otras palabras, traza el camino de llamados
que lo llevaron al punto actual de ejecución. También incluye el número
de línea en su archivo, donde cada uno de estos llamados tuvo lugar.

El primer paso consiste en examinar en el programa el lugar donde
ocurrió el error y ver si se puede deducir qué pasó. Aquí están algunos
de los errores en tiempo de ejecución más comunes:
\begin{description}
\item [{NameError:}] usted está tratando de usar una variable que no existe
en el ambiente actual. Recuerde que las variables locales son locales.
No es posible referirse a ellas afuera de la función donde se definieron.

\index{NameError} \index{TypeError}
\item [{TypeError:}] hay varias causas:

\begin{itemize}
\item Usted está tratando de usar un valor impropiamente. Por ejemplo: indexar
una cadena, lista o tupla con un valor que no es entero.

\index{índice}
\item No hay correspondencia entre los elementos en una cadena de formato
y los elementos pasados para hacer la conversión. Esto puede pasar
porque el número de elementos no coincide o porque se está pidiendo
una conversión invalida.

\index{operador de formato} \index{operador!formato}
\item Usted está pasando el número incorrecto de argumentos a una función
o método. Para los métodos, mire la definición de métodos y chequee
que el primer parámetro sea \texttt{self}. Luego mire el llamado,
asegúrese de que se hace el llamado sobre un objeto del tipo correcto
y de pasar los otros parámetros correctamente.
\end{itemize}
\item [{KeyError:}] usted está tratando de acceder a un elemento de un
diccionario usando una llave que éste no contiene.

\index{KeyError} \index{diccionario}
\item [{AttributeError:}] está tratando de acceder a un atributo o método
que no existe.

\index{AttributeError}
\item [{IndexError:}] el índice que está usando para acceder a una lista,
cadena o tupla es más grande que su longitud menos uno. Inmediatamente,
antes de la línea del error, agregue un \texttt{print} para desplegar
el valor del índice y la longitud de la secuencia. ¿Tiene ésta el
tamaño correcto? ¿Tiene el índice el valor correcto?

\index{IndexError}
\end{description}

\subsection{Agregué tantos \texttt{prints} que estoy inundado de texto de salida}

\index{función print} \index{función!print}

Uno de los problemas de \texttt{print} para depurar es que uno puede
terminar inundado de salida. Hay dos formas de proceder: simplificar
la salida o simplificar el programa.

Para simplificar la salida se pueden eliminar o comentar los \texttt{print}
que no son de ayuda, o se pueden combinar, o se puede dar formato
a la salida, de forma que quede más fácil de entender.

Para simplificar el programa hay varias cosas que se pueden hacer.
Primero, disminuya la escala del problema que el programa intenta
resolver. Por ejemplo, si usted está ordenando un arreglo, utilice
uno {\em pequeño} como entrada. Si el programa toma entrada del
usuario, pásele la entrada más simple que causa el problema.

Segundo, limpie el programa. Borre el código muerto y reorganízelo
para hacerlo lo más legible que sea posible. Por ejemplo, si usted
sospecha que el problema está en una sección de código profundamente
anidada, intente reescribir esa parte con una estructura más sencilla.
Si sospecha de una función grande, trate de partirla en funciones
mas pequeñas y pruébelas separadamente.

Este proceso de encontrar el caso mínimo de prueba que activa el problema
a menudo permite encontrar el error. Si usted encuentra que el programa
funciona en una situación, pero no en otras, esto le da una pista
de lo que está sucediendo.

Similarmente, reescribir un trozo de código puede ayudar a encontrar
errores muy sutiles. Si usted hace un cambio que no debería alterar
el comportamiento del programa, y sí lo hace, esto es una señal de
alerta.

\section{Errores semánticos}

\index{error semántico} \index{error!semántico}

Estos son los mas difíciles de depurar porque ni el compilador ni
el sistema de ejecución dan información sobre lo que está fallando.
Sólo usted sabe lo que el programa debe hacer y sólo usted sabe por
qué no lo está haciendo bien.

El primer paso consiste en hacer una conexión entre el código fuente
del programa y el comportamiento que está dándose. Ahora, usted necesita
una hipótesis sobre lo que el programa está haciendo realmente. Una
de las cosas que complica el asunto es que la ejecución de programas
en un computador moderno es muy rápida.

A veces deseará desacelerar el programa hasta una velocidad humana,
y con algunos programas depuradores esto es posible. Pero el tiempo
que toma insertar unos \texttt{print} bien situados a menudo es mucho
más corto comparado con la configuración del depurador, la inserción
y eliminación de puntos de quiebre (breakpoints en inglés) y ``saltar''
por el programa al punto donde el error se da.

\subsection{Mi programa no funciona}

Usted debe hacerse estas preguntas:
\begin{itemize}
\item ¿Hay algo que el programa debería hacer, pero no hace? Encuentre la
sección de código que tiene dicha funcionalidad y asegúrese de que
se ejecuta en los momentos adecuados.
\item ¿Está pasando algo que no debería? Encuentre código en su programa
que tenga una funcionalidad y vea si ésta se ejecuta cuando no debería.
\item ¿Hay una sección de código que produce un efecto que no esperaba usted?
Asegúrese de que entiende dicha sección de código, especialmente si
tiene llamados a funciones o métodos en otros módulos. Lea la documentación
para las funciones que usted llama. Intente escribir casos de prueba
más sencillos y chequee los resultados.
\end{itemize}
Para programar, usted necesita un modelo mental de cómo trabajan los
programas. Si usted escribe un programa que no hace lo que se espera,
muy frecuentemente el problema no está en el programa, sino en su
modelo mental.

\index{modelo!mental} \index{modelo mental}

La mejor forma de corregir su modelo mental es descomponer el programa
en sus componentes (usualmente funciones y métodos) para luego probarlos
independientemente. Una vez encuentre la discrepancia entre su modelo
y la realidad, el problema puede resolverse.

Por supuesto, usted debería construir y probar componentes a medida
que desarrolla el programa. Si encuentra un problema, debería haber
una pequeña cantidad de código nuevo que puede estar incorrecto.

\subsection{He obtenido una expresión grande y peluda que no hace lo que espero}

\index{expresión!grande y peluda}

Escribir expresiones complejas está bien en tanto queden legibles,
sin embargo, puede ser difícil depurarlas. Es una buena idea separar
una expresión compleja en una serie de asignaciones a variables temporales.

Por ejemplo:
\begin{pythoncode}
self.manos[i].agregarCarta( \
  self.manos[self.encontrarVecino(i)].eliminarCarta())
 Esto puede reescribirse como:
 
vecino = self.encontrarVecino (i)
cartaEscogida = self.manos[vecino].eliminarCarta()
self.manos[i].agregarCarta(cartaEscogida)
\end{pythoncode}
 La versión explícita es más fácil de leer porque los nombres de variables
proporcionan una documentación adicional y porque se pueden chequear
los tipos de los valores intermedios desplegándolos.

\index{variable temporal} \index{variable!temporal} \index{orden de evaluación}
\index{precedencia}

Otro problema que ocurre con las expresiones grandes es que el orden
de evaluación puede no ser el que usted espera. Por ejemplo, si usted
está traduciendo la expresión $\frac{x}{2\pi}$ a Python, podría escribir:

\begin{pythoncode}
y = x / 2 * math.pi;
\end{pythoncode}
 Esto es incorrecto, porque la multiplicación y la división tienen
la misma precedencia y se evalúan de izquierda a derecha. Así que
ese código calcula $x\pi/2$.

Una buena forma de depurar expresiones es agregar paréntesis para
hacer explícito el orden de evaluación:
\begin{pythoncode}
 y = x / (2 * math.pi);
\end{pythoncode}
 Cuando no esté seguro del orden de evaluación, use paréntesis. No
sólo corregirá el programa si había un error, sino que también lo
hará mas legible para otras personas que no se sepan las reglas de
precedencia.

\subsection{Tengo una función o método que no retorna lo que debería}

\index{sentencia return} \index{sentencia!return}

Si usted tiene una sentencia \texttt{return} con una expresión compleja
no hay posibilidad de imprimir el valor del \texttt{return} antes
de retornar. Aquí también se puede usar una variable temporal. Por
ejemplo, en vez de:

\begin{pythoncode}
return self.manos[i].eliminarParejas()
\end{pythoncode}
 se podría escribir:

\begin{pythoncode}
cont = self.manos[i].eliminarParejas()
return cont
\end{pythoncode}
 Ahora usted tiene la oportunidad de desplegar el valor de \texttt{count}
antes de retornar.

\subsection{Estoy REALMENTE atascado y necesito ayuda}

Primero, intente alejarse del computador por unos minutos. Los computadores
emiten ondas que afectan al cerebro causando estos efectos:
\begin{itemize}
\item Frustración e ira.
\item Creencias supersticiosas (``el computador me odia'') y pensamiento
mágico (``el programa sólo funciona cuando me pongo la gorra al revés'').
\item Programación aleatoria (el intento de programar escribiendo cualquier
programa posible y escogiendo posteriormente el que funcione correctamente).
\end{itemize}
Si usted está sufriendo de alguno de estos síntomas, tómese un paseo.
Cuando ya esté calmado, piense en el programa. ¿Qué está haciendo?
¿Cuáles son las causas posibles de éste comportamiento? ¿Cuándo fue
la última vez que funcionaba bien, y qué hizo usted después de eso?

Algunas veces, sólo toma un poco de tiempo encontrar un error. A menudo
encontramos errores cuando estamos lejos del computador y dejamos
que la mente divague. Algunos de los mejores lugares para encontrar
errores son los trenes, duchas y la cama, justo antes de dormir.

\subsection{No, realmente necesito ayuda}

Esto sucede. Hasta los mejores programadores se atascan alguna vez.
A veces se ha trabajado tanto tiempo en un programa que ya no se puede
ver el error. Un par de ojos frescos es lo que se necesita.

Antes de acudir a alguien más, asegúrese de agotar todas las técnicas
descritas aquí. Su programa debe ser tan sencillo como sea posible
y usted debería encontrar la entrada más pequeña que causa el error.
Debería tener varios \texttt{print} en lugares apropiados (y la salida
que despliegan debe ser comprensible). Usted debe entender el problema
lo suficientemente bien como para describirlo concisamente.

Cuando acuda a alguien, asegúrese de darle la información necesaria:
\begin{itemize}
\item Si hay un mensaje de error, ¿cuál es, y a qué parte del programa se
refiere?
\item ¿Cuál fue el último cambio antes de que se presentara el error? ¿Cuáles
fueron las últimas líneas de código que escribió usted o cuál es el
nuevo caso de prueba que falla?
\item ¿Qué ha intentado hasta ahora, y qué ha aprendido sobre el programa?
\end{itemize}
Cuando encuentre el error, tómese un segundo para pensar sobre lo
que podría haber realizado para encontrarlo más rápido. La próxima
vez que le ocurra algo similar, será capaz de encontrar el error rápidamente.

Recuerde, el objetivo no sólo es hacer que el programa funcione. El
objetivo es aprender cómo hacer que los programas funcionen.
