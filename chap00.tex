% Texto original LaTeX del libro ``Aprenda a Pensar Como un
% Científico Informático''
% Copyright (c)  2001  Allen B. Downey, Jeffrey Elkner y John Dewey.
% Se da permiso para copiar, distribuir o modificar este documento
% bajo los términos de la Licencia de Documentación Libre GNU, Version
% 1.1 o cualquier version posterior publicada por la Free Software
% Foundation. Las Secciones Invariables son "Lista de Contribuyentes",
% sin los Textos de Portada y sin los Textos de Cubierta Posterior. Una
% copia de la licencia esta incluida en la section titulada "Licencia de
% Documentación Libre GNU".
% Esta distribución incluye un archivo llamado fdl.tex que contiene el
% texto de la Licencia de Documentación Libre GNU. Si no se encuentra,
% se lo puede obtener de www.gnu.org o puede escribir a la Free Software
% Foundation, Inc., 59 Temple Place - Suite 330, Boston, MA 02111-1307, USA.

\chapter{Solución de problemas}
\index{solución de problemas}

\section{Solución de acertijos}
\index{acertijo}

Todos nos hemos topado con acertijos como el siguiente. Disponga 
los dígitos del 1 al 9 en el recuadro siguiente, de manera que la suma de cada fila, cada columna y las dos diagonales dé el mismo resultado:
\begin{center}
\begin{tabular}{|c|c|c|}
\hline 
 &  & \tabularnewline
\hline 
 &  & \tabularnewline
\hline 
 &  & \tabularnewline
\hline
\end{tabular}
\end{center}

Este acertijo  se denomina construcción de un { \bf cuadrado
mágico}. Un acertijo, normalmente, es un enigma o adivinanza que se propone como pasatiempo. Otros ejemplos de acertijo son un crucigrama, una sopa de letras y un sudoku.

Los acertijos pueden tener varias soluciones, por ejemplo, la siguiente es una solución propuesta al acertijo anterior:

\begin{center}
\begin{tabular}{|c|c|c|}
\hline 
1 & 2 & 3\tabularnewline
\hline
4 & 5 & 6\tabularnewline
\hline 
7 & 8 & 9 \tabularnewline
\hline
\end{tabular}
\end{center}

Usted puede notar que esta solución candidata no es correcta. Si tomamos la suma por filas, obtenemos valores distintos:

\begin{itemize}

\item En la fila 1: 1+2+3=6
\item En la fila 2: 4+5+6=15
\item En la fila 3: 7+8+9=24

\end{itemize}

Si tomamos las columnas, tampoco obtenemos el mismo resultado en 
cada suma:


\begin{itemize}

\item En la columna 1: 1+4+7=12
\item En la columna 2: 2+5+8=15
\item En la columna 3: 3+6+9=18

\end{itemize}

A pesar de que las diagonales sí suman lo mismo:

\begin{itemize}

\item En la diagonal 1: 1+5+9=15
\item En la diagonal 2: 7+5+3=15

\end{itemize}

Tómese un par de minutos para resolver este acertijo, es decir, para construir un cuadrado mágico y regrese a la lectura cuando
obtenga la solución. 

Ahora, responda para sí mismo las siguientes preguntas:

\begin{itemize}

\item ¿Cuál es la solución que encontró?
\item ¿Es correcta su solución?
\item ¿Cómo le demuestra a alguien que su solución es
correcta?
\item ¿Cuál fue el proceso de solución que llevo a cabo en su 
mente?
\item ¿Cómo le explicaría a alguien el proceso de solución que
llevó a cabo?
\item ¿Puede poner por escrito el proceso de solución que llevó
a cabo?
\end{itemize}

El reflexionar seriamente sobre estas preguntas es muy importante,
tenga la seguridad de que esta actividad será muy importante para 
continuar con la lectura.

Vamos a ir contestando las preguntas desde una solución particular,
y desde un proceso de solución particular, el de los autores. Su
solución y su proceso de solución son igualmente valiosos, el 
nuestro solo es otra alternativa; es más, puede que hayamos descubierto la misma:

\begin{center}
\begin{tabular}{|c|c|c|}
\hline 
4 & 9 & 2\tabularnewline
\hline
3 & 5 & 7\tabularnewline
\hline 
8 & 1 & 6 \tabularnewline
\hline
\end{tabular}
\end{center}


Esta solución es correcta, porque la suma por filas, columnas y de
 las dos diagonales da el mismo valor, 15. Ahora, para demostrarle
a alguien este hecho podemos revisar las sumas por filas, columnas y 
diagonales detalladamente.

% \begin{itemize}
% 
% \item En la fila 1: 4+9+2=15
% \item En la fila 2: 3+5+7=15
% \item En la fila 3: 8+1+6=15
% \item En la columna 1: 4+3+8=15
% \item En la columna 2: 9+5+1=15
% \item En la columna 3: 2+7+6=15
% \item En la diagonal 1: 4+5+6=15
% \item En la diagonal 2: 8+5+2=15
% \end{itemize}

El proceso de solución que llevamos a cabo fue el siguiente:

\begin{itemize}

\item Sospechábamos que el 5 debía estar en la casilla central, ya que
es el número medio de los 9: 1 2 3 4 {\bf 5} 6 7 8 9. 

\item Observamos un patrón interesante de la primera solución propuesta: las diagonales sumaban igual, 15: 

\begin{center}
\begin{tabular}{|c|c|c|}
\hline 
{\bf 1} & 2 & {\bf 3}\tabularnewline
\hline
4 & {\bf 5} & 6\tabularnewline
\hline 
{\bf 7} & 8 & {\bf 9} \tabularnewline
\hline
\end{tabular}
\end{center}

\item  La observación anterior, 1+5+9=7+5+3, también permite deducir
otro hecho interesante. Como el 5 está en las dos sumas, podemos 
deducir que 1+9=7+3, y esto es 10. 

\item Una posible estrategia consiste en colocar parejas de números
que sumen 10, dejando al 5 ``emparedado'', por ejemplo, poner la 
pareja 6,4:

\begin{center}
\begin{tabular}{|c|c|c|}
\hline 
  & {\bf 6} & \tabularnewline
\hline
  & {\bf 5} & \tabularnewline
\hline 
  & {\bf 4} &  \tabularnewline
\hline
\end{tabular}
\end{center}

\item Para agilizar ésta estrategia es conveniente enumerar todas
las parejas de números entre 1 y 9 que suman 10, excluyendo al 5:
(1,9),(2,8),(3,7),(4,6)

\item Ahora, podemos probar colocando éstas parejas en filas, columnas
y diagonales. 

\item Un primer ensayo: 

\begin{center}
\begin{tabular}{|c|c|c|}
\hline 
 7 & {\bf 6} & 2\tabularnewline
\hline
  & {\bf 5} & \tabularnewline
\hline 
 8 & {\bf 4} & 3 \tabularnewline
\hline
\end{tabular}
\end{center}

Aquí encontramos que es imposible armar el cuadrado, pues no hay como
situar el 9 ni el 1. Esto sugiere que la pareja (6,4) no va en la 
columna central, debemos cambiarla.

\item Después de varios ensayos, moviendo la pareja (6,4), y 
reacomodando los otros números logramos llegar a la solución 
correcta.

\end{itemize}

\section{El método de solución}
\label{sec:metodosolucion}
\index{método de solución de problemas}

Mas allá de la construcción de cuadrados mágicos, lo que el ejemplo
anterior pretende ilustrar es la importancia que tiene el usar un 
método ordenado de solución de problemas, acertijos en este caso.

Observe que la solución anterior fue conseguida a través de varios
pasos sencillos, aunque el encadenamiento de todos éstos hasta 
producir un resultado correcto pueda parecer algo complejo.

Lea cualquier historia de Sherlock Holmes y obtendrá la misma
impresión. Cada caso, tomado por el famoso detective, presenta un
enigma difícil, que al final es resuelto por un encadenamiento
de averiguaciones, deducciones, conjeturas y pruebas sencillas. Aunque
la solución completa de cada caso demanda un proceso complejo de
razonamiento, cada paso intermedio es sencillo y está guiado por
preguntas sencillas y puntuales.

Las aventuras de Sherlock Holmes quizás constituyen una de las 
mejores referencias bibliográficas en el proceso de solución de problemas. 
Son como los capítulos de C.S.I\footnote{http://www.cbs.com/primetime/csi/}, 
puestos por escrito. Para Holmes, y quizás para Grissom, existen varios 
principios que se deben seguir:

\begin{itemize}


\item Obtenerr las soluciones y quedarse satisfecho con ellas. El peor 
inconveniente para consolidar un método de solución de problemas 
poderoso consiste en que cuando nos topemos con una solución completa de 
un solo golpe nos quedemos sin reflexionar cómo llegamos a ella. Así, nunca
aprenderemos estrategias valiosas para resolver el siguiente problema. 
Holmes decía que nunca adivinaba, primero
recolectaba la mayor información posible sobre los problemas
que tenía a mano, de forma que estos datos le ``sugirieran'' alguna
solución.

\item Todo empieza por la observación, un proceso minucioso de recolección
de datos sobre el problema a mano. Ningún dato puede ignorarse de entrada,
hasta que uno tenga una comprensión profunda sobre el problema.

\item Hay que prepararse adecuadamente en una variedad de dominios. Para Holmes
esto iba desde tiro con pistola, esgrima, boxeo, análisis de huellas, entre otros.
Para cada problema que intentemos resolver habrá un dominio en el cual debemos
aprender lo más que podamos, esto facilitará enormemente el proceso de solución.


\item Prestar atención a los detalles ``inusuales''. Por ejemplo, en nuestro cuadrado  mágico
 fue algo inusual que las diagonales de la primera alternativa de 
solución, la más ingenua posible, sumaran lo mismo. 

\item Para  deducir más hechos, a partir de los datos recolectados y los detalles
inusuales, hay que usar todas las armas del razonamiento: el poder de la deducción 
(derivar nuevos hechos a partir de los datos), la inducción (generalizar a partir de casos), 
la refutación (el proceso de probar la falsedad de alguna conjetura), el pensamiento 
analógico (encontrando relaciones,metáforas, analogías) y, por último, pero no 
menos importante, el uso del sentido común.

 \item Después del análisis de datos, uno siempre debe proponer una alternativa de solución, 
así sea simple e ingenua, y proceder intentando probarla y {\bf refutarla}
al mismo tiempo. Vale la pena recalcar esto: no importa que tan ingenua, sencilla
e incompleta es una alternativa de solución, con tal de que nos permita seguir
indagando. Esto es como la primera frase que se le dice a una chica (o chico,
dado el caso) que uno quiere conocer; no importa qué frase sea, no importa que
tan trivial sea, con tal de que permita iniciar una conversación.

\item El proceso de búsqueda de soluciones es como una conversación que se inicia,
aunque en este caso el interlocutor no es una persona, sino el problema que tenemos a mano. 
Con una primera alternativa de solución---no importa lo sencilla e incompleta--- podemos 
formularnos una pregunta interesante: ¿resuelve esta alternativa el problema?

\item Lo importante de responder la pregunta anterior no es la obtención de un {\bf No} 
como respuesta; pues esto es lo que sucede la mayoría de las veces. Lo importante
viene cuando nos formulamos esta segunda pregunta ¿Con lo que sabemos del 
problema hasta ahora {\bf por qué mi alternativa no es capaz de resolverlo?} 

\item La respuesta a la pregunta anterior puede ser: todavía no sé lo suficiente 
sobre el problema para entender por qué mi alternativa de solución no lo resuelve;
esto es una señal de alerta para recolectar más datos y estudiar mas el dominio
del problema.

\item Una respuesta más constructiva a la pregunta anterior puede ser: mi alternativa
de solución no resuelve el problema porque no considera algunos hechos importantes,
y no considera algunas restricciones que debe cumplir una solución. Un ejemplo de
este tipo de respuesta lo da nuestro ensayo de colocar la pareja (6,4) emparedando
al 5 en el problema del cuadrado mágico:

\begin{center}
\begin{tabular}{|c|c|c|}
\hline 
 7 & {\bf 6} & 2\tabularnewline
\hline
  & {\bf 5} & \tabularnewline
\hline 
 8 & {\bf 4} & 3 \tabularnewline
\hline
\end{tabular}
\end{center}


Cuando notamos que la pareja (6,4) no puede colocarse en la columna central del 
cuadrado, intentamos otra alternativa de solución, colocando estos
números en filas o en las esquinas. Lo importante de este tipo de respuesta
es que nos va a permitir avanzar a {\em otra} alternativa de solución, casi
siempre más compleja y más cercana a la solución.


\item No poner obstáculos a la creatividad. Es muy difícil lograr esto porque la
mente humana siempre busca límites para respetar; así, hay que realizar un 
esfuerzo consciente para eliminar todo límite o restricción que nuestra mente 
va creando. Una estrategia interesante es el uso y fortalecimiento del 
pensamiento lateral.


\item Perseverar. El motivo más común de fracaso en la solución de acertijos y 
problemas es el abandono. No existe una receta mágica para resolver problemas, 
lo único que uno puede hacer es seguir un método y perseverar, perseverar
sin importar cuantas alternativas de solución incorrectas se hayan generado.
Esta es la clave para el éxito. Holmes decía que eran muchísimos más los casos
que no había podido resolver, quizás Grissom reconocería lo mismo. Lo importante
entonces es perseverar ante cada nuevo caso, esta es la única actitud razonable
para enfrentar y resolver problemas.

\end{itemize}

\section{Reflexión sobre este método de solución}

El proceso de solución de problemas que hemos presentado es muy sencillo. Los grandes
profesionales que tienen mucho éxito en su campo (científicos, humanistas, 
ingenieros, médicos, empresarios, etc.) tienen métodos de solución de problemas mucho
más avanzados que el que hemos presentado aquí. Con esto pretendemos ilustrar un 
punto: lo importante no es el método propuesto, porque muy probablemente no le
servirá para resolver todos los problemas o acertijos que enfrente; lo importante
es:

\begin{itemize}

\item Contar con un método de solución de problemas. Si no hay método, podemos
 solucionar problemas, claro está, pero cada vez que lo hagamos será por golpes
de suerte o inspiración --que no siempre nos acompaña, desafortunadamente. 

\item Desarrollar, a medida que transcurra el tiempo y adquiera más conocimientos 
sobre su profesión y la vida, un método propio de solución de problemas. Este
 desarrollo personal puede tomar como base el método expuesto aquí, o algún otro
que encuentre en la literatura o a través de la interacción con otras 
personas.

\end{itemize}


\section{Acertijos propuestos}

Para que empiece inmediatamente a construir su método personal de solución de
problemas, tome cada uno de los siguientes acertijos, siga el proceso de solución 
recomendado y documéntelo a manera de entrenamiento. Si usted descubre estrategias 
generales que no han sido consideradas aquí, compártalas con sus
compañeros, profesores y, mejor aún, con los autores del libro.


\begin{enumerate}

\item Considere un tablero de ajedrez de 4x4 y 4 damas del mismo color. Su misión
es colocar las 4 damas en el tablero sin que éstas se ataquen entre sí. Recuerde
que una dama ataca a otra ficha si se encuentran en la misma fila, columna o 
diagonal.

\item Partiendo de la igualdad $a=b$, encuentre cuál es el problema en el siguiente
razonamiento:

\begin{eqnarray*}
a & = & b\\
a^{2} & = & ba\\
a^{2}-b^{2} & = & ba-b^{2}\\
(a-b)(a+b) & = & b(a-b)\\
a+b & = & b\\
a & = & 2b\\
\frac{a}{b} & = & 2\\
1 & = & 2
\end{eqnarray*}

Tenga en cuenta que en el último paso se volvió a usar el hecho de que $a=b$, en la
forma $\frac{a}{b}=1$.

\item Encuentre el menor número entero positivo que pueda descomponerse como la
suma de los cubos de dos números enteros positivos de dos maneras distintas. Esto es,
encontrar el mínimo $A$ tal que $A=b^3+c^3$ y $A=d^3+e^3$, con $A,b,c,d$ y $e$ 
números positivos, mayores a cero, y distintos entre si.

\end{enumerate}


\section{Mas allá de los acertijos: problemas computacionales}
\index{problema computacional}

Un problema computacional es parecido a un acertijo; se presenta una situación
problemática y uno debe diseñar alguna solución. En los problemas computacionales
la solución consiste en una {\bf descripción general de procesos}; esto es, 
un problema computacional tiene como solución la descripción de un conjunto de 
pasos que se podrían llevar a cabo de manera general para lograr un objetivo.

Un ejemplo que ilustra esto es la multiplicación. Todos sabemos multiplicar números
de dos cifras, por ejemplo:

%\newpage
 \[
 \begin{array}{cccc}
           \ &\ &3&4\\
           \ & \times&2&1 \\ \hline
           \ &\ &3&4 \\
            +&6&8&\ \\  \hline
           \ &7&1&4 \\
 \end{array}
\]

Pero el problema computacional asociado a la multiplicación de números de dos
cifras consiste en hallar la descripción general de todos los procesos posibles
de multiplicación de parejas de números de dos cifras. Este problema ha sido
resuelto, desde hace varios milenios, por diferentes civilizaciones humanas,
siguiendo métodos alternativos. Un método de solución moderno podría describirse
así:

Tome los dos números de dos cifras, P y Q. Suponga que las cifras de P son $p_1$ y $p_2$,
esto es, $P=p_1p_2$. Igualmente, suponga que $Q=q_1q_2$. La descripción 
{\bf general} de todas las multiplicaciones de dos cifras puede hacerse así:

\[
  \begin{array}{cccc}
           \ &\ &p_1&p_2\\
           \ & \times&q_1&q_2 \\ \hline
           \ &\ &q_2p_1&q_2p_2 \\
            +&q_1p_1&q_1p_2&\ \\  \hline
           \ &q_1p_1&q_2p_1+q_1p_2&q_2p_2 \\
  \end{array}
\]

\begin{itemize}

\item Tome la cifra $q_2$ y multiplíquela por las cifras de  $P$ (con ayuda de una tabla de multiplicación). 
Ubique los resultados debajo de cada cifra de $P$ correspondiente. 

\item Tome la cifra $q_1$ y multiplíquela por las cifras de  $P$ (con ayuda de una tabla de multiplicación). 
Ubique los resultados debajo de las cifras que se generaron en el paso anterior, aunque desplazadas una columna hacia la izquierda.

\item Si en alguno de los pasos anteriores el resultado llega a 10 o se pasa de 10, ubique las unidades únicamente 
y lleve un acarreo, en decenas o centenas) para la columna de la izquierda.

\item Sume los dos resultados parciales, obteniendo el resultado final.

\end{itemize}

Usted puede estar quejándose en este momento, ¿para qué hay que complicar tanto
nuestro viejo y conocido proceso de multiplicación? Bueno, hay varias razones
para  esto:

\begin{itemize}

\item Una descripción impersonal como ésta puede ser leída y ejecutada por cualquier
persona ---o computador, como veremos mas adelante---.

\item Sólo creando descripciones generales de procesos se pueden analizar para
demostrar que funcionan correctamente.

\item Queríamos sorprenderlo, tomando algo tan conocido como la suma y dándole
una presentación que, quizás, nunca había visto. Este cambio de perspectiva
es una invitación a que abra su mente a pensar en descripciones generales 
de procesos.
\end{itemize}

Precisando un poco, un problema computacional es la descripción general de una
situación en la que se presentan unos datos de entrada y una salida deseada
que se quiere calcular. Por ejemplo, en el problema computacional de la 
multiplicación de números de dos cifras, los datos de entrada son los números 
a multiplicar; la salida es el producto de los dos números. Existen más 
problemas computacionales como el de ordenar un conjunto de números y el 
problema de encontrar una palabra en un párrafo de texto. Como ejercicio 
defina para estos problemas cuales son los datos de entrada 
y la salida deseada.

La solución de un problema computacional es una descripción general del  
conjunto de pasos que se deben llevar a cabo con las entradas del problema para producir los
datos de salida deseados.  Solucionar problemas computacionales no es muy diferente de 
solucionar acertijos, las dos actividades producen la misma clase de retos
intelectuales, y el método de solución de la sección \ref{sec:metodosolucion} es aplicable en los dos casos.
Lo único que hay que tener en cuenta es que la solución de un problema es una {\bf descripción 
general o programa, como veremos más adelante}, que se refiere a las entradas
y salidas de una manera más técnica de lo que estamos acostumbrados. Un ejemplo
de esto lo constituyen los nombres $p_1,p_2,q_1$ y $q_2$ que usamos en la 
descripción general de la multiplicación de números de dos cifras. 

Aunque la solución de problemas es una actividad compleja, es muy interesante,
estimulante e intelectualmente gratificante; incluso cuando no llegamos a solucionar
los problemas completamente. En el libro vamos a enfocarnos
en la solución de problemas computacionales por medio de programas, y, aunque
solo vamos a explorar este tipo de problemas, usted verá que las estrategias
de solución, los conceptos que aprenderá y la actitud de científico de la
computación que adquirirá serán valiosas herramientas para resolver todo tipo
de problemas de la vida real.


\section{Glosario}

\begin{description}

\item[Acertijo:] enigma o adivinanza que se propone como pasatiempo.

\item[Solución de problemas:]  el proceso de formular un problema,
hallar la solución y expresar la solución.

\item[Método de solución de problemas:] un conjunto de pasos, estrategias y técnicas
organizados que permiten solucionar problemas de una manera ordenada.

\item[Problema:] una situación o circunstancia en la que se dificulta lograr un
fin. 

\item[Problema computacional:] una situación general con una especificación de los 
datos de entrada y los datos de  salida deseados.

\item[Solución a un problema:] conjunto de pasos y estrategias que permiten lograr
un fin determinado en una situación problemática, cumpliendo ciertas restricciones.

\item[Solución a un problema computacional:] descripción general de los pasos
que toman cualquier entrada en un problema computacional y la transforman
en una salida deseada.

\item[Restricción:] una condición que tiene que cumplirse en un problema dado.

\index{acertijo}
\index{solución de problemas}
\index{método de solución de problemas}
\index{método!de solución de problemas}
\index{problema}
\index{solución a un problema}
\index{problema!solución}
\index{restricción}
\end{description}

\section{Ejercicios}

Intente resolver los siguientes problemas computacionales, proponiendo soluciones
{\bf generales e impersonales}:

\begin{enumerate}
\item Describa cómo ordenar tres números a, b y c.
\item Describa cómo encontrar el menor elemento en un conjunto de números.
\item Describa cómo encontrar una palabra dentro de un texto más largo.
\end{enumerate}
