% LaTeX source for textbook ``How to think like a computer scientist''
% Copyright (c)  2001  Allen B. Downey, Jeffrey Elkner, and Chris Meyers.

% Permission is granted to copy, distribute and/or modify this
% document under the terms of the GNU Free Documentation License,
% Version 1.1  or any later version published by the Free Software
% Foundation; with the Invariant Sections being "Contributor List",
% with no Front-Cover Texts, and with no Back-Cover Texts. A copy of
% the license is included in the section entitled "GNU Free
% Documentation License".

% This distribution includes a file named fdl.tex that contains the text
% of the GNU Free Documentation License.  If it is missing, you can obtain
% it from www.gnu.org or by writing to the Free Software Foundation,
% Inc., 59 Temple Place - Suite 330, Boston, MA 02111-1307, USA.

\chapter{Interludio 2: Triqui con interfaz gráfica}
\index{Triqui}


\section{Motivación}
En el capítulo \ref{cap:inter1:triqui} desarrollamos un juego de triqui completo
para integrar varios conceptos de programación. Ahora, para integrar varios
conceptos de programación con objetos haremos lo mismo, construiremos un triqui
con interfaz gráfica de usuario

El programa utiliza la biblioteca Kivy, que permite realizar interfaces
gráficas que corran en celulares, tablets (con pantallas sensibles al tacto) 
y computadores tradicionales (corriendo MacOS X, Windows y Linux). Hay que 
conseguir el instalador de http://kivy.org.

\section{Ganar y empatar}
Para empezar tomamos el código final del triqui desarrollado en el capítulo \ref{cap:inter1:triqui}
y lo convertimos en un módulo que nos permite verificar si alguien gana el juego o
si hay un empate. El proceso es sencillo: eliminamos todo el código que no tenga que ver con 
verificar quien gana o si hay empate, tomando triqui9.py y para transformarlo en validar.py, un 
módulo con 9 funciones: crear,  ganaDiagonal1,  ganaDiagonal2, ganaFila, ganaHorizontal, ganaColumna,
ganaVertical, gana y empate. Mas adelante lo utilizaremos, reutilizando el código.

\index{reutilización}

\section{Programación orientada a eventos}
\index{Programación orientada a eventos}

En muchas bibliotecas gráficas como kivy existe un gran ciclo que procesa
los eventos generados a través del teclado, apuntador o la pantalla
sensible al tacto. Este ciclo es infinito, pero se le hace un break
cuando se cierra la ventana de la aplicación. En la programación orientada
a eventos debemos acostumbrarnos a varias cosas.

Primero, el flujo del programa está determinado por lo que haga el usuario con
los elementos gráficos de la pantalla. Kivy procesará cada evento (click,
tecla digitada) de manera predeterminada, por ejemplo cerrar la
ventana hará un break en el gran ciclo.

\index{evento}

Segundo, las bibliotecas como Kivy se encargan de redibujar automáticamente 
partes de la ventana cuando otra ventana se pase por encima, o cuando se cambie de tamaño.

Nuestro primer programa importa lo que necesita de kivy:

\beforeverb
\begin{verbatim}
# triqui0.py
import kivy
kivy.require('1.8.0')
from kivy.app import App
from kivy.uix.gridlayout import GridLayout

class Triqui(GridLayout):
    def __init__(self, **kwargs):
        super(Triqui, self).__init__(**kwargs)
   

class Programa(App):
    def build(self):
        self.title = 'Triqui'
        return Triqui()

if __name__ == '__main__':
    Programa().run()
\end{verbatim}
\afterverb

El gran ciclo reside en la clase App de Kivy, de la que heredamos la clase
Programa. Cuando arranca el programa, al ejecutar run() se corre este
gran ciclo que procesa eventos.

El método build de programa retorna un objeto Triqui, que es nuestra ventana.
La clase Triqui hereda de GridLayout, una ventana que contiene elementos
en disposición de cuadrícula (como una matriz). El método init de Triqui
llama al médodo init de la clase madre y le pasa una lista con un número
de argumentos variable (**kwargs). 

Por ahora nuestra ventana, instancia de la clase Triqui, está vacía.

\section{Widgets}
Las ventanas suelen tener muchos elementos gráficos como menús, botones, paneles
entre otros. En bibliotecas como kivy se llaman widgets. Por 
ejemplo, un botón es un tipo de widget que  se define en la clase Button.

Como el flujo de los programas gráficos no está determinado por el programador,
sino por el usuario al interactuar con los widgets de la ventana, el mecanismo
que se utiliza para reaccionar es el de registrar métodos que atiendan a los eventos.

\beforeverb
\begin{verbatim}
# triqui1.py
class Triqui(GridLayout):
    def __init__(self, **kwargs):
        super(Triqui, self).__init__(**kwargs)
        self.add_widget(Button(font_size=100, 
                        on_press=self.boton_presionado))
    
    def boton_presionado(self, w):
        pass

\end{verbatim}
\afterverb

Los widgets se agregan a una ventana mediante el método add\_widget. Aquí agregamos
un botón y registramos un método que responde al evento de presionarlo. Por ahora 
el método no hace nada.

\section{El Triqui}

A continuación, definimos la geometría de la ventana como una matriz de 3 filas 
y 3 columnas en la que cada elemento es un botón. Ahora, en el método boton\_presionado
vamos a mostrar un cuadro de diálogo que muestra un texto sencillo.

\beforeverb
\begin{verbatim}
# triqui2.py
class Triqui(GridLayout):
    def __init__(self, **kwargs):
        super(Triqui, self).__init__(**kwargs)
        self.cols = 3
        self.rows = 3
        for i in range(3):
            for j in range(3):
                self.add_widget(Button(font_size=100, 
                                on_press=self.boton_presionado))
        
    def boton_presionado(self, w):
        MostrarMensaje("Titulo","Presionaste una casilla")

\end{verbatim}
\afterverb

MostrarMensaje es una clase que heredamos de PopUp, la clase que tiene kivy para
cuadros de diálogo:

\beforeverb
\begin{verbatim}
# triqui2.py
class MostrarMensaje(Popup): 
    def __init__(self, titulo, mensaje, **kwargs):
        self.size_hint_x = self.size_hint_y = .5
        self.title = titulo
        super(MostrarMensaje, self).__init__(**kwargs)
        self.add_widget(Button(text=mensaje, 
                        on_press=lambda x:self.dismiss()))
        self.open()


\end{verbatim}
\afterverb

El cuadro de diálogo tiene un título y un botón que, al ser presionado, cierra
todo el cuadro.

\section{Jugando por turnos}

Como el flujo de ejecución depende de los usuarios, vamos a llevar pista en el 
programa de quien tiene el turno de juego con un atributo en la clase Triqui.
Hay que crear el atributo en el método de inicialización y modificarlo en 
cada jugada. El turno será 'O' para el primer jugador y 'X' para el segundo.

\beforeverb
\begin{verbatim}
# triqui3.py
class Triqui(GridLayout):
    def __init__(self, **kwargs):
        super(Triqui, self).__init__(**kwargs)
        self.cols = 3
        self.rows = 3
        for i in range(3):
            for j in range(3):
                self.add_widget(Button(font_size=100, 
                  on_press=self.boton_presionado, text=' '))
        self.turno = 'O'

    def boton_presionado(self, w):
        if w.text != ' ':
            MostrarMensaje('Error!', "Ya se ha jugado en esa casilla!")
            return
        if self.turno == 'O':
            w.text =  'O'
            self.turno = 'X'
        else:
            w.text = 'X'
            self.turno = 'O'
\end{verbatim}
\afterverb

Cuando se presiona un botón se verifica si la casilla está vacía para poder jugar en 
ella. Si no es así se cambia el texto del botón y se cambia el turno para el otro 
jugador. Observe que en el método de inicialización de Triqui al texto de todos los 
botones se le asigna un espacio.

\section{Reutilización}

Agregamos un botón que tiene propiedades para 
registrar la fila y la columna.
\beforeverb
\begin{verbatim}
# triqui4.py
class Boton(Button):
    fila = NumericProperty(0)
    columna = NumericProperty(0)
    
    def __init__(self, **kwargs):      
        super(Boton, self).__init__(**kwargs)
        self.font_size=100
        self.text=' '
\end{verbatim}
\afterverb
        
Esto nos permite pasar el estado de los botones a una matriz con 
el siguiente método de la clase Triqui:

\beforeverb
\begin{verbatim}
# triqui4.py
    def botones_a_matriz(self,tablero):
        for i in self.children:
            f = i.fila
            c = i.columna
            self.tablero[f][c]=i.text
\end{verbatim}
\afterverb

Así podremos reutilizar el módulo validar, creando la matriz
que lleva el estado del juego :

\beforeverb
\begin{verbatim}
# triqui4.py

from validar import *
    def __init__(self, **kwargs):
        super(Triqui, self).__init__(**kwargs)
        self.cols = 3
        self.rows = 3
        for i in range(3):
            for j in range(3):
                self.add_widget(Boton(on_press=self.boton_presionado,
                       fila=i,columna=j))
        self.turno = 'O'
        self.tablero = crear()
\end{verbatim}
\afterverb

Ahora estamos en condiciones de poner valores a la matriz 
cada vez que se halla realizado una jugada:

\beforeverb
\begin{verbatim}
# triqui5.py
    def boton_presionado(self, w):
        if w.text != ' ':
            MostrarMensaje('Error!', "Ya se ha jugado en esa casilla!")
            return
        else:
            if self.turno == 'O':
                w.text =  'O'                
                self.turno = 'X'
                self.botones_a_matriz()
                if gana("O",self.tablero):
                    MostrarMensaje("Fin", "Gana el jugador O")
            else:
                # Muy similar para el otro jugador!
\end{verbatim}
\afterverb

\section{Reset}

Podemos reiniciar el juego cada vez que un jugador gane, mediante la 
creación del siguiente método de Triqui:

\beforeverb
\begin{verbatim}
# triqui6.py
   def reset(self):
        for i in self.children:
            i.text = ' '
        self.turno = 'O'
\end{verbatim}
\afterverb

Ahora lo llamamos cada vez que un jugador gana, así el tablero de
juego quedará limpio para que se inicie otra partida.

\beforeverb
\begin{verbatim}
# triqui6.py
    def boton_presionado(self, w):
        # Todo lo anterior igual
        else:
                w.text = 'X'
                self.turno = 'O'
                self.botones_a_matriz()
                if gana("X",self.tablero):
                    MostrarMensaje("Fin", "Gana el jugador X")
                    self.reset()
\end{verbatim}
\afterverb

Aprovechando al métod reset, añadimos el chequeo de empates entre 
los dos jugadores:

\beforeverb
\begin{verbatim}
# triqui7.py
    def boton_presionado(self, w):
        # Todo lo anterior igual
            if empate(self.tablero):
                MostrarMensaje("Fin", "Empate!")
                self.reset()
\end{verbatim}
\afterverb

\section{Reestructurando, otra vez}

Otra vez tenemos código parecido para revisar el estado del juego
con los dos jugadores que es mejor consolidar en una función para 
mejorar la calidad del programa. Para esto definimos el método 
revisar, que:

\begin{itemize}
 \item Si el jugador actual gana, muestra el mensaje y resetea todo.
 
 \item Si hay empate, muestra el mensaje y resetea todo.
 
 \item Si nadie gana y no hay empate, pasa el turno al otro jugador.
\end{itemize}

\beforeverb
\begin{verbatim}
# triqui8.py
  def revisar(self):          
        if gana(self.turno,self.tablero):
            mensaje = "Gana el jugador "+self.turno+"."
            MostrarMensaje("Fin", mensaje)
            self.reset()
        elif empate(self.tablero):
            MostrarMensaje("Fin", "Empate!")
            self.reset()
        else:
            self.turno = self.otro()
\end{verbatim}
\afterverb

Que depende del método otro:


\beforeverb
\begin{verbatim}
# triqui8.py
    def otro(self):
        if self.turno == 'O':
            return 'X'
        else:
            return 'O'

\end{verbatim}
\afterverb

Así terminamos con un programa que tiene en total cuatro clases, con 9 métodos
distribuidos en ellas, además de las 9 funciones del módulo validar. Tiene 96
líneas de código en triqui9.py y 66 en validar, para un total de 162.

Ilustra algo que siempre pasa con los programas textuales, cuando se convierten 
a gráficos se aumenta el código substancialmente. La ventaja, aparte de la estética,
es que el Triqui con la biblioteca kivy puede ejecutarse en Linux, Windows, Mac OS X,
Android y iOS (el sistema operativo de los teléfonos iphone y de los tablets ipad).

Por ejemplo, el paquete para android del triqui puede descargarse de:



\section{Resumen}

\begin{description}

\item[triqui0.py:] crea una ventana vacía
\item[triqui1.py:] Agrega un botón a la ventana (se ve feo!)
\item[triqui2.py:] Agrega 9 botones para formar el tablero del triqui
\item[triqui3.py:] Permite jugar a los dos jugadores sin ningún chequeo
\item[triqui4.py:] Agrega una clase heredada de Button para llevar fila y columna 
\item[triqui5.py:] Cada vez que se juega se copia el estado de los botones a una matriz
\item[triqui6.py:] Se valida si los jugadores ganan el juego con el código del triqui viejo
	    y se resetea el juego.
\item[triqui7.py:] Se revisa si hay empate, si lo hay, se resetea el juego.
\item[triqui8.py:] Se mejora el código evitando la duplicación.
\end{description}


\section{Glosario}

\begin{description}

\item[Evento:] Acción  
\item[Evento:] Acción  

\index{Evento}


\end{description}
