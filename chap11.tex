% LaTeX source for textbook ``How to think like a computer scientist''
% Copyright (c)  2001  Allen B. Downey, Jeffrey Elkner, and Chris Meyers.

% Permission is granted to copy, distribute and/or modify this
% document under the terms of the GNU Free Documentation License,
% Version 1.1  or any later version published by the Free Software
% Foundation; with the Invariant Sections being "Contributor List",
% with no Front-Cover Texts, and with no Back-Cover Texts. A copy of
% the license is included in the section entitled "GNU Free
% Documentation License".

% This distribution includes a file named fdl.tex that contains the text
% of the GNU Free Documentation License.  If it is missing, you can obtain
% it from www.gnu.org or by writing to the Free Software Foundation,
% Inc., 59 Temple Place - Suite 330, Boston, MA 02111-1307, USA.

\chapter{Archivos y excepciones}
\index{archivos}

Cuando un programa se está ejecutando, sus datos están en la memoria. Cuando
un programa termina, o se apaga el computador, los datos de la memoria
desaparecen. Para almacenar los datos de forma permanente se deben poner
en un {\bf archivo}. Normalmente los archivos se guardan en un disco duro,
disquete o CD-ROM.

Cuando hay un gran número de archivos, suelen estar organizados en
{\bf directorios} (también llamados ``carpetas''). Cada archivo se
identifica con un nombre único, o una combinación de nombre de archivo
y nombre de directorio.

Leyendo y escribiendo archivos, los programas pueden intercambiar
información entre ellos y generar formatos imprimibles como PDF.

Trabajar con archivos se parece mucho a hacerlo con libros. Para usar
un libro, hay que abrirlo. Cuando uno ha terminado, hay que cerrarlo.
Mientras el libro está abierto, se puede escribir en él o leer de él.
En cualquier caso, uno sabe en qué lugar del libro se encuentra. Casi
siempre se lee un libro según su orden natural, pero  también se
puede ir saltando de página en página.

Todo esto sirve también para los archivos. Para abrir un archivo,
se especifica su nombre y se indica si se desea leer o escribir.

La apertura de un archivo crea un objeto archivo. En este ejemplo, la
variable \texttt{f} apunta al nuevo objeto archivo.

\beforeverb
\begin{verbatim}
>>> f = open("test.dat","w")
>>> print f
<open file 'test.dat', mode 'w' at fe820>
\end{verbatim}
\afterverb
%
La función open toma dos argumentos: el primero, es el nombre del archivo
y el segundo, el modo. El modo {\verb+"w"+} significa que lo estamos abriendo
para escribir.

Si no hay un archivo llamado \texttt{test.dat} se creará. Si ya hay uno, el
archivo que estamos escribiendo lo reemplazará.

Al imprimir el objeto archivo, vemos el nombre del archivo, el
modo y la localización del objeto.

Para escribir datos en el archivo invocamos al método \texttt{write}
sobre el objeto archivo:

\beforeverb
\begin{verbatim}
>>> f.write("Ya es hora")
>>> f.write("de cerrar el archivo")
\end{verbatim}
\afterverb
%
El cierre del archivo le dice al sistema que hemos terminado de
escribir y deja el archivo listo para leer:

\beforeverb
\begin{verbatim}
>>> f.close()
\end{verbatim}
\afterverb
%
Ya podemos abrir el archivo de nuevo, esta vez para lectura, y poner su
contenido en una cadena. Esta vez el argumento de modo es {\verb+"r"+}, para
lectura:

\beforeverb
\begin{verbatim}
>>> f = open("test.dat","r")
\end{verbatim}
\afterverb
%
Si intentamos abrir un archivo que no existe, recibimos un mensaje de error:

\index{error en tiempo de ejecución}

\beforeverb
\begin{verbatim}
>>> f = open("test.cat","r")
IOError: [Errno 2] No such file or directory: 'test.cat'
\end{verbatim}
\afterverb
%
Como era de esperar, el método \texttt{read} lee datos del archivo. Sin
argumentos, lee el archivo completo:

\beforeverb
\begin{verbatim}
>>> texto = f.read()
>>> print texto
Ya es horade cerrar el archivo
\end{verbatim}
\afterverb
%
No hay un espacio entre ``hora'' y ``de'' porque no escribimos un espacio
entre las cadenas. \texttt{read} también puede aceptar un argumento que le indica cuántos
carácteres leer:

\beforeverb
\begin{verbatim}
>>> f = open("test.dat","r")
>>> print f.read(7)
Ya es h
\end{verbatim}
\afterverb
%
Si no quedan suficientes carácteres en el archivo, \texttt{read} devuelve los que
haya. Cuando llegamos al final del archivo, \texttt{read} devuelve una cadena vacía:

\beforeverb
\begin{verbatim}
>>> print f.read(1000006)
orade cerrar el archivo
>>> print f.read()

>>>
\end{verbatim}
\afterverb
%
La siguiente función copia un archivo, leyendo y escribiendo los
carácteres de cincuenta en cincuenta. El primer argumento es el
nombre del archivo original; el segundo es el nombre del archivo nuevo:

\beforeverb
\begin{verbatim}
def copiaArchivo(archViejo, archNuevo):
  f1 = open(archViejo, "r")
  f2 = open(archNuevo, "w")
  while True:
    texto = f1.read(50)
    if texto == "":
      break
    f2.write(texto)
  f1.close()
  f2.close()
  return
\end{verbatim}
\afterverb
%
La sentencia \texttt{break} es nueva. Su ejecución interrumpe el
ciclo; el flujo de la ejecución pasa a la primera sentencia después
del while.

\index{sentencia break}
\index{sentencia!break}

En este ejemplo, el ciclo \texttt{while} es infinito porque la condición \texttt{True} 
siempre es verdadera. La {\em única} forma de salir del
ciclo es ejecutar \texttt{break}, lo que sucede cuando \texttt{texto}
es una cadena vacía, y esto pasa cuando llegamos al final del
archivo.



\section{Archivos de texto}
\index{archivo de texto}
\index{archivo!texto}

Un {\bf archivo de texto} contiene carácteres imprimibles
y espacios organizados en líneas separadas por carácteres de salto de línea.
Como Python está diseñado específicamente para procesar archivos de texto,
proporciona métodos que facilitan esta tarea.

Para hacer una demostración, crearemos un archivo de texto con tres líneas de texto 
separadas por saltos de línea:

\beforeverb
\begin{verbatim}
>>> f = open("test.dat","w")
>>> f.write("línea uno\nlínea dos\nlínea tres\n")
>>> f.close()
\end{verbatim}
\afterverb
%
El método \texttt{readline} lee todos los carácteres hasta, e incluyendo, el
siguiente salto de línea:

\beforeverb
\begin{verbatim}
>>> f = open("test.dat","r")
>>> print f.readline()
línea uno

>>>
\end{verbatim}
\afterverb
%
\texttt{readlines} devuelve todas las líneas que queden como una lista de
cadenas:

\beforeverb
\begin{verbatim}
>>> print f.readlines()
['línea dos\012', 'línea tres\012']
\end{verbatim}
\afterverb
%
En este caso, la salida está en forma de lista, lo que significa que
las cadenas aparecen con comillas y el carácter de salto de línea
aparece como la secuencia de escape \texttt{012}.

Al final del archivo, \texttt{readline} devuelve una cadena vacía y
\texttt{readlines} devuelve una lista vacía:

\beforeverb
\begin{verbatim}
>>> print f.readline()

>>> print f.readlines()
[]
\end{verbatim}
\afterverb
%
Lo que sigue es un ejemplo de un programa de proceso de líneas.
\texttt{filtraArchivo} hace una copia de \texttt{archViejo}, omitiendo
las líneas que comienzan por \texttt{\#}:

\beforeverb
\begin{verbatim}
def filtraArchivo(archViejo, archNuevo):
  f1 = open(archViejo, "r")
  f2 = open(archNuevo, "w")
  while 1:
    texto = f1.readline()
    if texto == "":
      break
    if texto[0] == '#':
      continue
    f2.write(texto)
  f1.close()
  f2.close()
  return
\end{verbatim}
\afterverb
%
La sentencia \texttt{continue} termina la iteración actual del ciclo,
pero sigue haciendo las que le faltan. El flujo de ejecución pasa al principio
del ciclo, comprueba la condición y continúa normalmemente.

\index{sentencia continue}
\index{sentencia!continue}

Así, si \texttt{texto} es una cadena vacía, el ciclo termina. Si el primer
caracter de \texttt{texto} es una almohadilla \texttt{(\#)}, el flujo de ejecución va al
principio del ciclo. Sólo si ambas condiciones fallan copiamos \texttt{texto}
en el archivo nuevo.


\section{Escribir variables}
\index{operador de formato}
\index{cadena de formato}
\index{operador!formato}

El argumento de \texttt{write} debe ser una cadena, así que si queremos
poner otros valores en un archivo, tenemos que convertirlos previamente en
cadenas. La forma más fácil de hacerlo es con la función \texttt{str}:

\beforeverb
\begin{verbatim}
>>> x = 52
>>> f.write (str(x))
\end{verbatim}
\afterverb
%
Una alternativa es usar el {\bf operador de formato} \texttt{\%}. Cuando
aplica a enteros, \texttt{\%} es el operador de módulo. Pero cuando el
primer operando es una cadena, \texttt{\%} es el operador de formato.

El primer operando es la {\bf cadena de formato}, y el segundo,
una tupla de expresiones. El resultado es una cadena que contiene los
valores de las expresiones, formateados de acuerdo con la cadena de formato.

A modo de ejemplo simple, la {\bf secuencia de formato} {\verb+"%d"+}
significa que la primera expresión de la tupla debería formatearse como
un entero. Aquí la letra {\em d} quiere decir "decimal":

\beforeverb
\begin{verbatim}
>>> motos = 52
>>> "%d" % motos
'52'
\end{verbatim}
\afterverb
%
El resultado es la cadena \texttt{'52'}, que no debe confundirse con el valor
entero \texttt{52}.

Una secuencia de formato puede aparecer en cualquier lugar de la cadena de
formato, de modo que podemos incrustar un valor en una frase:

\beforeverb
\begin{verbatim}
>>> motos = 52
>>> "En julio vendimos %d motos." % motos
'En julio vendimos 52 motos.'
\end{verbatim}
\afterverb
%
La secuencia de formato {\verb+"%f"+} formatea el siguiente elemento
de la tupla como un número en punto flotante, y {\verb+"%s"+} formatea
el siguiente elemento como una cadena:

\beforeverb
\begin{verbatim}
>>> "En %d días ganamos %f millones de %s." % (4,1.2,'pesos')
'En 34 días ganamos 1.200000 millones de pesos.'
\end{verbatim}
\afterverb
%
Por defecto, el formato de punto flotante imprime seis decimales.

El número de expresiones en la tupla tiene que coincidir con el número
de secuencias de formato de la cadena. Igualmente, los tipos de las
expresiones deben coincidir con las secuencias de formato:

\index{error en tiempo de ejecución}

\beforeverb
\begin{verbatim}
>>> "%d %d %d" % (1,2)
TypeError: not enough arguments for format string
>>> "%d" % 'dólares'
TypeError: illegal argument type for built-in operation
\end{verbatim}
\afterverb
%
En el primer ejemplo no hay suficientes expresiones; en el
segundo, la expresión es de un tipo incorrecto.

Para tener más control sobre el formato de los números, podemos detallar
el número de dígitos como parte de la secuencia de formato:

\beforeverb
\begin{verbatim}
>>> "%6d" % 62
'    62'
>>> "%12f" % 6.1
'    6.100000'
\end{verbatim}
\afterverb
%
El número tras el signo de porcentaje es el número mínimo de espacios
que ocupará el número. Si el valor necesita menos dígitos, se añaden
espacios en blanco delante del número. Si el número de espacios es
negativo, se añaden los espacios tras el número:

\beforeverb
\begin{verbatim}
>>> "%-6d" % 62
'62    '
\end{verbatim}
\afterverb
%
También podemos especificar el número de decimales para los
números en coma flotante:

\beforeverb
\begin{verbatim}
>>> "%12.2f" % 6.1
'        6.10'
\end{verbatim}
\afterverb
%
En este ejemplo, el resultado ocupa doce espacios e incluye dos dígitos
tras la coma. Este formato es útil para imprimir cantidades de dinero con
las comas alineadas.

\index{diccionario}

Imagine, por ejemplo, un diccionario que contiene los nombres
de los estudiantes como clave y las tarifas horarias como valores.
He aquí una función que imprime el contenido del diccionario como
de un informe formateado:

\beforeverb
\begin{verbatim}
def informe (tarifas) :
  estudiantes = tarifas.keys()
  estudiantes.sort()
  for estudiante in estudiantes :
    print "%-20s %12.02f"%(estudiante, tarifas[estudiante])
\end{verbatim}
\afterverb
%
Para probar la función, crearemos un pequeño diccionario e
imprimiremos el contenido:

\beforeverb
\begin{verbatim}
>>> tarifas = {'maría': 6.23, 'josé': 5.45, 'jesús': 4.25}
>>> informe (tarifas)
josé                         5.45
jesús                        4.25
maría                        6.23
\end{verbatim}
\afterverb
%
Controlando el ancho de cada valor nos aseguramos de que las columnas
van a quedar alineadas, siempre que los nombre tengan menos de veintiún
carácteres y las tarifas sean menos de mil millones la hora.


\section{Directorios}
\index{directorio}

Cuando se crea un archivo nuevo abriéndolo y escribiendo, este 
va a quedar en el directorio en uso (aquél en el que se estuviese al ejecutar
el programa). Del mismo modo, cuando se abre un archivo para leerlo,
Python lo busca en el directorio en uso.

Si usted quiere abrir un archivo de cualquier otro sitio, tiene que
especificar la {\bf ruta} del archivo, que es el nombre del directorio
(o carpeta) donde se encuentra este:

\beforeverb
\begin{verbatim}
>>>   f = open("/usr/share/dict/words","r")
>>>   print f.readline()
Aarhus
\end{verbatim}
\afterverb
%
Este ejemplo abre un archivo denominado \texttt{words}, que se encuentra en un
directorio llamado \texttt{dict}, que está en \texttt{share}, en
en \texttt{usr}, que está en el directorio de nivel superior del
sistema, llamado \texttt{/}.

\index{ruta}
\index{delimitador}

No se puede usar \texttt{/} como parte del nombre de un archivo; está
reservado como delimitador entre nombres de archivo y directorios.

El archivo \texttt{/usr/share/dict/words} contiene una lista de palabras
en orden alfabético, la primera de las cuales es el nombre de una
universidad danesa.


\section{Encurtido}
\index{encurtido}

Para poner valores en un archivo,  se deben convertir a cadenas. 
Usted ya ha visto cómo hacerlo con \texttt{str}:

\beforeverb
\begin{verbatim}
>>> f.write (str(12.3))
>>> f.write (str([1,2,3]))
\end{verbatim}
\afterverb
%
El problema es que cuando se vuelve a leer el valor, se obtiene una cadena.
Se ha perdido la información del tipo de dato original. En realidad, no
se puede distinguir dónde termina un valor y dónde comienza el siguiente:

\beforeverb
\begin{verbatim}
>>>   f.readline()
'12.3[1, 2, 3]'
\end{verbatim}
\afterverb
%
La solución es el {\bf encurtido}, llamado así porque ``encurte''
estructuras de datos. El módulo \texttt{pickle} contiene las órdenes
necesarias. Para usarlo, se importa \texttt{pickle} y luego se abre 
el archivo de la forma habitual:

\beforeverb
\begin{verbatim}
>>> import pickle
>>> f = open("test.pck","w")
\end{verbatim}
\afterverb
%
Para almacenar una estructura de datos, se usa el método \texttt{dump}
y luego se cierra el archivo de la forma habitual:

\beforeverb
\begin{verbatim}
>>> pickle.dump(12.3, f)
>>> pickle.dump([1,2,3], f)
>>> f.close()
\end{verbatim}
\afterverb
%
Ahora podemos abrir el archivo para leer y cargar las estructuras
de datos que volcamos ahí:

\beforeverb
\begin{verbatim}
>>> f = open("test.pck","r")
>>> x = pickle.load(f)
>>> x
12.3
>>> type(x)
<type 'float'>
>>> y = pickle.load(f)
>>> y
[1, 2, 3]
>>> type(y)
<type 'list'>
\end{verbatim}
\afterverb
%
Cada vez que invocamos \texttt{load} obtenemos un valor del archivo
completo con su tipo original.


\section{Excepciones}
\index{sentencia try}
\index{sentencia!try}
\index{lanzar una excepción}
\index{manejar una excepción}
\index{sentencia except}
\index{sentencia!except}
\index{excepción}

Siempre que ocurre un error en tiempo de ejecución, se crea una
{\bf excepción}. Normalmente el programa se para y Python
presenta un mensaje de error.

Por ejemplo, la división por cero crea una excepción:

\beforeverb
\begin{verbatim}
>>> print 55/0
ZeroDivisionError: integer division or modulo
\end{verbatim}
\afterverb
%
Un elemento no existente en una lista hace lo mismo:

\beforeverb
\begin{verbatim}
>>> a = []
>>> print a[5]
IndexError: list index out of range
\end{verbatim}
\afterverb
%
O el acceso a una clave que no está en el diccionario:

%\beforeverb
\begin{verbatim}
>>> b = {}
>>> print b['qué']
KeyError: qué
\end{verbatim}
%\afterverb
%
En cada caso, el mensaje de error tiene dos partes: el tipo
de error antes de los dos puntos y detalles sobre el error después
de los dos puntos. Normalmente, Python también imprime una traza de
dónde se encontraba el programa, pero la hemos omitido en los ejemplos.

\index{traza}

A veces queremos realizar una operación que podría provocar una
excepción, pero no queremos que se pare el programa. Podemos
{\bf manejar} la excepción usando las sentencias \texttt{try} y
\texttt{except}.

Por ejemplo, podemos preguntar al usuario por el nombre de un archivo
y luego intentar abrirlo. Si el archivo no existe, no queremos que el
programa se aborte; queremos manejar la excepción.

%\beforeverb
\begin{verbatim}
nombreArch = raw_input('Introduce un nombre de archivo: ')
try:
  f = open (nombreArch, "r")
except:
  print 'No hay ningún archivo que se llame', nombreArch
\end{verbatim}
%\afterverb
%
La sentencia \texttt{try} ejecuta las sentencias del primer bloque.
Si no se produce ninguna excepción, pasa por alto la sentencia
\texttt{except}. Si ocurre cualquier excepción, ejecuta las sentencias
de la rama \texttt{except} y después continúa.

Podemos encapsular esta capacidad en una función: \texttt{existe}, que acepta un
nombre de archivo y devuelve verdadero si el archivo existe y falso si no:

%\pagebreak
%\beforeverb
\begin{verbatim}
def existe(nombreArch):
  try:
    f = open(nombreArch)
    f.close()
    return True
  except:
    return False
\end{verbatim}
%\afterverb
%
Se pueden usar múltiples bloques \texttt{except} para manejar diferentes tipos
de excepciones. El {\em Manual de Referencia de Python} contiene los detalles.


Si su programa detecta una condición de error, se puede lanzar 
({\bf raise} en inglés) una excepción. Aquí hay un ejemplo que acepta una entrada
del usuario y comprueba si es 17. Suponiendo que 17 no es una entrada válida
por cualquier razón, lanzamos una excepción.

%\beforeverb
\begin{verbatim}
def tomaNumero () :                 
  # ¡Recuerda, los acentos están prohibidos
  x = input ('Elige un número: ')   
  # en los nombres de funciones y variables!
  if x == 17 :
    raise 'ErrorNúmeroMalo', '17 es un mal número'
  return x
\end{verbatim}
%\afterverb
%
La sentencia \texttt{raise} acepta dos argumentos: el tipo de excepción
e información específica acerca del error. \texttt{ErrorNúmeroMalo}
es un nuevo tipo de excepción que hemos inventado para esta aplicación.

Si la función llamada \texttt{tomaNumero} maneja el error, el programa
puede continuar; en caso contrario, Python imprime el mensaje de error
y sale:

\beforeverb
\begin{verbatim}
>>> tomaNumero ()
Elige un número: 17
ErrorNúmeroMalo: 17 es un mal número
\end{verbatim}
\afterverb
%
El mensaje de error incluye el tipo de excepción y la información
adicional proporcionada.

\begin{quote}
{\em Como ejercicio, escribe una función que use \texttt{tomaNumero}
para leer un número del teclado y que maneje la excepción
\texttt{ErrorNumeroMalo}.}
\end{quote}


\section{Glosario}

\index{archivo}
\index{archivo de texto}
\index{sentencia break}
\index{sentencia!break}
\index{sentencia continue}
\index{sentencia!continue}
\index{operador de formato}
\index{cadena de formato}
\index{operador!formato}
\index{directorio}
\index{encurtido}
\index{try}
\index{lanzar excepción}
\index{manejar excepción}
\index{sentencia except}
\index{excepción}

\begin{description}

\item[Archivo:] entidad con nombre, normalmente almacenada en un disco
duro, disquete o CD-ROM, que contiene una secuencia de carácteres.

\item[Directorio:] colección de archivos, con nombre, también llamado carpeta.

\item[Ruta:] secuencia de nombres de directorio que especifica la localización
exacta de un archivo.

\item[Archivo de texto:] un archivo que contiene carácteres imprimibles organizados
en líneas separadas por carácteres de salto de línea.

\item[Sentencia break:] es una sentencia que provoca que el flujo de ejecución salga de
un ciclo.

\item[Sentencia continue:] sentencia que provoca que termine la
iteración actual de un ciclo. El flujo de la ejecución va al principio
del ciclo, evalúa la condición, y procede en consecuencia.

\item[Operador de formato:] el operador \texttt{\%} toma una cadena de
formato y una tupla de expresiones y entrega una cadena que incluye
las expresiones, formateadas de acuerdo con la cadena de formato.

\item[Cadena de formato:] una cadena que contiene carácteres imprimibles
y secuencias de formato que indican cómo dar formato a valores.

\item[Secuencia de formato:] secuencia de carácteres que comienza con
\texttt{\%} e indica cómo dar formato a un valor.

\item[Encurtir:] escribir el valor de un dato en un archivo junto con la
información sobre su tipo de forma que pueda ser reconstituido más tarde.

\item[Excepción:] error que ocurre en tiempo de ejecución.

\item[Manejar:] impedir que una excepción detenga un programa utilizando
las sentencias \texttt{except} y \texttt{try}.

% \item[Lanzar:] crausar una excepción usando la sentencia \texttt{raise}.


\end{description}

